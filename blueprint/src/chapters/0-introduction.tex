\chapter*{Introduction}
% \label{chap:introduction}
\addcontentsline{toc}{chapter}{Introduction}

% https://formal-mathematics.github.io/intro.html

% https://github.com/pitmonticone/FLT3

% https://pitmonticone.github.io/FLT3/docs/

% https://pitmonticone.github.io/FLT3/blueprint

% https://pitmonticone.github.io/FLT3/blueprint/dep_graph_document.html

% https://pitmonticone.github.io/FLT3/blueprint.pdf


% What is Formalisation / a proof assistant / interactive theorem prover? Why is it important?

% Proof assistants, also known as interactive theorem provers, are tools for producing formally correct mathematics.
% Mathematicians write definitions, theorems, and proofs in a specialized language, with the computer giving instant feedback about each line.
% Examples of popular proof assistants include Agda, Coq, HOL Light, Isabelle, Lean, and Mizar.
% These tools have been used to develop large libraries of "known" mathematics as well as to verify major recent or controversial results.
% The act of formalizing a proof is not mechanical: it involves substantial mathematical insight in its own right,
% and it constitutes the essence of formalized mathematics, which is now recognized as a mathematical discipline on its own.
% Despite its young age, this discipline is blossoming at a very fast pace; it nowadays occupies a significant place in the mathematical horizon.

% Related to proof assistants, automated reasoning tools take in the formal statement of a theorem and attempt to prove it without further human input. While the study of these tools themselves is the domain of computer science, their use on mathematical theorems requires sophisticated reductions and reformulations.


% What is Lean? Why is it important?
%% interactive theorem prover and general-purpose programming language
% What is Mathlib? Why is it important?

% Context, Motivation, Basic Definitions?

% MOTIVATION
% It's going to be used in the general FLT project coordinated by Kevin Buzzard https://github.com/ImperialCollegeLondon/FLT

% NOTATION
% number sets

% MAIN DEFINITIONS AND STRUCTURES?

% GOAL / AIM

% CONTRIBUTIONS
% - About 20 formal / formalised proofs
% - Code:
% - All informal / informalised proofs
% - all blueprint, PRs to Mathlib (https://github.com/leanprover-community/mathlib4/pull/11677, https://github.com/leanprover-community/mathlib4/pull/11695), Porting to Mathlib (https://github.com/leanprover-community/mathlib4/pull/11767), see LaTeX comments, talk, ...
% Code refactoring


% PROOF STRATEGY

% Motivation: It's going to be used in the general FLT project
% – Disclaimer / Acknowledgement with my contributions
% – Informalisation should be clear
% – Introduction: definitions + theorem statements + proof strategy
% Porting to Mathlib
% Fermat's Last Theorem (FLT) is a famous problem in number theory. It states that there are no positive integers $a$, $b$, and $c$ such that $a^n + b^n = c^n$ for any integer value of $n$ greater than $2$. The theorem was first conjectured by Pierre de Fermat in 1637 and remained unproven for over 350 years. The first successful proof was given by Andrew Wiles in 1994. The proof is long and complex, and it relies on many different areas of mathematics, including algebraic geometry and modular forms.


% \section*{Motivation}
% \section*{Contributions}
% \section*{Overview / Structure}

% https://github.com/pitmonticone/FLT3/graphs/contributors

% SEE FLT PROJECT