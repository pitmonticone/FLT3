\chapter*{Introduction}
% \label{chap:introduction}
\addcontentsline{toc}{chapter}{Introduction}

Fermat's Last Theorem (FLT) is one of the most renowned results in the history of mathematics.
Originally conjectured by Pierre de Fermat in 1637, the theorem asserts that there are no three
positive integers $a$, $b$, and $c$ that can satisfy the equation $a^n + b^n = c^n$ for any
integer value of $n$ greater than $2$. This deceptively simple statement eluded proof for over
350 years until it was finally proven by Andrew Wiles in 1994 using sophisticated methods from
algebraic geometry and number theory.

This paper serves as the blueprint for the project aimed at formalising FLT for the specific case of
exponent $3$ using the Lean 4 proof assistant.
The project consists in the formalisation of all necessary definitions, lemmas, and theorems leading up to
and including the proof of FLT for $n = 3$. This process involves translating natural, informal mathematical arguments
into the formal language of Lean, rigorously verifying each step for correctness and leveraging
an extensive library of formalised mathematics called Mathlib,
which will be described in more detail later.

The \textbf{source code} necessary to understand and verify the formal proofs of this project is contained
in my GitHub repository available at \url{https://github.com/pitmonticone/FLT3}. It is being integrated into
Mathlib and imported as a dependency in the broader FLT formalisation project led by Kevin Buzzard and Richard Taylor \cite{FLT}.

For detailed and comprehensive \textbf{documentation} of the code, please refer to the documentation website at \url{https://pitmonticone.github.io/FLT3/docs/}.

The \textbf{blueprint website}, which includes a detailed roadmap of the steps taken to formalise the proof,
can be accessed at \url{https://pitmonticone.github.io/FLT3/blueprint}.

To view the \textbf{dependency graph} of the blueprint, which visually represents the logical dependencies
between definitions and theorems, visit \url{https://pitmonticone.github.io/FLT3/blueprint/dep_graph_document.html}.

Lastly, a public version of this \textbf{blueprint paper} is available at \url{https://pitmonticone.github.io/FLT3/blueprint.pdf}.

\section{What is Formalisation of Mathematics?}

Formalisation of mathematics is the process of encoding mathematical concepts, statements, and proofs
in a formal language that can be interpreted and verified by a computer.
This contrasts with traditional mathematical writing, which often relies on a combination of formal
symbols and natural language \cite{Bourbaki2004} \cite{Avigad2023}.

Formalisation aims to eliminate ambiguity and ensure absolute rigor by using logical and computational
techniques to validate claims articulated with extreme precision \cite{Avigad2014} \cite{Avigad2018} \cite{Avigad2024}.
Every proof step must be justified by referring to prior definitions and theorems, tracing all the way
back to fundamental axioms and deduction rules.

As we know, the highest standard for justifying a mathematical statement is to provide a proof and modern
mathematical logic have shown that most, if not all, conventional
proof methods can be distilled into a limited set of axioms and inference rules within various
foundational formal systems \cite{Avigad2022}. This reduction allows computers to assist human mathematicians in two significant ways:

\begin{itemize}
    \item \textbf{Interactive Theorem Proving:} Verifying the validity of proposed proofs by checking each logical step,
    as demonstrated in this project. The strict syntactic and semantic rules of formal systems allow
    computers to check the correctness of proofs automatically, ensuring that no logical errors are present.
    \item \textbf{Automated Theorem Proving:} Aiding in the discovery of proofs by establishing the validity
    of first-order formulas (e.g. satisfiability solvers), by adopting clever search methods (e.g. satisfiability modulo theories)
    and by carrying out mathematical computations (e.g. computer algebra systems).
\end{itemize}

At the current stage, formalising a proof requires significant mathematical insight and is not a purely mechanical process.
This endeavor is central to the field of formalised mathematics, a rapidly growing discipline now acknowledged
as a distinct area of study.

\section{Why Formalise Mathematics?}

Formalising mathematics offers several significant benefits \cite{Massot2021}, such as:

\begin{itemize}
    \item \textbf{Certainty:} Formal proofs provide a higher level of certainty by rigorously verifying each logical step through computer checks, reducing the possibility of human error.
    \item \textbf{Consistency:} Formal systems ensure all results are internally consistent, preventing subtle errors from informal reasoning and ensuring coherence across different mathematical domains.
    \item \textbf{Clarity:} Formal proofs help clarify complex arguments by making all steps explicit and precise, improving communication and understanding of mathematical results.
    \item \textbf{Generalisation:} Formalised mathematics makes it much easier to generalise results, as the precise definitions and proofs can be systematically extended to broader contexts.
    \item \textbf{Reusability:} Formalised mathematics can be reused across various proofs and projects, building a repository of verified knowledge that can be easily accessed and applied.
    \item \textbf{Collaboration:} Formalised mathematics supports better collaboration among mathematicians by providing precise, unambiguous statements and facilitating easier delegation and verification of tasks.
    \item \textbf{Education:} The process of formalisation aids in teaching and learning by providing clear, detailed examples of rigorous reasoning and logical structure.
    \item \textbf{Research:} Formal methods assist in discovering new results and verifying complex proofs, particularly those involving extensive computation or intricate logical structures.
    \item \textbf{Peer Review:} The use of formal methods in peer review raises the standards of verification, making it easier to identify errors and increasing confidence in the validity of published results.
\end{itemize}

These benefits collectively contribute to the advancement of mathematical theory and practice,
making formalisation an invaluable tool in modern mathematical research and teaching.

\section{What is a Proof Assistant?}

Proof assistants, also known as interactive theorem provers (ITPs), are sophisticated software tools
designed to aid mathematicians in developing formal proofs. They provide an environment where users
can write definitions, statements, and proofs in a specialized language, with the computer checking
each step for correctness and providing instant feedback on every line.

Proof assistants combine automation with human guidance. They can automate routine tasks such as checking
the validity of logical steps and applying known theorems, but they require human input for more complex
reasoning and creative problem-solving.

Examples of popular proof assistants include \href{https://coq.inria.fr}{Coq}, \href{https://isabelle.in.tum.de}{Isabelle},
and \href{https://lean-lang.org}{Lean}. Each has its strengths and focuses, but all share the goal of
making rigorous, computer-verified mathematics a practical reality.

\newpage
\section{What is Lean?}

Lean is an open-source, general-purpose functional programming language and proof assistant
based on a version of dependent type theory known as the \textit{Calculus of Inductive Constructions} \cite{Carneiro2019} and
developed by Leonardo de Moura and Sebastian Ullrich, originally as part of Microsoft Research \cite{Moura2021} \cite{Moura2024}.

It is designed to facilitate both programming and formal verification, making it a versatile tool
for mathematicians and computer scientists, and distinguishes itself with its small inference kernel,
strong automation, and the availability of independent proof checkers which provide additional guarantees.

Lean's dependently-typed system allows for highly expressive and precise definitions, which is crucial
for both programming and the formalisation of complex mathematical concepts.
This feature enables Lean to handle a wide range of applications, from software development to advanced mathematical proofs.

One of Lean's most powerful features is its sophisticated metaprogramming framework \cite{Ebner2017}.
This framework allows users to extend the language by writing custom proof tactics, which can automate
repetitive and complex parts of the theorem proving process.
These tactics enable users to streamline their proof development, making the formalisation process
more efficient and less error-prone.

The metaprogramming capabilities of Lean are not limited to proof tactics.
Users can also develop new language constructs, define custom syntactic sugar,
and implement domain-specific optimizations. This flexibility makes Lean an adaptable tool that can
evolve to meet the needs of various formal verification tasks.

In addition to its technical capabilities, Lean benefits from a strong and active community of users and contributors.
This community continuously expands Lean's libraries and tools, ensuring that the system remains up-to-date with
the latest developments in both mathematics and computer science.
The collaborative nature of the Lean ecosystem fosters innovation and supports the development of
high-quality formalisations.

Overall, Lean's combination of a robust type system, advanced metaprogramming features, and a supportive
community makes it a powerful and flexible platform for both programming and formal verification.

\newpage
\section{What is Mathlib?}

Mathlib is the main library of formalised mathematics written in Lean \cite{Mathlib2020}.
It is a collaborative project that contains formalisations of a wide range of mathematical concepts,
from basic definitions to advanced theorems.

Key aspects of Mathlib include:
\begin{itemize}
    \item \textbf{Coverage:} Mathlib covers many areas of mathematics, including algebra, analysis, topology, and number theory.
    \item \textbf{Collaboration:} Mathlib is developed by a community of hundreds of mathematicians and computer scientists, ensuring that it is continually expanding and improving \cite{vanDoorn2020}.
    \item \textbf{Quality:} All formalisations in Mathlib are rigorously checked by Lean.
    \item \textbf{Accessibility:} Mathlib is open-source and freely available, making it accessible to anyone interested in formalising mathematics.
    \item \textbf{Blueprints:} The Mathlib community uses blueprints to outline the steps needed to formalise complex results, providing a roadmap for contributors.
\end{itemize}

% \section{Personal Contributions}

% My personal contributions to this collaborative project include:

% \begin{itemize}
%     \item \textbf{Source Code:} I have written approximately 20\% of the \href{https://github.com/pitmonticone/FLT3}{source code} of the project.
%     \item \textbf{Code Documentation:} I have written the entire \href{https://pitmonticone.github.io/FLT3/docs/}{documentation} of the code.
%     \item \textbf{Blueprint Website:} I have developed the entire \href{https://pitmonticone.github.io/FLT3/blueprint}{blueprint website} including the \href{https://pitmonticone.github.io/FLT3/blueprint/dep_graph_document.html}{dependency graph}.
%     \item \textbf{Blueprint Paper:} I have authored the entire \href{https://pitmonticone.github.io/FLT3/blueprint.pdf}{blueprint paper}.
% \end{itemize}

% For more details about my contributions, please visit the \href{https://github.com/pitmonticone/FLT3/graphs/contributors}{contributors page}
% and the \href{https://github.com/pitmonticone/FLT3/commits/master/?author=pitmonticone}{commit history} of the project.