\chapter*{Introduction}
%\label{chap:introduction}
\addcontentsline{toc}{chapter}{Introduction}


% https://github.com/pitmonticone/FLT3

% https://pitmonticone.github.io/FLT3/docs/

% https://pitmonticone.github.io/FLT3/blueprint

% https://pitmonticone.github.io/FLT3/blueprint/dep_graph_document.html

% https://pitmonticone.github.io/FLT3/blueprint.pdf


% What is Formalisation? Why is it important?
% What is Lean? Why is it important?
% What is Mathlib? Why is it important?

% Context, Motivation, Basic Definitions?

% MOTIVATION
% It's going to be used in the general FLT project coordinated by Kevin Buzzard

% NOTATION
% number sets

% MAIN DEFINITIONS AND STRUCTURES?

% GOAL / AIM

% CONTRIBUTIONS
% - About 20 formal / formalised proofs
% - Code:
% - All informal / informalised proofs
% - all blueprint, PRs to Mathlib (https://github.com/leanprover-community/mathlib4/pull/11677, https://github.com/leanprover-community/mathlib4/pull/11695), Porting to Mathlib (https://github.com/leanprover-community/mathlib4/pull/11767), see LaTeX comments, talk, ...
% Code refactoring


% PROOF STRATEGY

% Motivation: It's going to be used in the general FLT project
% – Disclaimer / Acknowledgement with my contributions
% – Informalisation should be clear
% – Introduction: definitions + theorem statements + proof strategy
% Porting to Mathlib
% Fermat's Last Theorem (FLT) is a famous problem in number theory. It states that there are no positive integers $a$, $b$, and $c$ such that $a^n + b^n = c^n$ for any integer value of $n$ greater than $2$. The theorem was first conjectured by Pierre de Fermat in 1637 and remained unproven for over 350 years. The first successful proof was given by Andrew Wiles in 1994. The proof is long and complex, and it relies on many different areas of mathematics, including algebraic geometry and modular forms.


% \section*{My Contributions}

% https://github.com/pitmonticone/FLT3/graphs/contributors