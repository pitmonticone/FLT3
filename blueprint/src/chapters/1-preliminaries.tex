\chapter{Preliminaries}
%\label{chap:preliminaries}
%\addcontentsline{toc}{chapter}{Preliminaries}

\section{Notation}

\begin{center}
    \begin{tabular}{>{\centering}m{1.5cm} m{6cm}}
    \toprule
    \textbf{Symbol} & \multicolumn{1}{c}{\textbf{Description}} \\
    \midrule
    $\lnot$ & Logical negation \\
    $\top$  & Logical truth / Tautology \\
    $\bot$  & Logical falsehood / Contradiction \\
    $\land$ & Logical conjunction \\
    $\lor$ & Logical inclusive disjunction \\
    $:=$   & Definition \\
    $\forall$ & Universal quantification \\
    $\exists$ & Existential quantification \\
    $\exists!$ & Unique existential quantification \\
    $\N$ & Set of natural numbers \\
    $\Z$ & Set of integer numbers \\
    $\Z_n$ & Set of integers modulo $n$ \\
    $\Q$ & Set of rational numbers \\
    $[n]$ & Equivalence class of $n$ \\
    $\divides$ & Divisibility relation \\
    $\notdivides$ & Non-divisibility relation \\
    $\gcd$ & Greatest common divisor \\
    $\zeta_n$ & Primitive $n$-th root of unity \\
    \bottomrule
    \end{tabular}
\end{center}

\newpage

\section{Definitions}

\begin{definition}[Monoid]
    \label{def:monoid}
    Let \(X\) be a non-empty set.\\
    Let \(\circ: X\times X \to X\) be an internal composition law on \(X\). \\\\
    A \textit{monoid} is a pair \(\cc{M}:= (X, \circ)\) satisfying:
    \begin{enumerate}
        \item [\textbf{(A)}] \(\forall x,y,z\in X,\ (x\circ y)\circ z= x\circ (y\circ z) = x\circ y \circ z\)
        \item [\textbf{(N)}] \(\exists e \in X : \forall x \in X,\ x\circ e = e \circ x = x\)
    \end{enumerate}
\end{definition}

\begin{definition}[Commutative Monoid]
    \label{def:commutative_monoid}
    Let \(X\) be a non-empty set.\\
    Let \(\circ: X\times X \to X\) be an internal composition law on \(X\). \\\\
    A \textit{commutative monoid} is a pair \(\cc{M}_a:= (X, \circ)\) satisfying:
    \begin{enumerate}
        \item [\textbf{(A)}] \(\forall x,y,z\in X,\ (x\circ y)\circ z= x\circ (y\circ z) = x\circ y \circ z\)
        \item [\textbf{(N)}] \(\exists e \in X : \forall x \in X,\ x\circ e = e \circ x = x\)
        \item [\textbf{(C)}] \(\forall x,y\in X,\ x\circ y = y\circ x\)
    \end{enumerate}
\end{definition}

\begin{definition}[Group]
    \label{def:group}
    Let \(X\) be a non-empty set.\\
    Let \(\circ: X\times X \to X\) be an internal composition law on \(X\). \\\\
    A \textit{group} is a pair \(\cc{G}:= (X, \circ)\) satisfying:
    \begin{enumerate}
        \item [\textbf{(A)}] \(\forall x,y,z\in X,\ (x\circ y)\circ z= x\circ (y\circ z) = x\circ y \circ z\)
        \item [\textbf{(N)}] \(\exists e \in X : \forall x \in X,\ x\circ e = e \circ x = x\)
        \item [\textbf{(I)}] \(\forall x \in X,\ \exists x'\in X: x\circ x' = x'\circ x = e\)
    \end{enumerate}
\end{definition}

\begin{definition}[Commutative Group]
    \label{def:commutative_group}
    Let \(X\) be a non-empty \bdefref{set}{set}.\\
    Let \(\circ: X\times X \to X\) be an \bdefref{internal-composition-law}{internal composition law} on \(X\). \\\\
    A \textit{commutative group} is a pair \(\cc{G}_a:= (X, \circ)\) satisfying:
    \begin{enumerate}
        \item [\textbf{(A)}] \(\forall x,y,z\in X,\ (x\circ y)\circ z= x\circ (y\circ z) = x\circ y \circ z\)
        \item [\textbf{(N)}] \(\exists e \in X : \forall x \in X,\ x\circ e = e \circ x = x\)
        \item [\textbf{(I)}] \(\forall x \in X,\ \exists x'\in X: x\circ x' = x'\circ x = e\)
        \item [\textbf{(C)}] \(\forall x,y\in X,\ x\circ y = y\circ x\)
    \end{enumerate}
\end{definition}

\begin{definition}[Semiring]
    \label{def:semiring}
\end{definition}

\begin{definition}[Commutative Semiring]
    \label{def:commutative_semiring}
\end{definition}

\begin{definition}[Ring]
    \label{def:ring}
\end{definition}

\begin{definition}[Field]
    \label{def:field}
\end{definition}
\begin{definition}[Algebraic Number Field / Number Field]
    \label{def:number_field}
\end{definition}

\begin{definition}[Fundamental System]
    \label{def:fundamental_system}
\end{definition}

\begin{definition}[Unit]
    \label{def:primitive_element}
\end{definition}
\begin{definition}[Primitive Root of Unity]
    \label{def:primitive_root_of_unity}
\end{definition}

\begin{definition}[Cyclotomic Polynomial]
    \label{def:cyclotomic_polynomial}
\end{definition}
\begin{definition}[Cyclotomic Extension of a Ring / Cyclotomic Ring]
    \label{def:cyclotomic_extension_field}
\end{definition}
\begin{definition}[Cyclotomic Extension of a Fiel / Cyclotomic Field]
    \label{def:cyclotomic_extension_field}
\end{definition}

\section{Results}

\begin{theorem}
    \label{thm:zeta_sub_one_prime1}
    \lean{IsPrimitiveRoot.zeta_sub_one_prime'}
    \leanok
    Let $p \in \N$ be prime. \\\\
    If $\zeta_p$ is a primitive $p$-th root of unity, then $\zeta_p - 1$ is prime.
\end{theorem}
\begin{proof}
    \leanok
    This has already been formalised and included in \href{https://pitmonticone.github.io/FLT3/docs/FLT3/Mathlib/NumberTheory/Cyclotomic/Rat.html#IsPrimitiveRoot.zeta_sub_one_prime'}{Mathlib}.
\end{proof}

\begin{lemma}
    \label{lmm:fermatLastTheoremWith_of_fermatLastTheoremWith_coprime}
    \lean{fermatLastTheoremWith_of_fermatLastTheoremWith_coprime}
    \leanok
    Let $R$ be a commutative semiring, domain and normalised $\gcd$ monoid.\\% ASK EXPERTS
    Let $a, b, c \in R$. \\
    Let $n \in \N$. \\\\
    Then, to prove Fermat's Last Theorem for exponent $n$ in $R$,
    one can assume, without loss of generality, that $\gcd(a,b,c)=1$.
  \end{lemma}
  \begin{proof}
    \leanok
    This has already been formalised and included in \href{https://pitmonticone.github.io/FLT3/docs/FLT3/Mathlib/NumberTheory/FLT/Basic.html#fermatLastTheoremWith_of_fermatLastTheoremWith_coprime}{Mathlib}.
  \end{proof}

  \begin{lemma}
    \label{lmm:cube_of_castHom_ne_zero}
    \lean{cube_of_castHom_ne_zero}
    \leanok
    Let $\Z_9$ be the ring of integers modulo $9$. \\
    Let $\Z_3$ be the ring of integers modulo $3$. \\
    Let $n \in \Z_9$. \\
    Let $\phi : \Z_9 \to \Z_3$ be the canonical ring homomorphism. \\
    Let $\phi(n) \neq 0$. \\ \\
    Then $n^3=1 \lor n^3=8$.
  \end{lemma}
  \begin{proof}
    \leanok
    This has already been formalised and included in \href{https://pitmonticone.github.io/FLT3/docs/FLT3/Mathlib/NumberTheory/FLT/Three.html#cube_of_castHom_ne_zero}{Mathlib}.
  \end{proof}