\chapter{Preliminaries}
%\label{chap:preliminaries}
%\addcontentsline{toc}{chapter}{Preliminaries}

\section{Notation}

% \begin{center}
\begin{tabular}{>{\centering}m{1.8cm} m{5.8cm}}
\toprule
\textbf{Symbol} & \multicolumn{1}{c}{\textbf{Description}} \\
\midrule
$\lnot$ & Logical negation \\
$\top$  & Logical truth / Tautology \\
$\bot$  & Logical falsehood / Contradiction \\
$\land$ & Logical conjunction \\
$\lor$ & Logical inclusive disjunction \\
$:=$   & Definition \\
$\forall$ & Universal quantification \\
$\exists$ & Existential quantification \\
$\exists!$ & Unique existential quantification \\
$\N$ & Set of natural numbers \\
$\Z$ & Set of integer numbers \\
$\Z_n$ & Set of integers modulo $n$ \\
$\Q$ & Set of rational numbers \\
$\bb{F}$ & Generic field \\
$\times$ & Cartesian product \\
$\circ$ & Internal composition law \\
$[n]$ & Equivalence class of $n$ \\
$\divides$ & Divisibility relation \\
$\notdivides$ & Negation of divisibility relation \\
$\gcd$ & Greatest common divisor \\
$\zeta_n$ & Primitive $n$-th root of unity \\
\bottomrule
\end{tabular}
% \end{center}

\newpage

\section{Definitions}

\begin{definition-pre}[Monoid]
    \label{def:monoid}
    Let \(X\) be a non-empty set.\\
    Let \(\circ: X\times X \to X\) be an internal composition law on \(X\). \\\\
    A \textit{monoid} is a pair \(\cc{M}:= (X, \circ)\) satisfying:
    \begin{enumerate}
        \item [\textbf{(A)}] \(\forall x,y,z\in X,\ (x\circ y)\circ z= x\circ (y\circ z) = x\circ y \circ z\)
        \item [\textbf{(N)}] \(\exists e \in X : \forall x \in X,\ x\circ e = e \circ x = x\)
    \end{enumerate}
\end{definition-pre}

\begin{definition-pre}[Commutative Monoid]
    \label{def:commutative_monoid}
    Let \(X\) be a non-empty set.\\
    Let \(\circ: X\times X \to X\) be an internal composition law on \(X\). \\\\
    A \textit{commutative monoid} is a pair \(\cc{M}_c:= (X, \circ)\) satisfying:
    \begin{enumerate}
        \item [\textbf{(A)}] \(\forall x,y,z\in X,\ (x\circ y)\circ z= x\circ (y\circ z) = x\circ y \circ z\)
        \item [\textbf{(N)}] \(\exists e \in X : \forall x \in X,\ x\circ e = e \circ x = x\)
        \item [\textbf{(C)}] \(\forall x,y\in X,\ x\circ y = y\circ x\)
    \end{enumerate}
\end{definition-pre}

\begin{definition-pre}[GCD Monoid]
    \label{def:gcd_monoid}
    Let \(X\) be a non-empty set.\\
    Let \(\circ: X\times X \to X\) be an internal composition law on \(X\). \\\\
    A \textit{gcd monoid} is a pair \(\cc{M}_{\gcd}:= (X, \circ)\) satisfying:
    \begin{enumerate}
        \item [\textbf{(A)}] \(\forall x,y,z\in X,\ (x\circ y)\circ z= x\circ (y\circ z) = x\circ y \circ z\)
        \item [\textbf{(N)}] \(\exists e \in X : \forall x \in X,\ x\circ e = e \circ x = x\)
        \item [\textbf{(C)}] \(\forall x,y\in X,\ x\circ y = y\circ x\)
        \item [\textbf{(G)}] \(\forall x,y\in X,\ \exists d \in X : (d\divides x) \land (d\divides y)
                             \land \left(\forall c \in X,\ c\divides x \land c\divides y \Rightarrow c\divides d\right)\)
    \end{enumerate}
\end{definition-pre}

\begin{definition-pre}[Group]
    \label{def:group}
    Let \(X\) be a non-empty set.\\
    Let \(\circ: X\times X \to X\) be an internal composition law on \(X\). \\\\
    A \textit{group} is a pair \(\cc{G}:= (X, \circ)\) satisfying:
    \begin{enumerate}
        \item [\textbf{(A)}] \(\forall x,y,z\in X,\ (x\circ y)\circ z= x\circ (y\circ z) = x\circ y \circ z\)
        \item [\textbf{(N)}] \(\exists e \in X : \forall x \in X,\ x\circ e = e \circ x = x\)
        \item [\textbf{(I)}] \(\forall x \in X,\ \exists x'\in X: x\circ x' = x'\circ x = e\)
    \end{enumerate}
\end{definition-pre}

\begin{definition-pre}[Commutative Group]
    \label{def:commutative_group}
    Let \(X\) be a non-empty set.\\
    Let \(\circ: X\times X \to X\) be an internal composition law on \(X\). \\\\
    A \textit{commutative group} is a pair \(\cc{G}_c:= (X, \circ)\) satisfying:
    \begin{enumerate}
        \item [\textbf{(A)}] \(\forall x,y,z\in X,\ (x\circ y)\circ z= x\circ (y\circ z) = x\circ y \circ z\)
        \item [\textbf{(N)}] \(\exists e \in X : \forall x \in X,\ x\circ e = e \circ x = x\)
        \item [\textbf{(I)}] \(\forall x \in X,\ \exists x'\in X: x\circ x' = x'\circ x = e\)
        \item [\textbf{(C)}] \(\forall x,y\in X,\ x\circ y = y\circ x\)
    \end{enumerate}
\end{definition-pre}

\begin{definition-pre}[Semiring]
    \label{def:semiring}
    Let \(X\) be a non-empty set. \\
    Let \(+: X\times X \to X\) be an additive internal composition law on \(X\). \\
    Let \(\cdot: X\times X \to X\) be a multiplicative internal composition law on \(X\). \\\\
    A \textit{semiring} is a triple \(\cc{R}:= (X, +, \cdot)\) satisfying:
    \begin{enumerate}
        \item [\textbf{(A1)}] \(\forall x,y,z\in X,\ (x+y)+z= x+(y+z) = x+y+z\)
        \item [\textbf{(C1)}] \(\forall x,y\in X,\ x+y = y+x\)
        \item [\textbf{(N1)}] \(\exists 0 \in X : \forall x \in X,\ x+0 = 0+x = x\)
        \item [\textbf{(A2)}] \(\forall x,y,z\in X,\ (x\cdot y)\cdot z= x\cdot(y\cdot z) = x\cdot y\cdot z\)
        \item [\textbf{(N2)}] \(\exists 1 \in X : \forall x \in X,\ x\cdot1 = 1\cdot x = x\)
        \item [\textbf{(D1)}] \(\forall x,y,z \in X,\ x\cdot(y+z)=x\cdot y+x\cdot z\)
        \item [\textbf{(D2)}] \(\forall x,y,z \in X,\ (x+y)\cdot z=x\cdot z+y\cdot z\)
    \end{enumerate}
\end{definition-pre}
\begin{definition-pre}[Commutative Semiring]
    \label{def:commutative_semiring}
    Let \(X\) be a non-empty set. \\
    Let \(+: X\times X \to X\) be an additive internal composition law on \(X\). \\
    Let \(\cdot: X\times X \to X\) be a multiplicative internal composition law on \(X\). \\\\
    A \textit{commutative semiring} is a triple \(\cc{R}:= (X, +, \cdot)\) satisfying:
    \begin{enumerate}
        \item [\textbf{(A1)}] \(\forall x,y,z\in X,\ (x+y)+z= x+(y+z) = x+y+z\)
        \item [\textbf{(C1)}] \(\forall x,y\in X,\ x+y = y+x\)
        \item [\textbf{(N1)}] \(\exists 0 \in X : \forall x \in X,\ x+0 = 0+x = x\)
        \item [\textbf{(A2)}] \(\forall x,y,z\in X,\ (x\cdot y)\cdot z= x\cdot(y\cdot z) = x\cdot y\cdot z\)
        \item [\textbf{(C2)}] \(\forall x,y\in X,\ x\cdot y = y\cdot x\)
        \item [\textbf{(N2)}] \(\exists 1 \in X : \forall x \in X,\ x\cdot1 = 1\cdot x = x\)
        \item [\textbf{(D1)}] \(\forall x,y,z \in X,\ x\cdot(y+z)=x\cdot y+x\cdot z\)
        \item [\textbf{(D2)}] \(\forall x,y,z \in X,\ (x+y)\cdot z=x\cdot z+y\cdot z\)
    \end{enumerate}
\end{definition-pre}

\begin{definition-pre}[Ring]
    \label{def:ring}
    Let \(X\) be a non-empty set. \\
    Let \(+: X\times X \to X\) be an additive internal composition law on \(X\). \\
    Let \(\cdot: X\times X \to X\) be a multiplicative internal composition law on \(X\). \\\\
    A \textit{ring} is a triple \(\cc{R}:= (X, +, \cdot)\) satisfying:
    \begin{enumerate}
        \item [\textbf{(A1)}] \(\forall x,y,z\in X,\ (x+y)+z= x+(y+z) = x+y+z\)
        \item [\textbf{(C1)}] \(\forall x,y\in X,\ x+y = y+x\)
        \item [\textbf{(N1)}] \(\exists 0 \in X : \forall x \in X,\ x+0 = 0+x = x\)
        \item [\textbf{(I1)}] \(\forall x \in X,\ \exists (-x)\in X: x+(-x) = (-x)+x = 0\)
        \item [\textbf{(A2)}] \(\forall x,y,z\in X,\ (x\cdot y)\cdot z= x\cdot(y\cdot z) = x\cdot y\cdot z\)
        \item [\textbf{(N2)}] \(\exists 1 \in X : \forall x \in X,\ x\cdot1 = 1\cdot x = x\)
        \item [\textbf{(D1)}] \(\forall x,y,z \in X,\ x\cdot(y+z)=x\cdot y+x\cdot z\)
        \item [\textbf{(D2)}] \(\forall x,y,z \in X,\ (x+y)\cdot z=x\cdot z+y\cdot z\)
    \end{enumerate}
\end{definition-pre}
\begin{definition-pre}[Commutative Ring]
    \label{def:commutative_ring}
    Let \(X\) be a non-empty set. \\
    Let \(+: X\times X \to X\) be an additive internal composition law on \(X\). \\
    Let \(\cdot: X\times X \to X\) be a multiplicative internal composition law on \(X\). \\\\
    A \textit{commutative ring} is a triple \(\cc{R}_c:= (X, +, \cdot)\) satisfying:
    \begin{enumerate}
        \item [\textbf{(A1)}] \(\forall x,y,z\in X,\ (x+y)+z= x+(y+z) = x+y+z\)
        \item [\textbf{(C1)}] \(\forall x,y\in X,\ x+y = y+x\)
        \item [\textbf{(N1)}] \(\exists 0 \in X : \forall x \in X,\ x+0 = 0+x = x\)
        \item [\textbf{(I1)}] \(\forall x \in X,\ \exists (-x)\in X: x+(-x) = (-x)+x = 0\)
        \item [\textbf{(A2)}] \(\forall x,y,z\in X,\ (x\cdot y)\cdot z= x\cdot(y\cdot z) = x\cdot y\cdot z\)
        \item [\textbf{(C2)}] \(\forall x,y\in X,\ x\cdot y = y\cdot x\)
        \item [\textbf{(N2)}] \(\exists 1 \in X : \forall x \in X,\ x\cdot1 = 1\cdot x = x\)
        \item [\textbf{(D1)}] \(\forall x,y,z \in X,\ x\cdot(y+z)=x\cdot y+x\cdot z\)
        \item [\textbf{(D2)}] \(\forall x,y,z \in X,\ (x+y)\cdot z=x\cdot z+y\cdot z\)
    \end{enumerate}
\end{definition-pre}

\begin{definition-pre}[Domain]
    \label{def:domain}
    Let \(X\) be a non-empty set. \\
    Let \(+: X\times X \to X\) be an additive internal composition law on \(X\). \\
    Let \(\cdot: X\times X \to X\) be a multiplicative internal composition law on \(X\). \\\\
    A \textit{domain} is a triple \(\cc{D}:= (X, +, \cdot)\) satisfying:
    \begin{enumerate}
        \item [\textbf{(A1)}] \(\forall x,y,z\in X,\ (x+y)+z= x+(y+z) = x+y+z\)
        \item [\textbf{(C1)}] \(\forall x,y\in X,\ x+y = y+x\)
        \item [\textbf{(N1)}] \(\exists 0 \in X : \forall x \in X,\ x+0 = 0+x = x\)
        \item [\textbf{(I1)}] \(\forall x \in X,\ \exists (-x)\in X: x+(-x) = (-x)+x = 0\)
        \item [\textbf{(A2)}] \(\forall x,y,z\in X,\ (x\cdot y)\cdot z= x\cdot(y\cdot z) = x\cdot y\cdot z\)
        \item [\textbf{(N2)}] \(\exists 1 \in X : \forall x \in X,\ x\cdot1 = 1\cdot x = x\)
        \item [\textbf{(D1)}] \(\forall x,y,z \in X,\ x\cdot(y+z)=x\cdot y+x\cdot z\)
        \item [\textbf{(D2)}] \(\forall x,y,z \in X,\ (x+y)\cdot z=x\cdot z+y\cdot z\)
        \item [\textbf{(Z2)}] \(\forall x,y \in X,\ x\cdot y = 0 \Rightarrow x=0 \lor y=0\)
    \end{enumerate}
\end{definition-pre}

\begin{definition-pre}[Commutative Domain]
    \label{def:commutative_domain}
    \label{def:domain}
    Let \(X\) be a non-empty set. \\
    Let \(+: X\times X \to X\) be an additive internal composition law on \(X\). \\
    Let \(\cdot: X\times X \to X\) be a multiplicative internal composition law on \(X\). \\\\
    A \textit{commutative} or \textit{integral domain} is a triple \(\cc{D}_c:= (X, +, \cdot)\) satisfying:
    \begin{enumerate}
        \item [\textbf{(A1)}] \(\forall x,y,z\in X,\ (x+y)+z= x+(y+z) = x+y+z\)
        \item [\textbf{(C1)}] \(\forall x,y\in X,\ x+y = y+x\)
        \item [\textbf{(N1)}] \(\exists 0 \in X : \forall x \in X,\ x+0 = 0+x = x\)
        \item [\textbf{(I1)}] \(\forall x \in X,\ \exists (-x)\in X: x+(-x) = (-x)+x = 0\)
        \item [\textbf{(A2)}] \(\forall x,y,z\in X,\ (x\cdot y)\cdot z= x\cdot(y\cdot z) = x\cdot y\cdot z\)
        \item [\textbf{(C2)}] \(\forall x,y\in X,\ x\cdot y = y\cdot x\)
        \item [\textbf{(N2)}] \(\exists 1 \in X : \forall x \in X,\ x\cdot1 = 1\cdot x = x\)
        \item [\textbf{(D1)}] \(\forall x,y,z \in X,\ x\cdot(y+z)=x\cdot y+x\cdot z\)
        \item [\textbf{(D2)}] \(\forall x,y,z \in X,\ (x+y)\cdot z=x\cdot z+y\cdot z\)
        \item [\textbf{(Z2)}] \(\forall x,y \in X,\ x\cdot y = 0 \Rightarrow x=0 \lor y=0\)
    \end{enumerate}
\end{definition-pre}

\begin{definition-pre}[Field]
    \label{def:field}
    Let \(X\) be a non-empty set. \\
    Let \(+: X\times X \to X\) be an additive internal composition law on \(X\). \\
    Let \(\cdot: X\times X \to X\) be a multiplicative internal composition law on \(X\). \\\\
    A \textit{field} is a triple \(\bb{F}:= (X, +, \cdot)\) satisfying:
    \begin{enumerate}
        \item [\textbf{(A1)}] \(\forall x,y,z\in X,\ (x+y)+z= x+(y+z) = x+y+z\)
        \item [\textbf{(C1)}] \(\forall x,y\in X,\ x+y = y+x\)
        \item [\textbf{(N1)}] \(\exists 0 \in X : \forall x \in X,\ x+0 = 0+x = x\)
        \item [\textbf{(I1)}] \(\forall x \in X,\ \exists (-x)\in X: x+(-x) = (-x)+x = 0\)
        \item [\textbf{(A2)}] \(\forall x,y,z\in X,\ (x\cdot y)\cdot z= x\cdot(y\cdot z) = x\cdot y\cdot z\)
        \item [\textbf{(N2)}] \(\exists 1 \in X : \forall x \in X,\ x\cdot1 = 1\cdot x = x\)
        \item [\textbf{(I2)}] \(\forall x \in X,\ \exists x^{-1}\in X: x\cdot x^{-1} = x^{-1}\cdot x = 1\)
        \item [\textbf{(D1)}] \(\forall x,y,z \in X,\ x\cdot(y+z)=x\cdot y+x\cdot z\)
        \item [\textbf{(D2)}] \(\forall x,y,z \in X,\ (x+y)\cdot z=x\cdot z+y\cdot z\)
    \end{enumerate}
\end{definition-pre}
\begin{definition-pre}[Commutative Field]
    \label{def:commutative_field}
    Let \(X\) be a non-empty set. \\
    Let \(+: X\times X \to X\) be an additive internal composition law on \(X\). \\
    Let \(\cdot: X\times X \to X\) be a multiplicative internal composition law on \(X\). \\\\
    A \textit{commutative field} is a triple \(\bb{F}_c:= (X, +, \cdot)\) satisfying:
    \begin{enumerate}
        \item [\textbf{(A1)}] \(\forall x,y,z\in X,\ (x+y)+z= x+(y+z) = x+y+z\)
        \item [\textbf{(C1)}] \(\forall x,y\in X,\ x+y = y+x\)
        \item [\textbf{(N1)}] \(\exists 0 \in X : \forall x \in X,\ x+0 = 0+x = x\)
        \item [\textbf{(I1)}] \(\forall x \in X,\ \exists (-x)\in X: x+(-x) = (-x)+x = 0\)
        \item [\textbf{(A2)}] \(\forall x,y,z\in X,\ (x\cdot y)\cdot z= x\cdot(y\cdot z) = x\cdot y\cdot z\)
        \item [\textbf{(C2)}] \(\forall x,y\in X,\ x\cdot y = y\cdot x\)
        \item [\textbf{(N2)}] \(\exists 1 \in X : \forall x \in X,\ x\cdot1 = 1\cdot x = x\)
        \item [\textbf{(I2)}] \(\forall x \in X,\ \exists x^{-1}\in X: x\cdot x^{-1} = x^{-1}\cdot x = 1\)
        \item [\textbf{(D1)}] \(\forall x,y,z \in X,\ x\cdot(y+z)=x\cdot y+x\cdot z\)
        \item [\textbf{(D2)}] \(\forall x,y,z \in X,\ (x+y)\cdot z=x\cdot z+y\cdot z\)
    \end{enumerate}
\end{definition-pre}

\begin{definition-pre}[Algebraic Number Field / Number Field]
    \label{def:number_field}
\end{definition-pre}

\begin{definition-pre}[Fundamental System]
    \label{def:fundamental_system}
\end{definition-pre}

\begin{definition-pre}[Unit]
    \label{def:unit}
\end{definition-pre}
\begin{definition-pre}[Primitive Root of Unity]
    \label{def:primitive_root_of_unity}
\end{definition-pre}

\begin{definition-pre}[Cyclotomic Polynomial]
    \label{def:cyclotomic_polynomial}
\end{definition-pre}
\begin{definition-pre}[Cyclotomic Extension of a Ring / Cyclotomic Ring]
    \label{def:cyclotomic_extension_field}
\end{definition-pre}
\begin{definition-pre}[Cyclotomic Extension of a Fiel / Cyclotomic Field]
    \label{def:cyclotomic_extension_field}
\end{definition-pre}

\section{Results}

\begin{theorem}
    \label{thm:zeta_sub_one_prime1}
    \lean{IsPrimitiveRoot.zeta_sub_one_prime'}
    \leanok
    Let $p \in \N$ be prime. \\\\
    If $\zeta_p$ is a primitive $p$-th root of unity, then $\zeta_p - 1$ is prime.
\end{theorem}
\begin{proof}
    \leanok
    This has already been formalised and included in \href{https://pitmonticone.github.io/FLT3/docs/FLT3/Mathlib/NumberTheory/Cyclotomic/Rat.html#IsPrimitiveRoot.zeta_sub_one_prime'}{Mathlib}.
\end{proof}

\begin{lemma}
    \label{lmm:fermatLastTheoremWith_of_fermatLastTheoremWith_coprime}
    \lean{fermatLastTheoremWith_of_fermatLastTheoremWith_coprime}
    \leanok
    Let $R$ be a commutative semiring, domain and normalised $\gcd$ monoid.\\% ASK EXPERTS
    Let $a, b, c \in R$. \\
    Let $n \in \N$. \\\\
    Then, to prove Fermat's Last Theorem for exponent $n$ in $R$,
    one can assume, without loss of generality, that $\gcd(a,b,c)=1$.
  \end{lemma}
  \begin{proof}
    \leanok
    This has already been formalised and included in \href{https://pitmonticone.github.io/FLT3/docs/FLT3/Mathlib/NumberTheory/FLT/Basic.html#fermatLastTheoremWith_of_fermatLastTheoremWith_coprime}{Mathlib}.
  \end{proof}

  \begin{lemma}
    \label{lmm:cube_of_castHom_ne_zero}
    \lean{cube_of_castHom_ne_zero}
    \leanok
    Let $\Z_9$ be the ring of integers modulo $9$. \\
    Let $\Z_3$ be the ring of integers modulo $3$. \\
    Let $n \in \Z_9$. \\
    Let $\phi : \Z_9 \to \Z_3$ be the canonical ring homomorphism. \\
    Let $\phi(n) \neq 0$. \\ \\
    Then $n^3=1 \lor n^3=8$.
  \end{lemma}
  \begin{proof}
    \leanok
    This has already been formalised and included in \href{https://pitmonticone.github.io/FLT3/docs/FLT3/Mathlib/NumberTheory/FLT/Three.html#cube_of_castHom_ne_zero}{Mathlib}.
  \end{proof}