% This file makes a printable version of the blueprint
% It should include all the \usepackage needed for the pdf version.
% The template version assume you want to use a modern TeX compiler
% such as xeLaTeX or luaLaTeX including support for unicode
% and Latin Modern Math font with standard bugfixes applied.
% It also uses expl3 in order to support macros related to the dependency graph.
% It also includes standard AMS packages (and their improved version
% mathtools) as well as support for links with a sober decoration
% (no ugly rectangles around links).
% It is otherwise a very minimal preamble (you should probably at least
% add cleveref and tikz-cd).

% Set document class
\documentclass[11pt,a4paper]{report}
% Import packages
%%%%%%%%%%%%%%%%%%%%%%%%%%%%%%%%%%%%%%
%%%%%%%%%%%%% PACKAGES %%%%%%%%%%%%%%%
%%%%%%%%%%%%%%%%%%%%%%%%%%%%%%%%%%%%%%

%%%%%%%%%%%% AUTHORSHIP %%%%%%%%%%%%%%
\usepackage{authblk}   % Helps format author and affiliation information

%%%%%%%%%%%%%%% TEXT %%%%%%%%%%%%%%%%%
\usepackage[T1]{fontenc}     % Output font encoding for international characters.
\usepackage[utf8]{inputenc}  % Allows the use of UTF-8 characters in the LaTeX source code.
%\usepackage[italian]{babel}  % Loads the babel package with Italian localization settings
\usepackage{lmodern}         % Modern font family for improved readability.
\usepackage{comment}         % Enable comment environment
\usepackage{microtype}       % Improve typography by enhancing character protrusion, adjusts kerning, and allowing for font expansion
\usepackage{lipsum}          % Generate "filler" text in documents for testing or demo
\usepackage{fancyvrb}        % Enables advanced verbatim text features
\usepackage{fontspec}
% \setmathfont{latinmodern-math.otf}
% \setmathfont[range=\varnothing]{Asana-Math.otf}
% \setmathfont[range=\pitchfork]{Asana-Math.otf}
% \setmathfont[range=\intprod]{Asana-Math.otf}
% \setmathfont[range=\int]{latinmodern-math.otf}
\usepackage{polyglossia}
\setdefaultlanguage{english}
\usepackage{relsize}

%%%%%%%%%%%%%%% CODE %%%%%%%%%%%%%%%%%
\usepackage{listings}
%\usepackage{minted}

%%%%%%%%%%%%%% LAYOUT %%%%%%%%%%%%%%%%
\usepackage{parskip}            % Sets space between paragraphs and setting \parindent to 0pt
\usepackage[a4paper]{geometry}  % Enables customization of page layout and margins
\usepackage{fancyhdr}           % Allows customization of page headers and footers
\usepackage{setspace}           % Controls line spacing
\usepackage{titlesec}           % Package for customizing section and chapter headings
\usepackage{booktabs}           % Enables better spacing and horizontal rules
\usepackage{array}              % Enhance column formatting
\usepackage{chngcntr}           % Control the resetting of counters

%%%%%%%%%%%% MATHEMATICS %%%%%%%%%%%%%
\usepackage{amsmath}   % Provides enhanced math environments and features
\usepackage{amsfonts}  % Gives access to a variety of mathematical fonts
\usepackage{amssymb}   % Provides additional mathematical symbols
\usepackage{amsthm}    % Allows for the definition of theorems, lemmas, and proofs
\usepackage{mathdots}  % Provides various dot symbols for mathematics.
\usepackage{mathtools} % Extends the functionality of amsmath package for more math tools.
\usepackage{expl3}
\usepackage{etexcmds}
\usepackage{thmtools}
\usepackage{xfrac}

%%%%%%%%%%%%%%% LISTS %%%%%%%%%%%%%%%%
\usepackage{enumerate} % Allows customization of enumeration and itemize lists
\usepackage{enumitem}  % Offers control over the formatting of itemized lists.

%%%%%%%%%%% VISUALISATION %%%%%%%%%%%%
\usepackage[dvipsnames]{xcolor} % Allows the definition and use of colors
\usepackage{tikz}               % A powerful tool for creating graphics and diagrams
\usepackage[most]{tcolorbox}    % Enables colored text boxes
\usepackage{graphicx}           % Supports the inclusion of external images
\usepackage{caption}            % Customizes captions for figures and tables
\usepackage{float}              % Improved interface for floating objects like figures and tables
\definecolor{keywordcolor}{rgb}{0.7, 0.1, 0.1}   % red
\definecolor{tacticcolor}{rgb}{0.0, 0.1, 0.6}    % blue
\definecolor{commentcolor}{rgb}{0.4, 0.4, 0.4}   % grey
\definecolor{symbolcolor}{rgb}{0.0, 0.1, 0.6}    % blue
\definecolor{sortcolor}{rgb}{0.1, 0.5, 0.1}      % green
\definecolor{attributecolor}{rgb}{0.7, 0.1, 0.1} % red
\usepackage{tikz-cd}

%%%%%%%%%%%%%% TABLES %%%%%%%%%%%%%%%%
\usepackage{tabularx}  % Extends the tabular environment for better table control
\usepackage{multirow}  % Allows cells in tables to span multiple rows

%%%%%%%%%%%% REFERENCES %%%%%%%%%%%%%%
\usepackage{hyperref} % Enables internal & external hyperlinks
\usepackage[nameinlink, capitalize]{cleveref}

%%%%%%%%%%% BIBLIOGRAPHY %%%%%%%%%%%%%
\usepackage[backend=biber]{biblatex}
% Import macros
\input{preamble/common}
% This file makes a printable version of the blueprint
% It should include all the \usepackage needed for the pdf version.
% The template version assume you want to use a modern TeX compiler
% such as xeLaTeX or luaLaTeX including support for unicode
% and Latin Modern Math font with standard bugfixes applied.
% It also uses expl3 in order to support macros related to the dependency graph.
% It also includes standard AMS packages (and their improved version
% mathtools) as well as support for links with a sober decoration
% (no ugly rectangles around links).
% It is otherwise a very minimal preamble (you should probably at least
% add cleveref and tikz-cd).

% Set document class
\documentclass[11pt,a4paper]{report}
% Import packages
%%%%%%%%%%%%%%%%%%%%%%%%%%%%%%%%%%%%%%
%%%%%%%%%%%%% PACKAGES %%%%%%%%%%%%%%%
%%%%%%%%%%%%%%%%%%%%%%%%%%%%%%%%%%%%%%

%%%%%%%%%%%% AUTHORSHIP %%%%%%%%%%%%%%
\usepackage{authblk}   % Helps format author and affiliation information

%%%%%%%%%%%%%%% TEXT %%%%%%%%%%%%%%%%%
\usepackage[T1]{fontenc}     % Output font encoding for international characters.
\usepackage[utf8]{inputenc}  % Allows the use of UTF-8 characters in the LaTeX source code.
%\usepackage[italian]{babel}  % Loads the babel package with Italian localization settings
\usepackage{lmodern}         % Modern font family for improved readability.
\usepackage{comment}         % Enable comment environment
\usepackage{microtype}       % Improve typography by enhancing character protrusion, adjusts kerning, and allowing for font expansion
\usepackage{lipsum}          % Generate "filler" text in documents for testing or demo
\usepackage{fancyvrb}        % Enables advanced verbatim text features
\usepackage{fontspec}
% \setmathfont{latinmodern-math.otf}
% \setmathfont[range=\varnothing]{Asana-Math.otf}
% \setmathfont[range=\pitchfork]{Asana-Math.otf}
% \setmathfont[range=\intprod]{Asana-Math.otf}
% \setmathfont[range=\int]{latinmodern-math.otf}
\usepackage{polyglossia}
\setdefaultlanguage{english}
\usepackage{relsize}

%%%%%%%%%%%%%%% CODE %%%%%%%%%%%%%%%%%
\usepackage{listings}
%\usepackage{minted}

%%%%%%%%%%%%%% LAYOUT %%%%%%%%%%%%%%%%
\usepackage{parskip}            % Sets space between paragraphs and setting \parindent to 0pt
\usepackage[a4paper]{geometry}  % Enables customization of page layout and margins
\usepackage{fancyhdr}           % Allows customization of page headers and footers
\usepackage{setspace}           % Controls line spacing
\usepackage{titlesec}           % Package for customizing section and chapter headings
\usepackage{booktabs}           % Enables better spacing and horizontal rules
\usepackage{array}              % Enhance column formatting
\usepackage{chngcntr}           % Control the resetting of counters

%%%%%%%%%%%% MATHEMATICS %%%%%%%%%%%%%
\usepackage{amsmath}   % Provides enhanced math environments and features
\usepackage{amsfonts}  % Gives access to a variety of mathematical fonts
\usepackage{amssymb}   % Provides additional mathematical symbols
\usepackage{amsthm}    % Allows for the definition of theorems, lemmas, and proofs
\usepackage{mathdots}  % Provides various dot symbols for mathematics.
\usepackage{mathtools} % Extends the functionality of amsmath package for more math tools.
\usepackage{expl3}
\usepackage{etexcmds}
\usepackage{thmtools}
\usepackage{xfrac}

%%%%%%%%%%%%%%% LISTS %%%%%%%%%%%%%%%%
\usepackage{enumerate} % Allows customization of enumeration and itemize lists
\usepackage{enumitem}  % Offers control over the formatting of itemized lists.

%%%%%%%%%%% VISUALISATION %%%%%%%%%%%%
\usepackage[dvipsnames]{xcolor} % Allows the definition and use of colors
\usepackage{tikz}               % A powerful tool for creating graphics and diagrams
\usepackage[most]{tcolorbox}    % Enables colored text boxes
\usepackage{graphicx}           % Supports the inclusion of external images
\usepackage{caption}            % Customizes captions for figures and tables
\usepackage{float}              % Improved interface for floating objects like figures and tables
\definecolor{keywordcolor}{rgb}{0.7, 0.1, 0.1}   % red
\definecolor{tacticcolor}{rgb}{0.0, 0.1, 0.6}    % blue
\definecolor{commentcolor}{rgb}{0.4, 0.4, 0.4}   % grey
\definecolor{symbolcolor}{rgb}{0.0, 0.1, 0.6}    % blue
\definecolor{sortcolor}{rgb}{0.1, 0.5, 0.1}      % green
\definecolor{attributecolor}{rgb}{0.7, 0.1, 0.1} % red
\usepackage{tikz-cd}

%%%%%%%%%%%%%% TABLES %%%%%%%%%%%%%%%%
\usepackage{tabularx}  % Extends the tabular environment for better table control
\usepackage{multirow}  % Allows cells in tables to span multiple rows

%%%%%%%%%%%% REFERENCES %%%%%%%%%%%%%%
\usepackage{hyperref} % Enables internal & external hyperlinks
\usepackage[nameinlink, capitalize]{cleveref}

%%%%%%%%%%% BIBLIOGRAPHY %%%%%%%%%%%%%
\usepackage[backend=biber]{biblatex}
% Import macros
\input{preamble/common}
% This file makes a printable version of the blueprint
% It should include all the \usepackage needed for the pdf version.
% The template version assume you want to use a modern TeX compiler
% such as xeLaTeX or luaLaTeX including support for unicode
% and Latin Modern Math font with standard bugfixes applied.
% It also uses expl3 in order to support macros related to the dependency graph.
% It also includes standard AMS packages (and their improved version
% mathtools) as well as support for links with a sober decoration
% (no ugly rectangles around links).
% It is otherwise a very minimal preamble (you should probably at least
% add cleveref and tikz-cd).

% Set document class
\documentclass[11pt,a4paper]{report}
% Import packages
%%%%%%%%%%%%%%%%%%%%%%%%%%%%%%%%%%%%%%
%%%%%%%%%%%%% PACKAGES %%%%%%%%%%%%%%%
%%%%%%%%%%%%%%%%%%%%%%%%%%%%%%%%%%%%%%

%%%%%%%%%%%% AUTHORSHIP %%%%%%%%%%%%%%
\usepackage{authblk}   % Helps format author and affiliation information

%%%%%%%%%%%%%%% TEXT %%%%%%%%%%%%%%%%%
\usepackage[T1]{fontenc}     % Output font encoding for international characters.
\usepackage[utf8]{inputenc}  % Allows the use of UTF-8 characters in the LaTeX source code.
%\usepackage[italian]{babel}  % Loads the babel package with Italian localization settings
\usepackage{lmodern}         % Modern font family for improved readability.
\usepackage{comment}         % Enable comment environment
\usepackage{microtype}       % Improve typography by enhancing character protrusion, adjusts kerning, and allowing for font expansion
\usepackage{lipsum}          % Generate "filler" text in documents for testing or demo
\usepackage{fancyvrb}        % Enables advanced verbatim text features
\usepackage{fontspec}
% \setmathfont{latinmodern-math.otf}
% \setmathfont[range=\varnothing]{Asana-Math.otf}
% \setmathfont[range=\pitchfork]{Asana-Math.otf}
% \setmathfont[range=\intprod]{Asana-Math.otf}
% \setmathfont[range=\int]{latinmodern-math.otf}
\usepackage{polyglossia}
\setdefaultlanguage{english}
\usepackage{relsize}

%%%%%%%%%%%%%%% CODE %%%%%%%%%%%%%%%%%
\usepackage{listings}
%\usepackage{minted}

%%%%%%%%%%%%%% LAYOUT %%%%%%%%%%%%%%%%
\usepackage{parskip}            % Sets space between paragraphs and setting \parindent to 0pt
\usepackage[a4paper]{geometry}  % Enables customization of page layout and margins
\usepackage{fancyhdr}           % Allows customization of page headers and footers
\usepackage{setspace}           % Controls line spacing
\usepackage{titlesec}           % Package for customizing section and chapter headings
\usepackage{booktabs}           % Enables better spacing and horizontal rules
\usepackage{array}              % Enhance column formatting
\usepackage{chngcntr}           % Control the resetting of counters

%%%%%%%%%%%% MATHEMATICS %%%%%%%%%%%%%
\usepackage{amsmath}   % Provides enhanced math environments and features
\usepackage{amsfonts}  % Gives access to a variety of mathematical fonts
\usepackage{amssymb}   % Provides additional mathematical symbols
\usepackage{amsthm}    % Allows for the definition of theorems, lemmas, and proofs
\usepackage{mathdots}  % Provides various dot symbols for mathematics.
\usepackage{mathtools} % Extends the functionality of amsmath package for more math tools.
\usepackage{expl3}
\usepackage{etexcmds}
\usepackage{thmtools}
\usepackage{xfrac}

%%%%%%%%%%%%%%% LISTS %%%%%%%%%%%%%%%%
\usepackage{enumerate} % Allows customization of enumeration and itemize lists
\usepackage{enumitem}  % Offers control over the formatting of itemized lists.

%%%%%%%%%%% VISUALISATION %%%%%%%%%%%%
\usepackage[dvipsnames]{xcolor} % Allows the definition and use of colors
\usepackage{tikz}               % A powerful tool for creating graphics and diagrams
\usepackage[most]{tcolorbox}    % Enables colored text boxes
\usepackage{graphicx}           % Supports the inclusion of external images
\usepackage{caption}            % Customizes captions for figures and tables
\usepackage{float}              % Improved interface for floating objects like figures and tables
\definecolor{keywordcolor}{rgb}{0.7, 0.1, 0.1}   % red
\definecolor{tacticcolor}{rgb}{0.0, 0.1, 0.6}    % blue
\definecolor{commentcolor}{rgb}{0.4, 0.4, 0.4}   % grey
\definecolor{symbolcolor}{rgb}{0.0, 0.1, 0.6}    % blue
\definecolor{sortcolor}{rgb}{0.1, 0.5, 0.1}      % green
\definecolor{attributecolor}{rgb}{0.7, 0.1, 0.1} % red
\usepackage{tikz-cd}

%%%%%%%%%%%%%% TABLES %%%%%%%%%%%%%%%%
\usepackage{tabularx}  % Extends the tabular environment for better table control
\usepackage{multirow}  % Allows cells in tables to span multiple rows

%%%%%%%%%%%% REFERENCES %%%%%%%%%%%%%%
\usepackage{hyperref} % Enables internal & external hyperlinks
\usepackage[nameinlink, capitalize]{cleveref}

%%%%%%%%%%% BIBLIOGRAPHY %%%%%%%%%%%%%
\usepackage[backend=biber]{biblatex}
% Import macros
\input{preamble/common}
% This file makes a printable version of the blueprint
% It should include all the \usepackage needed for the pdf version.
% The template version assume you want to use a modern TeX compiler
% such as xeLaTeX or luaLaTeX including support for unicode
% and Latin Modern Math font with standard bugfixes applied.
% It also uses expl3 in order to support macros related to the dependency graph.
% It also includes standard AMS packages (and their improved version
% mathtools) as well as support for links with a sober decoration
% (no ugly rectangles around links).
% It is otherwise a very minimal preamble (you should probably at least
% add cleveref and tikz-cd).

% Set document class
\documentclass[11pt,a4paper]{report}
% Import packages
\input{packages}
% Import macros
\input{preamble/common}
\input{preamble/print}
% Import author
\input{author}

% Set hyperref parameters
\hypersetup{
    colorlinks=true, % TRUE: colors the text of the links; FALSE: puts a colored box around them
    linkcolor=NavyBlue, % Sets color of internal links (\ref or \label)
    urlcolor=NavyBlue, % Sets color of external links
    citecolor=NavyBlue, % Sets color of citation links (\cite).
    anchorcolor=NavyBlue % Sets color of anchor text
}

% Set source for bibliography
\addbibresource{references.bib}

% Set title and subtitle
\title{Formalising Fermat's Last Theorem for Exponent 3 in Lean}
% Set date
\date{\today}

\begin{document}

% Add the title
\maketitle

% Add the abstract
\begin{abstract}
    This paper serves as the blueprint for the project aimed at formalising Fermat’s Last Theorem
    for the exponent 3 (\href{https://github.com/pitmonticone/FLT3}{FLT3}) using the Lean 4 proof assistant.
    It offers comprehensive coverage of all necessary definitions, theorems, and proofs, presented in natural,
    informal language to facilitate understanding and implementation. The related code is being integrated
    into the Lean library of formalized mathematics (\href{https://github.com/leanprover-community/mathlib4}{Mathlib})
    and imported as a dependency in the broader formalisation project of Fermat's Last Theorem led by
    Kevin Buzzard and Richard Taylor (\href{https://github.com/ImperialCollegeLondon/FLT}{FLT}).
\end{abstract}

% Add the table of contents
\tableofcontents

% Add main contents
\input{main}

% Bibliography
\nocite{*}
\printbibliography[title=References]
\addcontentsline{toc}{chapter}{References}
\end{document}
% Import author
\input{author}

% Set hyperref parameters
\hypersetup{
    colorlinks=true, % TRUE: colors the text of the links; FALSE: puts a colored box around them
    linkcolor=NavyBlue, % Sets color of internal links (\ref or \label)
    urlcolor=NavyBlue, % Sets color of external links
    citecolor=NavyBlue, % Sets color of citation links (\cite).
    anchorcolor=NavyBlue % Sets color of anchor text
}

% Set source for bibliography
\addbibresource{references.bib}

% Set title and subtitle
\title{Formalising Fermat's Last Theorem for Exponent 3 in Lean}
% Set date
\date{\today}

\begin{document}

% Add the title
\maketitle

% Add the abstract
\begin{abstract}
    This paper serves as the blueprint for the project aimed at formalising Fermat’s Last Theorem
    for the exponent 3 (\href{https://github.com/pitmonticone/FLT3}{FLT3}) using the Lean 4 proof assistant.
    It offers comprehensive coverage of all necessary definitions, theorems, and proofs, presented in natural,
    informal language to facilitate understanding and implementation. The related code is being integrated
    into the Lean library of formalized mathematics (\href{https://github.com/leanprover-community/mathlib4}{Mathlib})
    and imported as a dependency in the broader formalisation project of Fermat's Last Theorem led by
    Kevin Buzzard and Richard Taylor (\href{https://github.com/ImperialCollegeLondon/FLT}{FLT}).
\end{abstract}

% Add the table of contents
\tableofcontents

% Add main contents
% In this file you should put the actual content of the blueprint.
% It will be used both by the web and the print version.
% It should *not* include the \begin{document}
%
% If you want to split the blueprint content into several files then
% the current file can be a simple sequence of \input. Otherwise It
% can start with a \section or \chapter for instance.

% This is the main point of entry to the blueprint.
% Add chapters of the blueprint here.
% This file is not meant to be built. Build src/web.tex or src/print.tex instead.

% Introduction
\input{chapters/0-introduction}
% Preliminaries
\input{chapters/1-preliminaries}
% The Third Cyclotomic Field
\input{chapters/2-third_cyclotomic_extensions}
% Fermat's Last Theorem for Exponent 3
\input{chapters/3-fermat_last_theorem_3}
% Acknowledgements
%% \input{chapters/3-acknowledgements}

% Bibliography
\nocite{*}
\printbibliography[title=References]
\addcontentsline{toc}{chapter}{References}
\end{document}
% Import author
\input{author}

% Set hyperref parameters
\hypersetup{
    colorlinks=true, % TRUE: colors the text of the links; FALSE: puts a colored box around them
    linkcolor=NavyBlue, % Sets color of internal links (\ref or \label)
    urlcolor=NavyBlue, % Sets color of external links
    citecolor=NavyBlue, % Sets color of citation links (\cite).
    anchorcolor=NavyBlue % Sets color of anchor text
}

% Set source for bibliography
\addbibresource{references.bib}

% Set title and subtitle
\title{Formalising Fermat's Last Theorem for Exponent 3 in Lean}
% Set date
\date{\today}

\begin{document}

% Add the title
\maketitle

% Add the abstract
\begin{abstract}
    This paper serves as the blueprint for the project aimed at formalising Fermat’s Last Theorem
    for the exponent 3 (\href{https://github.com/pitmonticone/FLT3}{FLT3}) using the Lean 4 proof assistant.
    It offers comprehensive coverage of all necessary definitions, theorems, and proofs, presented in natural,
    informal language to facilitate understanding and implementation. The related code is being integrated
    into the Lean library of formalized mathematics (\href{https://github.com/leanprover-community/mathlib4}{Mathlib})
    and imported as a dependency in the broader formalisation project of Fermat's Last Theorem led by
    Kevin Buzzard and Richard Taylor (\href{https://github.com/ImperialCollegeLondon/FLT}{FLT}).
\end{abstract}

% Add the table of contents
\tableofcontents

% Add main contents
% In this file you should put the actual content of the blueprint.
% It will be used both by the web and the print version.
% It should *not* include the \begin{document}
%
% If you want to split the blueprint content into several files then
% the current file can be a simple sequence of \input. Otherwise It
% can start with a \section or \chapter for instance.

% This is the main point of entry to the blueprint.
% Add chapters of the blueprint here.
% This file is not meant to be built. Build src/web.tex or src/print.tex instead.

% Introduction
\chapter*{Introduction}
% \label{chap:introduction}
\addcontentsline{toc}{chapter}{Introduction}

% https://formal-mathematics.github.io/intro.html

% https://github.com/pitmonticone/FLT3

% https://pitmonticone.github.io/FLT3/docs/

% https://pitmonticone.github.io/FLT3/blueprint

% https://pitmonticone.github.io/FLT3/blueprint/dep_graph_document.html

% https://pitmonticone.github.io/FLT3/blueprint.pdf


% What is Formalisation / a proof assistant / interactive theorem prover? Why is it important?

% Proof assistants, also known as interactive theorem provers, are tools for producing formally correct mathematics.
% Mathematicians write definitions, theorems, and proofs in a specialized language, with the computer giving instant feedback about each line.
% Examples of popular proof assistants include Agda, Coq, HOL Light, Isabelle, Lean, and Mizar.
% These tools have been used to develop large libraries of "known" mathematics as well as to verify major recent or controversial results.
% The act of formalizing a proof is not mechanical: it involves substantial mathematical insight in its own right,
% and it constitutes the essence of formalized mathematics, which is now recognized as a mathematical discipline on its own.
% Despite its young age, this discipline is blossoming at a very fast pace; it nowadays occupies a significant place in the mathematical horizon.

% Related to proof assistants, automated reasoning tools take in the formal statement of a theorem and attempt to prove it without further human input. While the study of these tools themselves is the domain of computer science, their use on mathematical theorems requires sophisticated reductions and reformulations.


% What is Lean? Why is it important?
%% interactive theorem prover and general-purpose programming language
% What is Mathlib? Why is it important?

% Context, Motivation, Basic Definitions?

% MOTIVATION
% It's going to be used in the general FLT project coordinated by Kevin Buzzard https://github.com/ImperialCollegeLondon/FLT

% NOTATION
% number sets

% MAIN DEFINITIONS AND STRUCTURES?

% GOAL / AIM

% CONTRIBUTIONS
% - About 20 formal / formalised proofs
% - Code:
% - All informal / informalised proofs
% - all blueprint, PRs to Mathlib (https://github.com/leanprover-community/mathlib4/pull/11677, https://github.com/leanprover-community/mathlib4/pull/11695), Porting to Mathlib (https://github.com/leanprover-community/mathlib4/pull/11767), see LaTeX comments, talk, ...
% Code refactoring


% PROOF STRATEGY

% Motivation: It's going to be used in the general FLT project
% – Disclaimer / Acknowledgement with my contributions
% – Informalisation should be clear
% – Introduction: definitions + theorem statements + proof strategy
% Porting to Mathlib
% Fermat's Last Theorem (FLT) is a famous problem in number theory. It states that there are no positive integers $a$, $b$, and $c$ such that $a^n + b^n = c^n$ for any integer value of $n$ greater than $2$. The theorem was first conjectured by Pierre de Fermat in 1637 and remained unproven for over 350 years. The first successful proof was given by Andrew Wiles in 1994. The proof is long and complex, and it relies on many different areas of mathematics, including algebraic geometry and modular forms.


% \section*{Motivation}
% \section*{Contributions}
% \section*{Overview / Structure}

% https://github.com/pitmonticone/FLT3/graphs/contributors

% SEE FLT PROJECT
% Preliminaries
\chapter{Preliminaries}
%\label{chap:preliminaries}
%\addcontentsline{toc}{chapter}{Preliminaries}

\section{Notation}

% \begin{center}
\begin{tabular}{>{\centering}m{1.8cm} m{5.8cm}}
\toprule
\textbf{Symbol} & \multicolumn{1}{c}{\textbf{Description}} \\
\midrule
$\lnot$ & Logical negation \\
$\top$  & Logical truth / Tautology \\
$\bot$  & Logical falsehood / Contradiction \\
$\land$ & Logical conjunction \\
$\lor$ & Logical inclusive disjunction \\
$:=$   & Definition \\
$\forall$ & Universal quantification \\
$\exists$ & Existential quantification \\
$\exists!$ & Unique existential quantification \\
$\N$ & Set of natural numbers \\
$\Z$ & Set of integer numbers \\
$\Z_n$ & Set of integers modulo $n$ \\
$\Q$ & Set of rational numbers \\
$[n]$ & Equivalence class of $n$ \\
$\divides$ & Divisibility relation \\
$\notdivides$ & Non-divisibility relation \\
$\gcd$ & Greatest common divisor \\
$\zeta_n$ & Primitive $n$-th root of unity \\
\bottomrule
\end{tabular}
% \end{center}

\newpage

\section{Definitions}

\begin{definition-intro}[Monoid]
   % \label{def:monoid}
    Let \(X\) be a non-empty set.\\
    Let \(\circ: X\times X \to X\) be an internal composition law on \(X\). \\\\
    A \textit{monoid} is a pair \(\cc{M}:= (X, \circ)\) satisfying:
    \begin{enumerate}
        \item [\textbf{(A)}] \(\forall x,y,z\in X,\ (x\circ y)\circ z= x\circ (y\circ z) = x\circ y \circ z\)
        \item [\textbf{(N)}] \(\exists e \in X : \forall x \in X,\ x\circ e = e \circ x = x\)
    \end{enumerate}
\end{definition-intro}

\begin{definition-intro}[Commutative Monoid]
    %\label{def:commutative_monoid}
    Let \(X\) be a non-empty set.\\
    Let \(\circ: X\times X \to X\) be an internal composition law on \(X\). \\\\
    A \textit{commutative monoid} is a pair \(\cc{M}_a:= (X, \circ)\) satisfying:
    \begin{enumerate}
        \item [\textbf{(A)}] \(\forall x,y,z\in X,\ (x\circ y)\circ z= x\circ (y\circ z) = x\circ y \circ z\)
        \item [\textbf{(N)}] \(\exists e \in X : \forall x \in X,\ x\circ e = e \circ x = x\)
        \item [\textbf{(C)}] \(\forall x,y\in X,\ x\circ y = y\circ x\)
    \end{enumerate}
\end{definition-intro}

\begin{definition-intro}[Group]
    %\label{def:group}
    Let \(X\) be a non-empty set.\\
    Let \(\circ: X\times X \to X\) be an internal composition law on \(X\). \\\\
    A \textit{group} is a pair \(\cc{G}:= (X, \circ)\) satisfying:
    \begin{enumerate}
        \item [\textbf{(A)}] \(\forall x,y,z\in X,\ (x\circ y)\circ z= x\circ (y\circ z) = x\circ y \circ z\)
        \item [\textbf{(N)}] \(\exists e \in X : \forall x \in X,\ x\circ e = e \circ x = x\)
        \item [\textbf{(I)}] \(\forall x \in X,\ \exists x'\in X: x\circ x' = x'\circ x = e\)
    \end{enumerate}
\end{definition-intro}

\begin{definition-intro}[Commutative Group]
    %\label{def:commutative_group}
    Let \(X\) be a non-empty set.\\
    Let \(\circ: X\times X \to X\) be an internal composition law on \(X\). \\\\
    A \textit{commutative group} is a pair \(\cc{G}_a:= (X, \circ)\) satisfying:
    \begin{enumerate}
        \item [\textbf{(A)}] \(\forall x,y,z\in X,\ (x\circ y)\circ z= x\circ (y\circ z) = x\circ y \circ z\)
        \item [\textbf{(N)}] \(\exists e \in X : \forall x \in X,\ x\circ e = e \circ x = x\)
        \item [\textbf{(I)}] \(\forall x \in X,\ \exists x'\in X: x\circ x' = x'\circ x = e\)
        \item [\textbf{(C)}] \(\forall x,y\in X,\ x\circ y = y\circ x\)
    \end{enumerate}
\end{definition-intro}

\begin{definition-intro}[Semiring]
    %\label{def:semiring}
\end{definition-intro}

\begin{definition-intro}[Commutative Semiring]
    %\label{def:commutative_semiring}
\end{definition-intro}

\begin{definition-intro}[Ring]
    %\label{def:ring}
\end{definition-intro}

\begin{definition-intro}[Field]
    %\label{def:field}
\end{definition-intro}
\begin{definition-intro}[Algebraic Number Field / Number Field]
    %\label{def:number_field}
\end{definition-intro}

\begin{definition-intro}[Fundamental System]
    %\label{def:fundamental_system}
\end{definition-intro}

\begin{definition-intro}[Unit]
    %\label{def:primitive_element}
\end{definition-intro}
\begin{definition-intro}[Primitive Root of Unity]
    %\label{def:primitive_root_of_unity}
\end{definition-intro}

\begin{definition-intro}[Cyclotomic Polynomial]
    %\label{def:cyclotomic_polynomial}
\end{definition-intro}
\begin{definition-intro}[Cyclotomic Extension of a Ring / Cyclotomic Ring]
    %\label{def:cyclotomic_extension_field}
\end{definition-intro}
\begin{definition-intro}[Cyclotomic Extension of a Fiel / Cyclotomic Field]
    %\label{def:cyclotomic_extension_field}
\end{definition-intro}

\section{Results}

\begin{theorem}
    \label{thm:zeta_sub_one_prime1}
    \lean{IsPrimitiveRoot.zeta_sub_one_prime'}
    \leanok
    Let $p \in \N$ be prime. \\\\
    If $\zeta_p$ is a primitive $p$-th root of unity, then $\zeta_p - 1$ is prime.
\end{theorem}
\begin{proof}
    \leanok
    This has already been formalised and included in \href{https://pitmonticone.github.io/FLT3/docs/FLT3/Mathlib/NumberTheory/Cyclotomic/Rat.html#IsPrimitiveRoot.zeta_sub_one_prime'}{Mathlib}.
\end{proof}

\begin{lemma}
    \label{lmm:fermatLastTheoremWith_of_fermatLastTheoremWith_coprime}
    \lean{fermatLastTheoremWith_of_fermatLastTheoremWith_coprime}
    \leanok
    Let $R$ be a commutative semiring, domain and normalised $\gcd$ monoid.\\% ASK EXPERTS
    Let $a, b, c \in R$. \\
    Let $n \in \N$. \\\\
    Then, to prove Fermat's Last Theorem for exponent $n$ in $R$,
    one can assume, without loss of generality, that $\gcd(a,b,c)=1$.
  \end{lemma}
  \begin{proof}
    \leanok
    This has already been formalised and included in \href{https://pitmonticone.github.io/FLT3/docs/FLT3/Mathlib/NumberTheory/FLT/Basic.html#fermatLastTheoremWith_of_fermatLastTheoremWith_coprime}{Mathlib}.
  \end{proof}

  \begin{lemma}
    \label{lmm:cube_of_castHom_ne_zero}
    \lean{cube_of_castHom_ne_zero}
    \leanok
    Let $\Z_9$ be the ring of integers modulo $9$. \\
    Let $\Z_3$ be the ring of integers modulo $3$. \\
    Let $n \in \Z_9$. \\
    Let $\phi : \Z_9 \to \Z_3$ be the canonical ring homomorphism. \\
    Let $\phi(n) \neq 0$. \\ \\
    Then $n^3=1 \lor n^3=8$.
  \end{lemma}
  \begin{proof}
    \leanok
    This has already been formalised and included in \href{https://pitmonticone.github.io/FLT3/docs/FLT3/Mathlib/NumberTheory/FLT/Three.html#cube_of_castHom_ne_zero}{Mathlib}.
  \end{proof}
% The Third Cyclotomic Field
\chapter{Third Cyclotomic Extensions}
%\chapter{Cyclotomic Extensions}
%\chapter{Cyclotomic Fields and Rings}
%\chapter{Third Cyclotomic Fields and Rings}
%\chapter{Cyclotomic Extensions of Fields and Rings}
%\label{chap:cyclo}
%\addcontentsline{toc}{chapter}{The Third Cyclotomic Field}

% INSERT BASIC DEFINITIONS
% - Number field
% - Cyclotomic extension of a field
% - Cyclotomic extension of a ring
% - Group of units
% - Fundamental system

\begin{theorem}
    \label{thm:mem}
    \lean{IsCyclotomicExtension.Rat.Three.Units.mem}
    \leanok
    Let $K = \Q(\zeta_3)$ be the third cyclotomic field. \\
    Let $\cc{O}_K = \Z[\zeta_3]$ be the ring of integers of $K$. \\
    Let $\cc{O}^\times_K$ be the group of units of $\cc{O}_K$. \\
    Let $\zeta_3 \in K$ be any primitive third root of unity. \\
    Let $\eta \in \cc{O}_K$ be the element corresponding to $\zeta_3 \in K$. \\
    Let $\lambda \in \cc{O}_K$ be such that $\lambda = \eta -1$. \\
    Let $u \in \cc{O}^\times_K$ be a unit. \\\\
    Then $u \in \set{1, -1, \eta, -\eta, \eta^2, -\eta^2}$.
\end{theorem}
\begin{proof}
    \leanok
    Let $\cc{F}$ be the fundamental system of $K$. \\
    By properties of cyclotomic fields, we know that $\rank{K} = 0$
    (see \href{https://pitmonticone.github.io/FLT3/docs/Mathlib/NumberTheory/NumberField/Embeddings.html#NumberField.InfinitePlace.card_eq_nrRealPlaces_add_nrComplexPlaces}{this lemma},
    \href{https://pitmonticone.github.io/FLT3/docs/FLT3/Mathlib/NumberTheory/Cyclotomic/Embeddings.html#IsCyclotomicExtension.Rat.nrRealPlaces_eq_zero}{this lemma}
    and \href{https://pitmonticone.github.io/FLT3/docs/FLT3/Mathlib/NumberTheory/Cyclotomic/Embeddings.html#IsCyclotomicExtension.Rat.nrComplexPlaces_eq_totient_div_two}{this lemma}
    which have already been formalised and included in Mathlib).
    By the Dirichlet Unit Theorem (see \href{https://pitmonticone.github.io/FLT3/docs/Mathlib/NumberTheory/NumberField/Units.html#NumberField.Units.exist_unique_eq_mul_prod}{Mathlib}),
    we know that
    $$\exists x \in K \text{ with finite order, such that } u = x \prod_{v\in\cc{F}} v,$$
    but since $\rank{K} = 0$, then $\cc{F} = \emptyset$, which implies that $u = x$.\\
    Since $u = x$ has finite order, by properties of primitive roots
    (see \href{https://pitmonticone.github.io/FLT3/docs/Mathlib/NumberTheory/Cyclotomic/PrimitiveRoots.html#IsPrimitiveRoot.exists_pow_or_neg_mul_pow_of_isOfFinOrder}{this lemma}
    that has already been formalised and included in Mathlib), we can deduce that
    $$\exists r < 3 \text{ such that } u = \eta^r \lor u = -\eta^r.$$
    Therefore, we can conclude
    $$u \in \setb{\pm \eta^r}{r \in \set{0,1,2}} = \set{1, -1, \eta, -\eta, \eta^2, -\eta^2}.$$
\end{proof}

\begin{theorem}
    \label{thm:not_exists_int_three_dvd_sub}
    \lean{IsCyclotomicExtension.Rat.Three.Units.not_exists_int_three_dvd_sub}
    \leanok
    Let $K = \Q(\zeta_3)$ be the third cyclotomic field. \\
    Let $\cc{O}_K = \Z[\zeta_3]$ be the ring of integers of $K$. \\
    Let $\cc{O}^\times_K$ be the group of units of $\cc{O}_K$. \\
    Let $\zeta_3 \in K$ be any primitive third root of unity. \\
    Let $\eta \in \cc{O}_K$ be the element corresponding to $\zeta_3 \in K$. \\
    Let $m \in \Z$. \\\\
    Then $3 \notdivides \eta - m$.
\end{theorem}
\begin{proof}
    \leanok
    By properties of cyclotomic fields, we know that $\set{1,\eta}$ is an integral power basis of $\cc{O}_K$
    (see \href{https://pitmonticone.github.io/FLT3/docs/FLT3/Mathlib/NumberTheory/Cyclotomic/Rat.html#IsPrimitiveRoot.integralPowerBasis'}{this lemma},
    \href{https://pitmonticone.github.io/FLT3/docs/FLT3/Mathlib/NumberTheory/Cyclotomic/Rat.html#IsPrimitiveRoot.power_basis_int'_dim}{this lemma}
    and \href{https://pitmonticone.github.io/FLT3/docs/FLT3/Mathlib/NumberTheory/Cyclotomic/Rat.html#IsPrimitiveRoot.integralPowerBasis'_gen}{this lemma}
    which have already been formalised and included in Mathlib).\\
    For every $\xi \in \cc{O}_K$, we define $\pi_1(\xi)$ and $\pi_2(\xi)$ to be the first and second
    coordinates of $\xi$ with respect to the basis $\set{1,\eta} \in \cc{O}_K$, i.e.
    $$\xi = \pi_1(\xi) + \pi_2(\xi)\eta.$$
    By contradiction we assume that
    $$\exists m \in \Z \text{ such that } 3 \divides \eta - m,$$
    which implies that
    $$\exists x \in \cc{O}_K \text{ such that } \eta - m = 3 x.$$
    By linearity of $\pi_2$,
    $$\pi_2(\eta) = \pi_2(3x + m) = 3\pi_2(x) + \pi_2(m).$$
    Since $\pi_2(\eta) = 1$ and $\pi_2(m) = 0$, then we have that $3 \divides 1$, which is a contradiction.
\end{proof}

\begin{lemma}
    \label{lmm:lambda_sq}
    \lean{IsCyclotomicExtension.Rat.Three.lambda_sq}
    \leanok
    Let $K = \Q(\zeta_3)$ be the third cyclotomic field. \\
    Let $\cc{O}_K = \Z[\zeta_3]$ be the ring of integers of $K$. \\
    Let $\cc{O}^\times_K$ be the group of units of $\cc{O}_K$. \\
    Let $\zeta_3 \in K$ be any primitive third root of unity. \\
    Let $\eta \in \cc{O}_K$ be the element corresponding to $\zeta_3 \in K$. \\
    Let $\lambda \in \cc{O}_K$ be such that $\lambda = \eta -1$. \\\\
    Then $\lambda^2 = -3 \eta$.
\end{lemma}
\begin{proof}
    \leanok
    By definition we have that $\lambda = \eta -1$, which implies that
    $$\lambda^2 = (\eta - 1)^2 = \eta^2 - 2\eta + 1.$$
    Since $\eta$ corresponds to a root of the equation $x^2 + x + 1 = 0$, then $\eta^2 = -1 - \eta$.
    Substituting back, we can conclude that
    $$\lambda^2 = (-1 - \eta) - 2\eta + 1 = -3\eta.$$
\end{proof}

\begin{theorem}
    \label{lmm:eq_one_or_neg_one_of_unit_of_congruent}
    \lean{IsCyclotomicExtension.Rat.Three.eq_one_or_neg_one_of_unit_of_congruent}
    \leanok
    Let $K = \Q(\zeta_3)$ be the third cyclotomic field. \\
    Let $\cc{O}_K = \Z[\zeta_3]$ be the ring of integers of $K$. \\
    Let $\cc{O}^\times_K$ be the group of units of $\cc{O}_K$. \\
    Let $\zeta_3 \in K$ be any primitive third root of unity. \\
    Let $\eta \in \cc{O}_K$ be the element corresponding to $\zeta_3 \in K$. \\
    Let $\lambda \in \cc{O}_K$ be such that $\lambda = \eta -1$. \\
    Let $u \in \cc{O}^\times_K$ be a unit. \\\\
    If $\exists m \in \Z$ such that $\lambda^2 \divides u - m$, then
    $u = 1 \lor u = -1$. \\
    This is a special case of the Kummer's Lemma.
\end{theorem}
\begin{proof}
    \leanok
    \uses{thm:not_exists_int_three_dvd_sub, thm:mem, lmm:lambda_sq}
    By \Cref{lmm:lambda_sq}, we have that $-3\eta = \lambda^2 \divides u - m$, which implies that
    $3 \divides u - m$.\\
    By \Cref{thm:mem}, we know that $u \in \set{1, -1, \eta, -\eta, \eta^2, -\eta^2}$. \\
    We proceed by analysing each case:
    \begin{itemize}
        \item Case $u = 1 \lor u = -1$. This finishes the proof.
        \item Case $u = \eta$.\\
              Since $3 \divides u - m$, we have that $3 \divides \eta - m$, which contradicts
              \Cref{thm:not_exists_int_three_dvd_sub} forcing us to conclude that $u \neq \eta$.
        \item Case $u = -\eta$.\\
              Since $3 \divides u - m$, we have that $3 \divides - \eta - m$, then by properties of
              divisibility $3 \divides \eta + m$, which contradicts
              \Cref{thm:not_exists_int_three_dvd_sub} forcing us to conclude that $u \neq -\eta$.
        \item Case $u = \eta^2$.\\
              Since $3 \divides u - m$, we have that $3 \divides \eta^2 - m$, which contradicts
              \Cref{thm:not_exists_int_three_dvd_sub} since $\eta^2$ is a third root of unity
              (see \href{https://pitmonticone.github.io/FLT3/docs/Mathlib/RingTheory/RootsOfUnity/Basic.html#IsPrimitiveRoot.pow_of_coprime}{Mathlib}),
              forcing us to conclude that $u \neq \eta^2$.
        \item Case $u = -\eta^2$.\\
              Since $3 \divides u - m$, we have that $3 \divides - \eta^2 - m$, then by properties of
              divisibility $3 \divides \eta^2 + m$, which contradicts
              \Cref{thm:not_exists_int_three_dvd_sub} since $\eta^2$ is a third root of unity
              (see \href{https://pitmonticone.github.io/FLT3/docs/Mathlib/RingTheory/RootsOfUnity/Basic.html#IsPrimitiveRoot.pow_of_coprime}{Mathlib}),
              forcing us to conclude that $u \neq -\eta^2$.
    \end{itemize}
    Therefore, $u = 1 \lor u = -1$.
\end{proof}

\begin{lemma}
    \label{lmm:norm_lambda}
    \lean{IsCyclotomicExtension.Rat.Three.norm_lambda}
    \leanok
    Let $K = \Q(\zeta_3)$ be the third cyclotomic field. \\
    Let $\cc{O}_K = \Z[\zeta_3]$ be the ring of integers of $K$. \\
    Let $\cc{O}^\times_K$ be the group of units of $\cc{O}_K$. \\
    Let $\zeta_3 \in K$ be any primitive third root of unity. \\
    Let $\eta \in \cc{O}_K$ be the element corresponding to $\zeta_3 \in K$. \\
    Let $\lambda \in \cc{O}_K$ be such that $\lambda = \eta -1$. \\\\
    Then the norm of $\lambda$ is $3$.
\end{lemma}
\begin{proof}
    \leanok
    Since the third cyclotomic polynomial over $\Q$ is irreducible, then the norm of $\lambda$ is $3$
    by properties of primitive roots (see
    \href{https://pitmonticone.github.io/FLT3/docs/Mathlib/NumberTheory/Cyclotomic/PrimitiveRoots.html#IsPrimitiveRoot.sub_one_norm_prime}{this lemma}
    that has already been formalised and included in Mathlib).
\end{proof}

\begin{lemma}
    \label{lmm:norm_lambda_prime}
    \lean{IsCyclotomicExtension.Rat.Three.norm_lambda_prime}
    \leanok
    Let $K = \Q(\zeta_3)$ be the third cyclotomic field. \\
    Let $\cc{O}_K = \Z[\zeta_3]$ be the ring of integers of $K$. \\
    Let $\cc{O}^\times_K$ be the group of units of $\cc{O}_K$. \\
    Let $\zeta_3 \in K$ be any primitive third root of unity. \\
    Let $\eta \in \cc{O}_K$ be the element corresponding to $\zeta_3 \in K$. \\
    Let $\lambda \in \cc{O}_K$ be such that $\lambda = \eta -1$. \\\\
    Then the norm of $\lambda$ is a prime number.
\end{lemma}
\begin{proof}
    \leanok
    \uses{lmm:norm_lambda}
    It directly follows from \Cref{lmm:norm_lambda} since $3$ is a prime number.
\end{proof}

\begin{lemma}
    \label{lmm:lambda_dvd_three}
    \lean{IsCyclotomicExtension.Rat.Three.lambda_dvd_three}
    \leanok
    Let $K = \Q(\zeta_3)$ be the third cyclotomic field. \\
    Let $\cc{O}_K = \Z[\zeta_3]$ be the ring of integers of $K$. \\
    Let $\cc{O}^\times_K$ be the group of units of $\cc{O}_K$. \\
    Let $\zeta_3 \in K$ be any primitive third root of unity. \\
    Let $\eta \in \cc{O}_K$ be the element corresponding to $\zeta_3 \in K$. \\
    Let $\lambda \in \cc{O}_K$ be such that $\lambda = \eta -1$. \\\\
    Then $\lambda \divides 3$.
\end{lemma}
\begin{proof}
    \leanok
    \uses{lmm:norm_lambda}
    By properties of norms and divisibility, if the norm of an element in the ring of integers
    divides a number, then the element itself must divide that number.
    In this case, by \Cref{lmm:norm_lambda} we know that the norm of $\lambda$ is $3$, that divides $3$,
    which implies that $\lambda \divides 3$.
\end{proof}

\begin{theorem}
    \label{thm:zeta_sub_one_prime1}
    \lean{IsPrimitiveRoot.zeta_sub_one_prime'}
    \leanok
    Let $p \in \N$ be prime. \\\\
    If $\zeta_p$ is a primitive $p$-th root of unity, then $\zeta_p - 1$ is prime.
\end{theorem}
\begin{proof}
    \leanok
    This has already been formalised and included in \href{https://pitmonticone.github.io/FLT3/docs/FLT3/Mathlib/NumberTheory/Cyclotomic/Rat.html#IsPrimitiveRoot.zeta_sub_one_prime'}{Mathlib}.
\end{proof}

\begin{lemma}
    \label{lmm:lambda_prime}
    \lean{IsPrimitiveRoot.lambda_prime}
    \leanok
    Let $K = \Q(\zeta_3)$ be the third cyclotomic field. \\
    Let $\cc{O}_K = \Z[\zeta_3]$ be the ring of integers of $K$. \\
    Let $\cc{O}^\times_K$ be the group of units of $\cc{O}_K$. \\
    Let $\zeta_3 \in K$ be any primitive third root of unity. \\
    Let $\eta \in \cc{O}_K$ be the element corresponding to $\zeta_3 \in K$. \\
    Let $\lambda \in \cc{O}_K$ be such that $\lambda = \eta -1$. \\\\
    Then $\lambda$ is prime.
\end{lemma}
\begin{proof}
    \leanok
    \uses{thm:zeta_sub_one_prime1}
    Since $3$ is prime and $\zeta_3$ is a primitive third root of unity, then $\lambda$ is prime
    by \Cref{thm:zeta_sub_one_prime1}.
\end{proof}

\begin{lemma}
    \label{lmm:lambda_ne_zero}
    \lean{IsCyclotomicExtension.Rat.Three.lambda_ne_zero}
    \leanok
    Let $K = \Q(\zeta_3)$ be the third cyclotomic field. \\
    Let $\cc{O}_K = \Z[\zeta_3]$ be the ring of integers of $K$. \\
    Let $\cc{O}^\times_K$ be the group of units of $\cc{O}_K$. \\
    Let $\zeta_3 \in K$ be any primitive third root of unity. \\
    Let $\eta \in \cc{O}_K$ be the element corresponding to $\zeta_3 \in K$. \\
    Let $\lambda \in \cc{O}_K$ be such that $\lambda = \eta -1$. \\\\
    Then $\lambda \neq 0$.
\end{lemma}
\begin{proof}
    \leanok
    \uses{lmm:lambda_prime}
    It directly follows from \Cref{lmm:lambda_prime} since zero is not prime.
\end{proof}

\begin{lemma}
    \label{lmm:lambda_not_unit}
    \lean{IsCyclotomicExtension.Rat.Three.lambda_not_unit}
    \leanok
    Let $K = \Q(\zeta_3)$ be the third cyclotomic field. \\
    Let $\cc{O}_K = \Z[\zeta_3]$ be the ring of integers of $K$. \\
    Let $\cc{O}^\times_K$ be the group of units of $\cc{O}_K$. \\
    Let $\zeta_3 \in K$ be any primitive third root of unity. \\
    Let $\eta \in \cc{O}_K$ be the element corresponding to $\zeta_3 \in K$. \\
    Let $\lambda \in \cc{O}_K$ be such that $\lambda = \eta -1$. \\\\
    Then $\lambda$ is not a unit.
\end{lemma}
\begin{proof}
    \leanok
    \uses{lmm:lambda_prime}
    It directly follows from \Cref{lmm:lambda_prime} since prime numbers are not units.
\end{proof}

\begin{lemma}
    \label{lmm:card_quot}
    \lean{IsCyclotomicExtension.Rat.Three.card_quot}
    \leanok
    Let $K = \Q(\zeta_3)$ be the third cyclotomic field. \\
    Let $\cc{O}_K = \Z[\zeta_3]$ be the ring of integers of $K$. \\
    Let $\cc{O}^\times_K$ be the group of units of $\cc{O}_K$. \\
    Let $\zeta_3 \in K$ be any primitive third root of unity. \\
    Let $\eta \in \cc{O}_K$ be the element corresponding to $\zeta_3 \in K$. \\
    Let $\lambda \in \cc{O}_K$ be such that $\lambda = \eta -1$. \\
    Let $I$ be the ideal generated by $\lambda$. \\\\
    Then $\cc{O}_K / I$ has cardinality $3$.
\end{lemma}
\begin{proof}
    \leanok
    \uses{lmm:norm_lambda}
    It directly follows from \Cref{lmm:norm_lambda} by the fundamental properties of ideals.
\end{proof}

\begin{lemma}
    \label{lmm:two_ne_zero}
    \lean{IsCyclotomicExtension.Rat.Three.two_ne_zero}
    \leanok
    Let $K = \Q(\zeta_3)$ be the third cyclotomic field. \\
    Let $\cc{O}_K = \Z[\zeta_3]$ be the ring of integers of $K$. \\
    Let $\cc{O}^\times_K$ be the group of units of $\cc{O}_K$. \\
    Let $\zeta_3 \in K$ be any primitive third root of unity. \\
    Let $\eta \in \cc{O}_K$ be the element corresponding to $\zeta_3 \in K$. \\
    Let $\lambda \in \cc{O}_K$ be such that $\lambda = \eta -1$. \\
    Let $I$ be the ideal generated by $\lambda$. \\
    Let $2 \in \cc{O}_K ⧸ I$. \\\\
    Then $2 \neq 0$.
\end{lemma}
\begin{proof}
    \leanok
    \uses{lmm:norm_lambda}
    By contradiction we assume that $2 \in I$, then, by definition,
    $\lambda$ would divide $2 \in \cc{O}_K$.
    Recall from \Cref{lmm:norm_lambda} that the norm of $\lambda$ is $3$.
    If $\lambda$ divided $2$, then by properties of divisibility in number fields,
    the norm of $\lambda$ would also divide $2$.
    However $3 \notdivides 2$ showing a contradiction.
    Therefore, $\lambda \notdivides 2$, then $2 \notin I$, which implies
    that $2 \in \mathcal{O}_K / I$ is non-zero.
\end{proof}

\begin{lemma}
    \label{lmm:lambda_not_dvd_two}
    \lean{IsCyclotomicExtension.Rat.Three.lambda_not_dvd_two}
    \leanok
    Let $K = \Q(\zeta_3)$ be the third cyclotomic field. \\
    Let $\cc{O}_K = \Z[\zeta_3]$ be the ring of integers of $K$. \\
    Let $\cc{O}^\times_K$ be the group of units of $\cc{O}_K$. \\
    Let $\zeta_3 \in K$ be any primitive third root of unity. \\
    Let $\eta \in \cc{O}_K$ be the element corresponding to $\zeta_3 \in K$. \\
    Let $\lambda \in \cc{O}_K$ be such that $\lambda = \eta -1$. \\\\
    Then $\lambda \notdivides 2$.
\end{lemma}
\begin{proof}
    \leanok
    \uses{lmm:two_ne_zero}
    By contradiction we assume that $\lambda \divides 2$, that implies that $2 \in I$
    from which it follows that $2 = 0$ contradicting \Cref{lmm:two_ne_zero}
    forcing us to conclude that $\lambda \notdivides 2$.
\end{proof}

\begin{lemma}
    \label{lmm:univ_quot}
    \lean{IsCyclotomicExtension.Rat.Three.univ_quot}
    \leanok
    Let $K = \Q(\zeta_3)$ be the third cyclotomic field. \\
    Let $\cc{O}_K = \Z[\zeta_3]$ be the ring of integers of $K$. \\
    Let $\cc{O}^\times_K$ be the group of units of $\cc{O}_K$. \\
    Let $\zeta_3 \in K$ be any primitive third root of unity. \\
    Let $\eta \in \cc{O}_K$ be the element corresponding to $\zeta_3 \in K$. \\
    Let $\lambda \in \cc{O}_K$ be such that $\lambda = \eta -1$. \\
    Let $I$ be the ideal generated by $\lambda$. \\\\
    Then $\cc{O}_K / I = \set{0, 1, -1}$.
\end{lemma}
\begin{proof}
    \leanok
    \uses{lmm:card_quot, lmm:two_ne_zero}
    %It is obvious that $\cc{O}_K / I \supseteq \set{0, 1, -1}$.
    By \Cref{lmm:card_quot}, the cardinality of $\cc{O}_K / I$ is $3$,
    so it suffices to prove that $1,-1$ and $0$ are distinct.\\
    We proceed by contradiction analysing each case:
    \begin{itemize}
        \item Case $1 = -1$. By basic algebraic properties, $1 = -1$ implies that $2 = 0$,
              which contradicts \Cref{lmm:two_ne_zero} forcing us to conclude that $1 \neq -1$.
        \item Case $1 = 0$. Trivial contradiction.
        \item Case $-1 = 0$. It implies that $1 = 0$, which is a contradiction.
    \end{itemize}
\end{proof}

\begin{lemma}
    \label{lmm:dvd_or_dvd_sub_one_or_dvd_add_one}
    \lean{IsCyclotomicExtension.Rat.Three.dvd_or_dvd_sub_one_or_dvd_add_one}
    \leanok
    Let $K = \Q(\zeta_3)$ be the third cyclotomic field. \\
    Let $\cc{O}_K = \Z[\zeta_3]$ be the ring of integers of $K$. \\
    Let $\cc{O}^\times_K$ be the group of units of $\cc{O}_K$. \\
    Let $\zeta_3 \in K$ be any primitive third root of unity. \\
    Let $\eta \in \cc{O}_K$ be the element corresponding to $\zeta_3 \in K$. \\
    Let $\lambda \in \cc{O}_K$ be such that $\lambda = \eta -1$. \\
    Let $x \in \cc{O}_K$. \\\\
    Then $(\lambda \divides x) \lor (\lambda \divides x-1) \lor (\lambda \divides x+1)$.
\end{lemma}
\begin{proof}
    \leanok
    \uses{lmm:univ_quot}
    Let $I$ be the ideal generated by $\lambda$. Let $\pi : \cc{O}_K \to \cc{O}_K / I$.\\
    By \Cref{lmm:univ_quot}, we have that $\pi(x) \in \cc{O}_K / I = \set{0, 1, -1}$.\\
    We proceed by analysing each case:
    \begin{itemize}
        \item Case $\pi(x) = 0$. By properties of ideals, $\lambda \divides x$.
        \item Case $\pi(x) = 1$. Then $0=\pi(x)-1=\pi(x-1)$, which, by properties of ideals,
        implies that $\lambda \divides x-1$.
        \item Case $\pi(x) = -1$. Then $0=\pi(x)+1=\pi(x+1)$, which, by properties of ideals,
        implies that $\lambda \divides x+1$.
    \end{itemize}
\end{proof}

\begin{lemma}
    \label{lmm:toInteger_cube_eq_one}
    \lean{IsPrimitiveRoot.toInteger_cube_eq_one}
    \leanok
    Let $K = \Q(\zeta_3)$ be the third cyclotomic field. \\
    Let $\cc{O}_K = \Z[\zeta_3]$ be the ring of integers of $K$. \\
    Let $\cc{O}^\times_K$ be the group of units of $\cc{O}_K$. \\
    Let $\zeta_3 \in K$ be any primitive third root of unity. \\
    Let $\eta \in \cc{O}_K$ be the element corresponding to $\zeta_3 \in K$. \\\\
    Then $\eta^3 = 1$.
\end{lemma}
\begin{proof}
    \leanok
    Since $\zeta_3 \in K$ is a primitive third root of unity, then $\zeta_3^3 = 1$.
    Given that $\eta \in \cc{O}_K$ is the element corresponding to $\zeta_3 \in K$, then
    $\eta^3 = 1$ by the extension of the field properties into the ring of integers.
\end{proof}

\begin{lemma}
    \label{lmm:eta_isUnit}
    \lean{IsPrimitiveRoot.eta_isUnit}
    \leanok
    Let $K = \Q(\zeta_3)$ be the third cyclotomic field. \\
    Let $\cc{O}_K = \Z[\zeta_3]$ be the ring of integers of $K$. \\
    Let $\cc{O}^\times_K$ be the group of units of $\cc{O}_K$. \\
    Let $\zeta_3 \in K$ be any primitive third root of unity. \\
    Let $\eta \in \cc{O}_K$ be the element corresponding to $\zeta_3 \in K$. \\\\
    Then $\eta$ is a unit.
\end{lemma}
\begin{proof}
    \leanok
    \uses{lmm:toInteger_cube_eq_one}
    It directly follows from \Cref{lmm:toInteger_cube_eq_one}.
\end{proof}

\begin{lemma}
    \label{lmm:toInteger_eval_cyclo}
    \lean{IsPrimitiveRoot.toInteger_eval_cyclo}
    \leanok
    Let $K = \Q(\zeta_3)$ be the third cyclotomic field. \\
    Let $\cc{O}_K = \Z[\zeta_3]$ be the ring of integers of $K$. \\
    Let $\cc{O}^\times_K$ be the group of units of $\cc{O}_K$. \\
    Let $\zeta_3 \in K$ be any primitive third root of unity. \\
    Let $\eta \in \cc{O}_K$ be the element corresponding to $\zeta_3 \in K$. \\\\
    Then $\eta^2 + \eta + 1 = 0$.
\end{lemma}
\begin{proof}
    \leanok
    Since $\eta$ corresponds to a root of the equation $x^2 + x + 1 = 0$,
    then $\eta^2 + \eta + 1 = 0$.
\end{proof}

\begin{lemma}
    \label{lmm:cube_sub_one}
    \lean{IsCyclotomicExtension.Rat.Three.cube_sub_one}
    \leanok
    Let $K = \Q(\zeta_3)$ be the third cyclotomic field. \\
    Let $\cc{O}_K = \Z[\zeta_3]$ be the ring of integers of $K$. \\
    Let $\cc{O}^\times_K$ be the group of units of $\cc{O}_K$. \\
    Let $\zeta_3 \in K$ be any primitive third root of unity. \\
    Let $\eta \in \cc{O}_K$ be the element corresponding to $\zeta_3 \in K$. \\
    Let $x \in \cc{O}_K$. \\\\
    Then $x^3 - 1 = (x - 1)(x - \eta)(x - \eta^ 2)$.
\end{lemma}
\begin{proof}
    \leanok
    \uses{lmm:toInteger_cube_eq_one, lmm:toInteger_eval_cyclo}
    Applying \Cref{lmm:toInteger_cube_eq_one} and \Cref{lmm:toInteger_eval_cyclo}, we have that
    \begin{align*}
        (x - 1)(x - \eta)(x - \eta^ 2)
        &= x^3 - x^2 (\eta^2 + \eta + 1) + x (\eta^2 + \eta + \eta^3) - \eta^3 \\
        &= x^3 - x^2 (\eta^2 + \eta + 1) + x (\eta^2 + \eta + 1) - 1 \\
        &= x^3 - 1.
    \end{align*}
\end{proof}

\begin{lemma}
    \label{lmm:lambda_dvd_mul_sub_one_mul_sub_eta_add_one}
    \lean{IsCyclotomicExtension.Rat.Three.lambda_dvd_mul_sub_one_mul_sub_eta_add_one}
    \leanok
    Let $K = \Q(\zeta_3)$ be the third cyclotomic field. \\
    Let $\cc{O}_K = \Z[\zeta_3]$ be the ring of integers of $K$. \\
    Let $\cc{O}^\times_K$ be the group of units of $\cc{O}_K$. \\
    Let $\zeta_3 \in K$ be any primitive third root of unity. \\
    Let $\eta \in \cc{O}_K$ be the element corresponding to $\zeta_3 \in K$. \\
    Let $\lambda \in \cc{O}_K$ be such that $\lambda = \eta -1$. \\
    Let $x \in \cc{O}_K$. \\\\
    Then $\lambda \divides x(x - 1)(x - (\eta + 1))$.
\end{lemma}
\begin{proof}
    \leanok
    \uses{lmm:dvd_or_dvd_sub_one_or_dvd_add_one, lmm:lambda_dvd_three}
    By \Cref{lmm:dvd_or_dvd_sub_one_or_dvd_add_one}, we have that
    $$(\lambda \divides x) \lor (\lambda \divides x-1) \lor (\lambda \divides x+1).$$
    We proceed by analysing each case:
    \begin{itemize}
        \item Case $\lambda \divides x$. \\
              By properties of divisibility, we have that
              $\lambda \divides x(x - 1)(x - (\eta + 1))$.
        \item Case $\lambda \divides x-1$. \\
              By properties of divisibility, we have that
              $\lambda \divides x(x - 1)(x - (\eta + 1))$.
        \item Case $\lambda \divides x+1$.\\
              By properties of divisibility, it suffices to prove that
              $$\lambda \divides x - (\eta + 1) = x + 1 - (\eta - 1 + 3).$$
              By definition of $\lambda$, we have that
              $$x + 1 - (\eta - 1 + 3) = x + 1 - (\lambda + 3).$$
              By properties of divisibility and \Cref{lmm:lambda_dvd_three}, we can deduce that
              $\lambda \divides \lambda + 3$.\\
              Therefore, by properties of divisibility, we can conclude that
              $$\lambda \divides x(x - 1)(x - (\eta + 1)).$$
    \end{itemize}
\end{proof}

\begin{lemma}
    \label{lmm:lambda_pow_four_dvd_cube_sub_one_of_dvd_sub_one}
    \lean{IsCyclotomicExtension.Rat.Three.lambda_pow_four_dvd_cube_sub_one_of_dvd_sub_one}
    \leanok
    Let $K = \Q(\zeta_3)$ be the third cyclotomic field. \\
    Let $\cc{O}_K = \Z[\zeta_3]$ be the ring of integers of $K$. \\
    Let $\cc{O}^\times_K$ be the group of units of $\cc{O}_K$. \\
    Let $\zeta_3 \in K$ be any primitive third root of unity. \\
    Let $\eta \in \cc{O}_K$ be the element corresponding to $\zeta_3 \in K$. \\
    Let $\lambda \in \cc{O}_K$ be such that $\lambda = \eta -1$. \\
    Let $x \in \cc{O}_K$. \\\\
    If $\lambda\divides x - 1$, then $\lambda^ 4 \divides x^3 - 1$.
\end{lemma}
\begin{proof}
    \leanok
    \uses{lmm:cube_sub_one, lmm:lambda_dvd_mul_sub_one_mul_sub_eta_add_one}
    Let $\lambda \divides x - 1$, which is equivalent to say that
    $$\exists y\in \cc{O}_K \text{ such that } x - 1 = \lambda y.$$
    By ring properties and \Cref{lmm:cube_sub_one}, we have that
    $$x^3 - 1 = \lambda^3 (y (y - 1) (y - (\eta + 1))).$$
    By properties of divisibility and \Cref{lmm:lambda_dvd_mul_sub_one_mul_sub_eta_add_one},
    we can conclude that $$\lambda^ 4 \divides x^3 - 1.$$
\end{proof}

\begin{lemma}
    \label{lmm:lambda_pow_four_dvd_cube_add_one_of_dvd_add_one}
    \lean{IsCyclotomicExtension.Rat.Three.lambda_pow_four_dvd_cube_add_one_of_dvd_add_one}
    \leanok
    Let $K = \Q(\zeta_3)$ be the third cyclotomic field. \\
    Let $\cc{O}_K = \Z[\zeta_3]$ be the ring of integers of $K$. \\
    Let $\cc{O}^\times_K$ be the group of units of $\cc{O}_K$. \\
    Let $\zeta_3 \in K$ be any primitive third root of unity. \\
    Let $\eta \in \cc{O}_K$ be the element corresponding to $\zeta_3 \in K$. \\
    Let $\lambda \in \cc{O}_K$ be such that $\lambda = \eta -1$. \\
    Let $x \in \cc{O}_K$. \\\\
    If $\lambda \divides x + 1$, then $\lambda^4 \divides x ^ 3 + 1$.
\end{lemma}
\begin{proof}
    \leanok
    \uses{lmm:lambda_pow_four_dvd_cube_sub_one_of_dvd_sub_one}
    By properties of divisibility, if $\lambda \divides x + 1$
    then $$\lambda \divides - (x + 1) = (- x) - 1.$$
    By \Cref{lmm:lambda_dvd_mul_sub_one_mul_sub_eta_add_one}, we can deduce that
    $$\lambda^ 4 \divides (-x)^3 - 1.$$
    By divisibility and ring properties we can conclude that $$\lambda^ 4 \divides x^3 + 1.$$
\end{proof}

\begin{lemma}
    \label{lmm:lambda_pow_four_dvd_cube_sub_one_or_add_one_of_lambda_not_dvd}
    \lean{IsCyclotomicExtension.Rat.Three.lambda_pow_four_dvd_cube_sub_one_or_add_one_of_lambda_not_dvd}
    \leanok
    Let $K = \Q(\zeta_3)$ be the third cyclotomic field. \\
    Let $\cc{O}_K = \Z[\zeta_3]$ be the ring of integers of $K$. \\
    Let $\cc{O}^\times_K$ be the group of units of $\cc{O}_K$. \\
    Let $\zeta_3 \in K$ be any primitive third root of unity. \\
    Let $\eta \in \cc{O}_K$ be the element corresponding to $\zeta_3 \in K$. \\
    Let $\lambda \in \cc{O}_K$ be such that $\lambda = \eta -1$. \\
    Let $x \in \cc{O}_K$. \\\\
    If $\lambda \notdivides x$, then $(\lambda^4 \divides x^3 - 1)
    \lor (\lambda^4 \divides x^3 + 1)$.
\end{lemma}
\begin{proof}
    \leanok
    \uses{lmm:dvd_or_dvd_sub_one_or_dvd_add_one,
    lmm:lambda_pow_four_dvd_cube_sub_one_of_dvd_sub_one,
    lmm:lambda_pow_four_dvd_cube_add_one_of_dvd_add_one}
    By \Cref{lmm:dvd_or_dvd_sub_one_or_dvd_add_one}, we have that
    $$(\lambda \divides x) \lor (\lambda \divides x-1) \lor (\lambda \divides x+1).$$
    We proceed by analysing each case:
    \begin{itemize}
        \item Case $\lambda \divides x$. From trivially contradictory hypotheses we can conclude that
        $$(\lambda^4 \divides x^3 - 1) \lor (\lambda^4 \divides x^3 + 1).$$
        \item Case $\lambda \divides x-1$. By \Cref{lmm:lambda_pow_four_dvd_cube_sub_one_of_dvd_sub_one},
        we have that $\lambda^ 4 \divides x^3 - 1$, which implies that
        $$(\lambda^4 \divides x^3 - 1) \lor (\lambda^4 \divides x^3 + 1).$$
        \item Case $\lambda \divides x+1$. By \Cref{lmm:lambda_pow_four_dvd_cube_add_one_of_dvd_add_one},
        we have that $\lambda^ 4 \divides x^3 + 1$, which implies that
        $$(\lambda^4 \divides x^3 - 1) \lor (\lambda^4 \divides x^3 + 1).$$
    \end{itemize}
\end{proof}
% Fermat's Last Theorem for Exponent 3

\chapter{Fermat's Last Theorem for Exponent 3}
%\label{chap:flt3}
%\addcontentsline{toc}{chapter}{Fermat's Last Theorem for Exponent 3}

\section{Case 1}

\begin{lemma}
  \label{lmm:cube_of_castHom_ne_zero}
  \lean{cube_of_castHom_ne_zero}
  \leanok
  Let $\Z_9$ be the ring of integers modulo $9$. \\
  Let $\Z_3$ be the ring of integers modulo $3$. \\
  Let $n \in \Z_9$. \\
  Let $\phi : \Z_9 \to \Z_3$ be the canonical ring homomorphism. \\
  Let $\phi(n) \neq 0$. \\ \\
  Then $n^3=1 \lor n^3=8$.
\end{lemma}
\begin{proof}
  \leanok
  This has already been formalised and included in \href{https://pitmonticone.github.io/FLT3/docs/FLT3/Mathlib/NumberTheory/FLT/Three.html#cube_of_castHom_ne_zero}{Mathlib}.
\end{proof}

\begin{lemma}
  \label{lmm:cube_of_not_dvd}
  \lean{cube_of_not_dvd}
  \leanok
  Let $n \in \N$. \\
  Let $\left[n \right] \in \Z_9$. \\
  Let $3 \notdivides n$. \\ \\
  Then $\left[n \right]^3 = 1 \lor \left[n \right]^3 = 8$.
\end{lemma}
\begin{proof}
  \leanok
  \uses{lmm:cube_of_castHom_ne_zero}
  By \Cref{lmm:cube_of_castHom_ne_zero}, we can conclude that $\left[n \right]^3 = 1 \lor \left[n \right]^3 = 8$.
\end{proof}

\begin{theorem}[Fermat's Last Theorem for 3: Case 1]
    \label{thm:fermatLastTheoremThree_case_1}
    \lean{fermatLastTheoremThree_case_1}
    \leanok
    Let $a, b, c \in \N$. \\
    Let $3 \notdivides abc$. \\\\
    Then $a ^ 3 + b ^ 3 \neq c ^ 3$.
\end{theorem}
\begin{proof}
  \leanok
  \uses{lmm:cube_of_not_dvd}
  By hypothesis we know that $3 \notdivides abc$, which implies that $3 \notdivides a$, $3 \notdivides b$ and $3 \notdivides c$. \\
  It is enough to apply \Cref{lmm:cube_of_not_dvd} repeatedly and compute each case.
\end{proof}

\section{Case 2}

\begin{lemma}
  \label{lmm:three_dvd_gcd_of_dvd_a_of_dvd_b}
  \lean{three_dvd_gcd_of_dvd_a_of_dvd_b}
  \leanok
  Let $a, b, c \in \N$. \\
  Let $3 \divides a$ and $3 \divides b$. \\
  Let $a ^ 3 + b ^ 3 = c ^ 3$. \\\\
  Then $3 \divides \gcd(a,b,c)$.
\end{lemma}
\begin{proof}
  \leanok
  By hypothesis we have that $3 \divides a^3 + b^3 = c^3$, which implies that $3 \divides c$,
  from which we can conclude that $3 \divides \gcd(a,b,c)$.
\end{proof}

\begin{lemma}
  \label{lmm:three_dvd_gcd_of_dvd_a_of_dvd_c}
  \lean{three_dvd_gcd_of_dvd_a_of_dvd_c}
  \leanok
  Let $a, b, c \in \N$. \\
  Let $3 \divides a$ and $3 \divides c$. \\
  Let $a ^ 3 + b ^ 3 = c ^ 3$. \\\\
  Then $3 \divides \gcd(a,b,c)$.
\end{lemma}
\begin{proof}
  \leanok
  By hypothesis we have that $3 \divides c^3 - a^3 = b^3$, which implies that $3 \divides b$,
  from which we can conclude that $3 \divides \gcd(a,b,c)$.
\end{proof}

\begin{lemma}
  \label{lmm:three_dvd_gcd_of_dvd_b_of_dvd_c}
  \lean{three_dvd_gcd_of_dvd_b_of_dvd_c}
  \leanok
  Let $a, b, c \in \N$. \\
  Let $3 \divides b$ and $3 \divides c$. \\
  Let $a ^ 3 + b ^ 3 = c ^ 3$. \\\\
  Then $3 \divides \gcd(a,b,c)$.
\end{lemma}
\begin{proof}
  \leanok
  By hypothesis we have that $3 \divides c^3 - b^3 = a^3$, which implies that $3 \divides a$,
  from which we can conclude that $3 \divides \gcd(a,b,c)$.
\end{proof}

\begin{lemma}
  \label{lmm:fermatLastTheoremWith_of_fermatLastTheoremWith_coprime}
  \lean{fermatLastTheoremWith_of_fermatLastTheoremWith_coprime}
  \leanok
  Let $R$ be a commutative semiring, domain and normalised gcd monoid.\\% ASK EXPERTS
  Let $a, b, c \in R$. \\
  Let $n \in \N$. \\\\
  Then, to prove Fermat's Last Theorem for exponent $n$ in $R$,
  one can assume, without loss of generality, that $\gcd(a,b,c)=1$.
\end{lemma}
\begin{proof}
  \leanok
  This has already been formalised and included in \href{https://pitmonticone.github.io/FLT3/docs/FLT3/Mathlib/NumberTheory/FLT/Basic.html#fermatLastTheoremWith_of_fermatLastTheoremWith_coprime}{Mathlib}.
\end{proof}

\begin{theorem}
  \label{thm:fermatLastTheoremThree_of_three_dvd_only_c}
  \lean{fermatLastTheoremThree_of_three_dvd_only_c}
  \leanok
  To prove \Cref{thm:fermatLastTheoremThree}, it suffices to prove that
  $$\forall a, b, c \in \Z, \text{ if } c \neq 0 \text{ and } 3 \notdivides a \text{ and }
  3 \notdivides b \text{ and } 3 \divides c \text{ and } \gcd(a,b)=1,
  \text{ then } a^3 + b^3 \neq c^3.$$
  Equivalently, $$\forall a, b, c \in \Z, \text{ if } c \neq 0 \text{ and } 3 \notdivides a \text{ and }
  3 \notdivides b \text{ and } 3 \divides c \text{ and } \gcd(a,b)=1,
  \text{ then } a^3 + b^3 \neq c^3$$ implies \Cref{thm:fermatLastTheoremThree}.
\end{theorem}
\begin{proof}
  \leanok
  \uses{lmm:fermatLastTheoremWith_of_fermatLastTheoremWith_coprime,
  thm:fermatLastTheoremThree_case_1,
  lmm:three_dvd_gcd_of_dvd_a_of_dvd_b,
  lmm:three_dvd_gcd_of_dvd_a_of_dvd_c,
  lmm:three_dvd_gcd_of_dvd_b_of_dvd_c}
  By contradiction we assume that
  $$\exists a,b,c \in \N \smallsetminus \set{0} \text{ such that } a^3 + b^3 = c^3.$$
  By \Cref{lmm:fermatLastTheoremWith_of_fermatLastTheoremWith_coprime}
  we can assume that $\gcd(a,b,c)=1$. \\
  By \Cref{thm:fermatLastTheoremThree_case_1} we can assume that $3 \divides a b c$,
  from which it follows that $$(3 \divides a) \lor (3 \divides b) \lor (3 \divides c).$$
  We proceed by analysing each case:
  \begin{itemize}
      \item Case $3 \divides a$. \\
      Let $a'=-c$, $b'=b$, $c'=-a$, then $3 \divides c'$ and
      $$(a'\neq 0) \land (b'\neq 0) \land (c' \neq 0).$$
      Then $3 \notdivides a'$ since otherwise by \Cref{lmm:three_dvd_gcd_of_dvd_a_of_dvd_c}
      we would have that $3 \divides \gcd(a,b,c)=1$ which is absurd. \\
      Analogously, by \Cref{lmm:three_dvd_gcd_of_dvd_a_of_dvd_b} we have that $3 \notdivides b'$.\\
      By contradiction we assume that $\gcd(a',b') \neq 1$ which, by basic divisibility properties,
      implies that there is a prime $p$ such that $p \divides a'$ and $p \divides b'$.
      It follows that $p \divides b'^3 + a'^3 = b^3 - c^3 = -a^3$, which implies that $p \divides a$.\\
      Therefore $p \divides \gcd(a,b,c)=1$ which is absurd. \\
      Moreover, we have that $a'^3 + b'^3 = -c^3 + b^3 = -a^3 = c'^3$ that contradicts our hypothesis.
      \item Case $3 \divides b$. \\
      Let $a'=a$, $b'=-c$, $c'=-b$.\\
      The rest of the proof is analogous to the first case using \Cref{lmm:three_dvd_gcd_of_dvd_a_of_dvd_b} and
      \Cref{lmm:three_dvd_gcd_of_dvd_b_of_dvd_c}.
      \item Case $3 \divides c$.
      Let $a'=a$, $b'=b$, $c'=c$.\\
      The rest of the proof is analogous to the first case using \Cref{lmm:three_dvd_gcd_of_dvd_a_of_dvd_c} and
      \Cref{lmm:three_dvd_gcd_of_dvd_b_of_dvd_c}.
  \end{itemize}
  Therefore, we can conclude that $a^3 + b^3 \neq c^3$.
\end{proof}

\begin{definition}[Solution']
  \label{def:Solution1}
  \lean{Solution'}
  \leanok
  Let $a, b, c \in \cc{O}_K$ such that $c \neq 0$ and $\gcd(a,b)=1$.\\
  Let $\lambda \notdivides a$, $\lambda \notdivides b$ and $\lambda \divides c$. \\\\
  A $\boldsymbol{solution'}$ is a tuple $S'=(a, b, c, u)$
  satisfying the equation $a^3 + b^3 = u c^3.$
\end{definition}

\begin{definition}[Solution]
  \label{def:Solution}
  \lean{Solution}
  \leanok
  Let $a, b, c \in \cc{O}_K$ such that $c \neq 0$ and $\gcd(a,b)=1$.\\
  Let $\lambda \notdivides a$, $\lambda \notdivides b$, $\lambda \divides c$ and
  $\lambda^2 \divides a+b$. \\\\
  A $\boldsymbol{solution}$ is a tuple $S=(a, b, c, u)$
  satisfying the equation $a^3 + b^3 = u c^3$.
\end{definition}

\begin{definition}[Multiplicity of Solution']
  \label{def:Solution1_Multiplicity}
  \lean{Solution'.multiplicity}
  \leanok
  \uses{def:Solution1}
  Let $S'=(a, b, c, u)$ be a $solution'$. \\\\
  The $\boldsymbol{multiplicity}$ of $S'$ is the largest $n \in \N$ such that
  $\lambda^n \divides c$.
\end{definition}

\begin{definition}[Multiplicity of Solution]
  \label{def:Solution_Multiplicity}
  \lean{Solution.multiplicity}
  \leanok
  \uses{def:Solution}
  Let $S=(a, b, c, u)$ be a $solution$. \\\\
  The $\boldsymbol{multiplicity}$ of $S$ is the largest $n \in \N$ such that
  $\lambda^n \divides c$.
\end{definition}

\begin{definition}[Minimal Solution]
  \label{def:Solution_Minimal}
  \lean{Solution.isMinimal}
  \leanok
  \uses{def:Solution}
  Let $S=(a, b, c, u)$ be a $solution$. \\\\
  We say that $S$ is $\boldsymbol{minimal}$ if for all solutions $S_1=(a_1,b_1,c_1,u_1)$,
  the $multiplicity$ of $S$ is less than or equal to the $multiplicity$ of $S_1$.
\end{definition}

\begin{lemma}
  \label{lmm:multiplicity_lambda_c_finite}
  \lean{Solution'.multiplicity_lambda_c_finite}
  \leanok
  \uses{def:Solution1}
  Let $S'=(a, b, c, u)$ be a $solution'$. \\\\
  Then the multiplicity of $S'$ is finite.
\end{lemma}
\begin{proof}
  \leanok
  \uses{lmm:lambda_not_unit}
  It directly follows from \Cref{lmm:lambda_not_unit}.
\end{proof}

\begin{lemma}
  \label{lmm:exists_minimal}
  \lean{Solution.exists_minimal}
  \leanok
  \uses{def:Solution, def:Solution_Minimal}
  Let $S$ be a $solution$ with multiplicity $n$. \\\\
  Then there is a minimal solution $S_1$.
\end{lemma}
\begin{proof}
  \leanok
  Straightforward since $n \in \N$ and $\N$ is well-ordered.
\end{proof}

\begin{lemma}
  \label{lmm:a_cube_b_cube_same_congr}
  \lean{a_cube_b_cube_same_congr}
  \leanok
  \uses{def:Solution1}
  Let $S'=(a, b, c, u)$ be a $solution'$. \\\\
  Then $\lambda^4 \divides a^3 - 1 \land \lambda^4 \divides b^3 + 1$ or
  $\lambda^4 \divides a^3 + 1 \land \lambda^4 \divides b^3 - 1$.
\end{lemma}
\begin{proof}
  \leanok
  \uses{lmm:lambda_pow_four_dvd_cube_sub_one_or_add_one_of_lambda_not_dvd,
  lmm:lambda_not_dvd_two}
  Since $\lambda \notdivides a$, then
  $\lambda^4 \divides a^3 - 1 \lor \lambda^4 \divides a^3 + 1$ by
  \Cref{lmm:lambda_pow_four_dvd_cube_sub_one_or_add_one_of_lambda_not_dvd}.
  Since $\lambda \notdivides b$, then
  $\lambda^4 \divides b^3 - 1 \lor \lambda^4 \divides b^3 + 1$ by
  \Cref{lmm:lambda_pow_four_dvd_cube_sub_one_or_add_one_of_lambda_not_dvd}.
  We proceed by analysing each case:
  \begin{itemize}
      \item Case $\lambda^4 \divides a^3 - 1 \land \lambda^4 \divides b^3 - 1$.
      Since $\lambda \divides c$ we have that $\lambda \divides c^3-(a^3-1)-(b^3-1) = 2$,
      which is absurd by \Cref{lmm:lambda_not_dvd_two}.
      \item Case $\lambda^4 \divides a^3 + 1 \land \lambda^4 \divides b^3 + 1$.
      Since $\lambda \divides c$ we have that $\lambda \divides (a^3-1)+(b^3-1)-c^3 = 2$,
      which is absurd by \Cref{lmm:lambda_not_dvd_two}.
      \item Case $\lambda^4 \divides a^3 - 1 \land \lambda^4 \divides b^3 + 1$. Trivial.
      \item Case $\lambda^4 \divides a^3 + 1 \land \lambda^4 \divides b^3 - 1$. Trivial.
  \end{itemize}
\end{proof}

\begin{lemma}
  \label{lmm:lambda_pow_four_dvd_c_cube}
  \lean{lambda_pow_four_dvd_c_cube}
  \leanok
  \uses{def:Solution1}
  Let $S'=(a, b, c, u)$ be a $solution'$. \\\\
  Then $\lambda^4 \divides c^3$.
\end{lemma}
\begin{proof}
  \leanok
  \uses{lmm:a_cube_b_cube_same_congr}
  Apply \Cref{lmm:a_cube_b_cube_same_congr} and then compute each case.
\end{proof}

\begin{lemma}
  \label{lmm:lambda_pow_two_dvd_c}
  \lean{lambda_pow_two_dvd_c}
  \leanok
  \uses{def:Solution1}
  Let $S'=(a, b, c, u)$ be a $solution'$. \\\\
  Then $\lambda^2 \divides c$.
\end{lemma}
\begin{proof}
  \leanok
  \uses{lmm:multiplicity_lambda_c_finite,
  lmm:lambda_pow_four_dvd_c_cube, lmm:lambda_prime}
  Apply \Cref{lmm:lambda_pow_four_dvd_c_cube}.
\end{proof}

\begin{lemma}
  \label{lmm:Solution1_two_le_multiplicity}
  \lean{Solution'.two_le_multiplicity}
  \leanok
  \uses{def:Solution1}
  Let $S'=(a, b, c, u)$ be a $solution'$ with multiplicity $n$.\\\\
  Then $2 \leq n$.
\end{lemma}
\begin{proof}
  \leanok
  \uses{lmm:lambda_pow_two_dvd_c}
  It directly follows from \Cref{lmm:lambda_pow_two_dvd_c}.
\end{proof}

\begin{lemma}
  \label{lmm:Solution_two_le_multiplicity}
  \lean{Solution.two_le_multiplicity}
  \leanok
  \uses{def:Solution}
  Let $S=(a, b, c, u)$ be a $solution$ with multiplicity $n$.\\\\
  Then $2 \leq n$.
\end{lemma}
\begin{proof}
  \leanok
  \uses{lmm:Solution1_two_le_multiplicity}
  It directly follows from \Cref{lmm:Solution1_two_le_multiplicity}.
\end{proof}

\begin{lemma}
  \label{lmm:cube_add_cube_eq_mul}
  \lean{cube_add_cube_eq_mul}
  \leanok
  \uses{def:Solution1}
  Let $K = \Q(\zeta_3)$ be the third cyclotomic field. \\
  Let $\cc{O}_K = \Z[\zeta_3]$ be the ring of integers of $K$. \\
  Let $\cc{O}^\times_K$ be the group of units of $\cc{O}_K$. \\
  Let $\zeta_3 \in K$ be any primitive third root of unity. \\
  Let $\eta \in \cc{O}_K$ be the element corresponding to $\zeta_3 \in K$. \\
  Let $S'=(a, b, c, u)$ be a $solution'$.\\\\
  Then $a^3 + b^3 = (a + b) (a + \eta b) (a + \eta^2 b)$.
\end{lemma}
\begin{proof}
  \leanok
  \uses{lmm:toInteger_cube_eq_one, lmm:toInteger_eval_cyclo}
  Straightforward calculation using \Cref{lmm:toInteger_cube_eq_one}
  and \Cref{lmm:toInteger_eval_cyclo}.
\end{proof}

\begin{lemma}
  \label{lmm:lambda_sq_dvd_or_dvd_or_dvd}
  \lean{lambda_sq_dvd_or_dvd_or_dvd}
  \leanok
  \uses{def:Solution1}
  Let $K = \Q(\zeta_3)$ be the third cyclotomic field. \\
  Let $\cc{O}_K = \Z[\zeta_3]$ be the ring of integers of $K$. \\
  Let $\cc{O}^\times_K$ be the group of units of $\cc{O}_K$. \\
  Let $\zeta_3 \in K$ be any primitive third root of unity. \\
  Let $\eta \in \cc{O}_K$ be the element corresponding to $\zeta_3 \in K$. \\
  Let $\lambda \in \cc{O}_K$ be such that $\lambda = \eta -1$. \\
  Let $S'=(a, b, c, u)$ be a $solution'$.\\\\
  Then $(\lambda^2 \divides a + b) \lor (\lambda^2 \divides a +
  \eta b) \lor (\lambda^2 \divides a + \eta^2 b)$.
\end{lemma}
\begin{proof}
  \leanok
  \uses{lmm:lambda_pow_two_dvd_c, lmm:cube_add_cube_eq_mul,
  lmm:lambda_prime}
  By contradiction we assume that
  $$(\lambda^2 \notdivides a + b) \land (\lambda^2 \notdivides a +
  \eta b) \land (\lambda^2 \notdivides a + \eta^2 b).$$
  Then, by definition, the multiplicity of $\lambda$ in $a + b$, in $a +
  \eta b$ and in $a + \eta^2 b$ is less than $2$.
  By properties of divisibility, \Cref{lmm:lambda_pow_two_dvd_c} and \Cref{lmm:cube_add_cube_eq_mul},
  we have that
  $$\lambda^6 ∣ u c^3 = a^3 + b^3 = (a + b) (a + \eta b) (a + \eta^2 b).$$
  Then, the multiplicity of $\lambda$ in $(a + b) (a + \eta b) (a + \eta^2 b)$ is greater than
  or equal to $6$. \\
  By \Cref{lmm:lambda_prime} $\lambda$ is prime, so we have that the multiplicity of $\lambda$
  in $(a + b) (a + \eta b) (a + \eta^2 b)$ is the sum of the multiplicities of $\lambda$ in
  $a + b$, in $a + \eta b$ and in $a + \eta^2 b$, which is less than $6$.
  This is a contradiction that forces us to conclude that
  $$(\lambda^2 \divides a + b) \lor (\lambda^2 \divides a +
  \eta b) \lor (\lambda^2 \divides a + \eta^2 b).$$
\end{proof}

\begin{lemma}
  \label{lmm:ex_dvd_a_add_b}
  \lean{ex_dvd_a_add_b}
  \leanok
  \uses{def:Solution1}
  Let $S'=(a, b, c, u)$ be a $solution'$.\\\\
  Then $\exists a_1,b_1 \in \cc{O}_k$ such that $S_1=(a_1,b_1,c,u)$ is a $solution$.
\end{lemma}
\begin{proof}
  \leanok
  \uses{lmm:lambda_sq_dvd_or_dvd_or_dvd, lmm:toInteger_cube_eq_one, lmm:eta_isUnit}
  By \Cref{lmm:lambda_sq_dvd_or_dvd_or_dvd}, we have that
  $$(\lambda^2 \divides a + b) \lor (\lambda^2 \divides a +
  \eta b) \lor (\lambda^2 \divides a + \eta^2 b).$$
  We proceed by analysing each case:
  \begin{itemize}
      \item Case $\lambda^2 \divides a + b$. Trivial using $a_1=a$ and $b_1=b$.
      \item Case $\lambda^2 \divides a + \eta b$. Let $a_1=a$ and $b_1=\eta b$. \\
      By \Cref{lmm:toInteger_cube_eq_one}, we have that $a^3 + (\eta b)^3 = a^3 + b^3 = u c^3$.\\
      By properties of coprimes and \Cref{lmm:eta_isUnit}, we have that
      $\gcd(a,b)=1$ implies that $\gcd(a,\eta b)=1$.\\
      Since $a_1=a$, we already know that $\lambda \notdivides a = a_1$.\\
      By contradiction we assume that $\lambda \divides b_1 = \eta b$, which,
      by \Cref{lmm:toInteger_cube_eq_one}, it implies that $\lambda \divides \eta^2 \eta b = b$
      that contradicts our assumption, forcing us to conclude that $\lambda \notdivides b_1$.
      \item Case $\lambda^2 \divides a + \eta^2 b$. Let $a_1=a$ and $b_1=\eta^2 b$. \\
      By \Cref{lmm:toInteger_cube_eq_one}, we have that $a^3 + (\eta^2 b)^3 = a^3 + b^3 = u c^3$.\\
      By properties of coprimes and \Cref{lmm:eta_isUnit}, we have that
      $\gcd(a,b)=1$ implies that $\gcd(a,\eta^2 b)=1$.\\
      Since $a_1=a$, we already know that $\lambda \notdivides a = a_1$.\\
      By contradiction we assume that $\lambda \divides b_1 = \eta^2 b$, which,
      by \Cref{lmm:toInteger_cube_eq_one}, it implies that $\lambda \divides \eta \eta^2 b = b$
      that contradicts our assumption, forcing us to conclude that $\lambda \notdivides b_1$.
  \end{itemize}
  Therefore, we can conclude that
  $\exists a_1,b_1 \in \cc{O}_k$ such that $S_1=(a_1,b_1,c,u)$ is a $solution$.
\end{proof}

\begin{lemma}
  \label{lmm:exists_Solution_of_Solution1}
  \lean{exists_Solution_of_Solution'}
  \leanok
  \uses{def:Solution1, def:Solution}
  Let $S'$ be a $solution'$ with multiplicity $n$.\\\\
  Then there is a $solution\text{ }S$ with multiplicity $n$.
\end{lemma}
\begin{proof}
  \leanok
  \uses{lmm:ex_dvd_a_add_b}
  Let $S'=(a',b',c',u')$. Let $a, b$ be the units given by \Cref{lmm:ex_dvd_a_add_b}.
  Then $S=(a,b,c',u')$ is a $solution'$ with multiplicity $n$.
\end{proof}

\begin{lemma}
  \label{lmm:a_add_eta_b}
  \lean{Solution.a_add_eta_b}
  \leanok
  \uses{def:Solution}
  Let $K = \Q(\zeta_3)$ be the third cyclotomic field. \\
  Let $\cc{O}_K = \Z[\zeta_3]$ be the ring of integers of $K$. \\
  Let $\cc{O}^\times_K$ be the group of units of $\cc{O}_K$. \\
  Let $\zeta_3 \in K$ be any primitive third root of unity. \\
  Let $\eta \in \cc{O}_K$ be the element corresponding to $\zeta_3 \in K$. \\
  Let $S=(a, b, c, u)$ be a $solution$.\\\\
  Then $a + \eta  b = (a + b) + \lambda  b$.
\end{lemma}
\begin{proof}
  \leanok
  Trivial calculation.
\end{proof}

\begin{lemma}
  \label{lmm:lambda_dvd_a_add_eta_mul_b}
  \lean{Solution.lambda_dvd_a_add_eta_mul_b}
  \leanok
  \uses{def:Solution}
  Let $K = \Q(\zeta_3)$ be the third cyclotomic field. \\
  Let $\cc{O}_K = \Z[\zeta_3]$ be the ring of integers of $K$. \\
  Let $\cc{O}^\times_K$ be the group of units of $\cc{O}_K$. \\
  Let $\zeta_3 \in K$ be any primitive third root of unity. \\
  Let $\eta \in \cc{O}_K$ be the element corresponding to $\zeta_3 \in K$. \\
  Let $\lambda \in \cc{O}_K$ be such that $\lambda = \eta -1$. \\
  Let $S=(a, b, c, u)$ be a $solution$.\\\\
  Then $\lambda \divides a + \eta  b$.
\end{lemma}
\begin{proof}
  \leanok
  \uses{lmm:a_add_eta_b}
  Trivial since $\lambda \divides a+b$.
\end{proof}

\begin{lemma}
  \label{lmm:lambda_dvd_a_add_eta_sq_mul_b}
  \lean{Solution.lambda_dvd_a_add_eta_sq_mul_b}
  \leanok
  \uses{def:Solution}
  Let $K = \Q(\zeta_3)$ be the third cyclotomic field. \\
  Let $\cc{O}_K = \Z[\zeta_3]$ be the ring of integers of $K$. \\
  Let $\cc{O}^\times_K$ be the group of units of $\cc{O}_K$. \\
  Let $\zeta_3 \in K$ be any primitive third root of unity. \\
  Let $\eta \in \cc{O}_K$ be the element corresponding to $\zeta_3 \in K$. \\
  Let $\lambda \in \cc{O}_K$ be such that $\lambda = \eta -1$. \\
  Let $S=(a, b, c, u)$ be a $solution$.\\\\
  Then $\lambda \divides a + \eta^2  b$.
\end{lemma}
\begin{proof}
  \leanok
  Since $\lambda \divides a+b$, then
  $\lambda \divides (a + b) + \lambda^2  b + 2  \lambda  b
  = a + \eta^2  b$.
\end{proof}

\begin{lemma}
  \label{lmm:lambda_sq_not_dvd_a_add_eta_mul_b}
  \lean{Solution.lambda_sq_not_dvd_a_add_eta_mul_b}
  \leanok
  \uses{def:Solution}
  Let $K = \Q(\zeta_3)$ be the third cyclotomic field. \\
  Let $\cc{O}_K = \Z[\zeta_3]$ be the ring of integers of $K$. \\
  Let $\cc{O}^\times_K$ be the group of units of $\cc{O}_K$. \\
  Let $\zeta_3 \in K$ be any primitive third root of unity. \\
  Let $\eta \in \cc{O}_K$ be the element corresponding to $\zeta_3 \in K$. \\
  Let $\lambda \in \cc{O}_K$ be such that $\lambda = \eta -1$. \\
  Let $S=(a, b, c, u)$ be a $solution$.\\\\
  Then $\lambda^2 \notdivides a + \eta b$.
\end{lemma}
\begin{proof}
  \leanok
  \uses{lmm:a_add_eta_b, lmm:lambda_ne_zero}
  By contradiction we assume that $\lambda^2 \divides a + \eta b$, which implies that
  $\lambda^2 \divides a + b + \lambda  b$ by \Cref{lmm:a_add_eta_b}.
  Since $\lambda^2 \divides a+b$, then $\lambda^2 \divides \lambda  b$, which implies that
  $\lambda \divides b$, that contradicts \Cref{def:Solution} forcing us to conclude that
  $\lambda^2 \notdivides a + \eta b$.
\end{proof}

\begin{lemma}
  \label{lmm:lambda_sq_not_dvd_a_add_eta_sq_mul_b}
  \lean{Solution.lambda_sq_not_dvd_a_add_eta_sq_mul_b}
  \leanok
  \uses{def:Solution}
  Let $K = \Q(\zeta_3)$ be the third cyclotomic field. \\
  Let $\cc{O}_K = \Z[\zeta_3]$ be the ring of integers of $K$. \\
  Let $\cc{O}^\times_K$ be the group of units of $\cc{O}_K$. \\
  Let $\zeta_3 \in K$ be any primitive third root of unity. \\
  Let $\eta \in \cc{O}_K$ be the element corresponding to $\zeta_3 \in K$. \\
  Let $\lambda \in \cc{O}_K$ be such that $\lambda = \eta -1$. \\
  Let $S=(a, b, c, u)$ be a $solution$.\\\\
  Then $\lambda^2 \notdivides a + \eta^2  b$.
\end{lemma}
\begin{proof}
  \leanok
  \uses{lmm:lambda_ne_zero, lmm:toInteger_eval_cyclo}
  By contradiction using \Cref{lmm:toInteger_eval_cyclo}, we assume
  $\lambda^2 \divides a +\eta^2 b = a + b -b + \eta^2  b$.
  Since $\lambda^2 \divides a+b$, then $\lambda^2 \divides b (\eta^2 -1)
  = \lambda b (\eta + 1)$. Since $\lambda \notdivides b$, then
  $\lambda \divides \eta+1 = \lambda +2$, then $\lambda \divides 2$ which is absurd.
\end{proof}

\begin{lemma}
  \label{lmm:eta_add_one_inv}
  \lean{Solution.eta_add_one_inv}
  \leanok
  \uses{def:Solution}
  Let $K = \Q(\zeta_3)$ be the third cyclotomic field. \\
  Let $\cc{O}_K = \Z[\zeta_3]$ be the ring of integers of $K$. \\
  Let $\cc{O}^\times_K$ be the group of units of $\cc{O}_K$. \\
  Let $\zeta_3 \in K$ be any primitive third root of unity. \\
  Let $\eta \in \cc{O}_K$ be the element corresponding to $\zeta_3 \in K$. \\
  Let $S=(a, b, c, u)$ be a $solution$.\\\\
  Then $(\eta + 1)  (-\eta) = 1$.
\end{lemma}
\begin{proof}
  \leanok
  \uses{lmm:toInteger_eval_cyclo}
  Trivial calculation using \Cref{lmm:toInteger_eval_cyclo}.
\end{proof}

\begin{lemma}
  \label{lmm:associated_of_dvd_a_add_b_of_dvd_a_add_eta_mul_b}
  \lean{Solution.associated_of_dvd_a_add_b_of_dvd_a_add_eta_mul_b}
  \leanok
  \uses{def:Solution}
  Let $K = \Q(\zeta_3)$ be the third cyclotomic field. \\
  Let $\cc{O}_K = \Z[\zeta_3]$ be the ring of integers of $K$. \\
  Let $\cc{O}^\times_K$ be the group of units of $\cc{O}_K$. \\
  Let $\zeta_3 \in K$ be any primitive third root of unity. \\
  Let $\eta \in \cc{O}_K$ be the element corresponding to $\zeta_3 \in K$. \\
  Let $\lambda \in \cc{O}_K$ be such that $\lambda = \eta -1$. \\
  Let $S=(a, b, c, u)$ be a $solution$.\\
  Let $p \in \cc{O}_K$ be a prime such that $p \divides a+b$
  and $p \divides a+\eta  b$.\\\\
  Then $p$ is associated with $\lambda$.
\end{lemma}
\begin{proof}
  \leanok
  \uses{lmm:lambda_prime}
  We proceed by analysis each case:
  \begin{itemize}
      \item Case $p \divides \lambda$. It directly follows from \Cref{lmm:lambda_prime}.
      \item Case $p \notdivides \lambda$. \\
            By hypothesis, we have that $p \divides a+b$ and $p \divides a+\eta b$.
            Then $p \divides (a+\eta b) - (a+b) = b (\eta-1) = b \lambda$, which implies that
            $p \divides b$ and we proceed analogously to show that $p \divides a$.\\
            Therefore $p \divides \gcd(a,b)=1$ which is absurd.
  \end{itemize}
  Therefore, we can conclude that $p$ is associated with $\lambda$.
\end{proof}

\begin{lemma}
  \label{lmm:associated_of_dvd_a_add_b_of_dvd_a_add_eta_sq_mul_b}
  \lean{Solution.associated_of_dvd_a_add_b_of_dvd_a_add_eta_sq_mul_b}
  \leanok
  \uses{def:Solution}
  Let $K = \Q(\zeta_3)$ be the third cyclotomic field. \\
  Let $\cc{O}_K = \Z[\zeta_3]$ be the ring of integers of $K$. \\
  Let $\cc{O}^\times_K$ be the group of units of $\cc{O}_K$. \\
  Let $\zeta_3 \in K$ be any primitive third root of unity. \\
  Let $\eta \in \cc{O}_K$ be the element corresponding to $\zeta_3 \in K$. \\
  Let $\lambda \in \cc{O}_K$ be such that $\lambda = \eta -1$. \\
  Let $S=(a, b, c, u)$ be a $solution$.\\
  Let $p \in \cc{O}_K$ be a prime such that $p \divides a+b$
  and $p \divides a+\eta^2  b$.\\\\
  Then $p$ is associated with $\lambda$.
\end{lemma}
\begin{proof}
  \leanok
  \uses{lmm:lambda_prime, lmm:toInteger_cube_eq_one, lmm:eta_isUnit}
  We proceed by analysis each case:
  \begin{itemize}
      \item Case $p \divides \lambda$. It directly follows from \Cref{lmm:lambda_prime}.
      \item Case $p \notdivides \lambda$. \\
            By hypothesis, we have that $p \divides a+ b$ and $p \divides a+\eta^2 b$.
            By \Cref{lmm:toInteger_cube_eq_one} and \Cref{lmm:eta_isUnit}, we have that
            $$p \divides \eta ((a+\eta^2 b) - (a+ b)) = - (\eta^3 - \eta) b = \lambda b,$$
            which implies that $p \divides b$
            and we proceed analogously to show that $p \divides a$.\\
            Therefore $p \divides \gcd(a,b)=1$ which is absurd.
  \end{itemize}
  Therefore, we can conclude that $p$ is associated with $\lambda$.
\end{proof}

\begin{lemma}
  \label{lmm:associated_of_dvd_a_add_eta_mul_b_of_dvd_a_add_eta_sq_mul_b}
  \lean{Solution.associated_of_dvd_a_add_eta_mul_b_of_dvd_a_add_eta_sq_mul_b}
  \leanok
  \uses{def:Solution}
  Let $K = \Q(\zeta_3)$ be the third cyclotomic field. \\
  Let $\cc{O}_K = \Z[\zeta_3]$ be the ring of integers of $K$. \\
  Let $\cc{O}^\times_K$ be the group of units of $\cc{O}_K$. \\
  Let $\zeta_3 \in K$ be any primitive third root of unity. \\
  Let $\eta \in \cc{O}_K$ be the element corresponding to $\zeta_3 \in K$. \\
  Let $\lambda \in \cc{O}_K$ be such that $\lambda = \eta -1$. \\
  Let $S=(a, b, c, u)$ be a $solution$.\\
  Let $p \in \cc{O}_K$ be a prime such that $p \divides a+\eta b$
  and $p \divides a+\eta^2  b$.\\\\
  Then $p$ is associated with $\lambda$.
\end{lemma}
\begin{proof}
  \leanok
  \uses{lmm:lambda_prime, lmm:eta_isUnit}
  We proceed by analysis each case:
  \begin{itemize}
      \item Case $p \divides \lambda$. It directly follows from \Cref{lmm:lambda_prime}.
      \item Case $p \notdivides \lambda$. \\
            By hypothesis, we have that $p \divides a+\eta b$ and $p \divides a+\eta^2 b$.
            Then $p \divides (a+\eta^2 b) - (a+\eta b) = \eta b (\eta-1) = \eta b \lambda$,
            which, by \Cref{lmm:eta_isUnit}, implies that $p \divides b$
            and we proceed analogously to show that $p \divides a$.\\
            Therefore $p \divides \gcd(a,b)=1$ which is absurd.
  \end{itemize}
  Therefore, we can conclude that $p$ is associated with $\lambda$.
\end{proof}

\begin{definition}[$x,y,z,w$]
  \label{def:Solution_x_y_z_w}
  %\lean{}
  \leanok
  \uses{def:Solution}
  Let $K = \Q(\zeta_3)$ be the third cyclotomic field. \\
  Let $\cc{O}_K = \Z[\zeta_3]$ be the ring of integers of $K$. \\
  Let $\cc{O}^\times_K$ be the group of units of $\cc{O}_K$. \\
  Let $\zeta_3 \in K$ be any primitive third root of unity. \\
  Let $\eta \in \cc{O}_K$ be the element corresponding to $\zeta_3 \in K$. \\
  Let $\lambda \in \cc{O}_K$ be such that $\lambda = \eta -1$. \\
  Let $S=(a, b, c, u)$ be a $solution$.\\\\
  We define $x \in \cc{O}_K$ such that $a + b = \lambda^{3n-2}  x$.\\
  We define $y \in \cc{O}_K$ such that $a + \eta  b = \lambda  y$.\\
  We define $z \in \cc{O}_K$ such that $a + \eta^2  b = \lambda  z$.\\
  We define $w \in \cc{O}_K$ such that $c = \lambda^n  w$.
\end{definition}

\begin{lemma}
  \label{lmm:lambda_not_dvd_y}
  \lean{Solution.lambda_not_dvd_y}
  \leanok
  \uses{def:Solution}
  Let $K = \Q(\zeta_3)$ be the third cyclotomic field. \\
  Let $\cc{O}_K = \Z[\zeta_3]$ be the ring of integers of $K$. \\
  Let $\cc{O}^\times_K$ be the group of units of $\cc{O}_K$. \\
  Let $\zeta_3 \in K$ be any primitive third root of unity. \\
  Let $\eta \in \cc{O}_K$ be the element corresponding to $\zeta_3 \in K$. \\
  Let $\lambda \in \cc{O}_K$ be such that $\lambda = \eta -1$. \\
  Let $S$ be a $solution$.\\\\
  Then $\lambda \notdivides y$.
\end{lemma}
\begin{proof}
  \leanok
  \uses{lmm:lambda_sq_not_dvd_a_add_eta_mul_b}
  By contradiction we assume that $\lambda \divides y$, which implies that
  $\lambda^2 \divides \lambda y = a + \eta b$, that contradicts
  \Cref{lmm:lambda_sq_not_dvd_a_add_eta_mul_b} forcing us to conclude that
  $\lambda \notdivides y$.
\end{proof}

\begin{lemma}
  \label{lmm:lambda_not_dvd_z}
  \lean{Solution.lambda_not_dvd_z}
  \leanok
  \uses{def:Solution}
  Let $K = \Q(\zeta_3)$ be the third cyclotomic field. \\
  Let $\cc{O}_K = \Z[\zeta_3]$ be the ring of integers of $K$. \\
  Let $\cc{O}^\times_K$ be the group of units of $\cc{O}_K$. \\
  Let $\zeta_3 \in K$ be any primitive third root of unity. \\
  Let $\eta \in \cc{O}_K$ be the element corresponding to $\zeta_3 \in K$. \\
  Let $\lambda \in \cc{O}_K$ be such that $\lambda = \eta -1$. \\
  Let $S$ be a $solution$.\\\\
  Then $\lambda \notdivides z$.
\end{lemma}
\begin{proof}
  \leanok
  \uses{lmm:lambda_sq_not_dvd_a_add_eta_sq_mul_b}
  By contradiction we assume that $\lambda \divides z$, which implies that
  $\lambda^2 \divides \lambda z = a + \eta^2 b$, that contradicts
  \Cref{lmm:lambda_sq_not_dvd_a_add_eta_sq_mul_b} forcing us to conclude $\lambda \notdivides z$.
\end{proof}

\begin{lemma}
  \label{lmm:lambda_pow_dvd_a_add_b}
  \lean{Solution.lambda_pow_dvd_a_add_b}
  \leanok
  \uses{def:Solution}
  Let $K = \Q(\zeta_3)$ be the third cyclotomic field. \\
  Let $\cc{O}_K = \Z[\zeta_3]$ be the ring of integers of $K$. \\
  Let $\cc{O}^\times_K$ be the group of units of $\cc{O}_K$. \\
  Let $\zeta_3 \in K$ be any primitive third root of unity. \\
  Let $\eta \in \cc{O}_K$ be the element corresponding to $\zeta_3 \in K$. \\
  Let $\lambda \in \cc{O}_K$ be such that $\lambda = \eta -1$. \\
  Let $S=(a, b, c, u)$ be a $solution$ with multiplicity $n$.\\\\
  Then $\lambda^{3n -2} \divides a + b$.
\end{lemma}
\begin{proof}
  \leanok
  \uses{lmm:cube_add_cube_eq_mul, lmm:lambda_prime,
  lmm:lambda_not_dvd_z, lmm:lambda_not_dvd_y, lmm:Solution_two_le_multiplicity,
  lmm:lambda_ne_zero}
  By \Cref{def:Solution_Multiplicity} we have that $\lambda^n \divides c$.
  Since $u$ is a unit, then by \Cref{lmm:cube_add_cube_eq_mul} we have that
  $$\lambda^{3n} \divides u  c^3 = a^3 + b^3 = (a+b)(a + \eta b)(a + \eta^2 b)
  = (a+b)(\lambda y)(\lambda z).$$
  Then applying \Cref{lmm:lambda_not_dvd_y} and \Cref{lmm:lambda_not_dvd_z}, we can conclude
  that $\lambda^{3n-2} \divides a+b$.
\end{proof}

\begin{lemma}
  \label{lmm:lambda_not_dvd_w}
  \lean{Solution.lambda_not_dvd_w}
  \leanok
  \uses{def:Solution}
  Let $K = \Q(\zeta_3)$ be the third cyclotomic field. \\
  Let $\cc{O}_K = \Z[\zeta_3]$ be the ring of integers of $K$. \\
  Let $\cc{O}^\times_K$ be the group of units of $\cc{O}_K$. \\
  Let $\zeta_3 \in K$ be any primitive third root of unity. \\
  Let $\eta \in \cc{O}_K$ be the element corresponding to $\zeta_3 \in K$. \\
  Let $\lambda \in \cc{O}_K$ be such that $\lambda = \eta -1$. \\
  Let $S$ be a $solution$.\\\\
  Then $\lambda \notdivides w$.
\end{lemma}
\begin{proof}
  \leanok
  \uses{lmm:multiplicity_lambda_c_finite}
  By contradiction we assume that $\lambda \divides w$, which implies
  $\lambda^{n+1} \divides \lambda^n  w = c$ that contradicts \Cref{def:Solution_Multiplicity}
  forcing us to conclude that $\lambda \notdivides w$.
\end{proof}

\begin{lemma}
  \label{lmm:lambda_not_dvd_x}
  \lean{Solution.lambda_not_dvd_x}
  \leanok
  \uses{def:Solution}
  Let $K = \Q(\zeta_3)$ be the third cyclotomic field. \\
  Let $\cc{O}_K = \Z[\zeta_3]$ be the ring of integers of $K$. \\
  Let $\cc{O}^\times_K$ be the group of units of $\cc{O}_K$. \\
  Let $\zeta_3 \in K$ be any primitive third root of unity. \\
  Let $\eta \in \cc{O}_K$ be the element corresponding to $\zeta_3 \in K$. \\
  Let $\lambda \in \cc{O}_K$ be such that $\lambda = \eta -1$. \\
  Let $S$ be a $solution$.\\\\
  Then $\lambda \notdivides x$.
\end{lemma}
\begin{proof}
  \leanok
  \uses{lmm:lambda_dvd_a_add_eta_mul_b, lmm:lambda_dvd_a_add_eta_sq_mul_b,
  lmm:cube_add_cube_eq_mul, lmm:Solution_two_le_multiplicity, lmm:lambda_prime,
  lmm:lambda_not_dvd_w, lmm:lambda_ne_zero}
  By contradiction, if $\lambda \divides x$, then
  $\lambda^{3n-1} \divides \lambda^{3n-2}  x = a+b$. Using \Cref{lmm:lambda_dvd_a_add_eta_mul_b}
  and \Cref{lmm:lambda_dvd_a_add_eta_sq_mul_b}, we have that $\lambda^{3n+1} \divides
  (a+b)  (a + \eta  b)  (a + \eta^2 cdot b) = a^3+b^3
  = u c^3 = u \lambda^{3n} w^3$.
  Then $\lambda \divides w^3$ which implies that $\lambda \divides w$, that
  contradicts \Cref{lmm:lambda_not_dvd_w} forcing us to conclude $\lambda \notdivides x$.
\end{proof}

\begin{lemma}
  \label{lmm:coprime_x_y}
  \lean{Solution.coprime_x_y}
  \leanok
  \uses{def:Solution, def:Solution_x_y_z_w}
  Let $S$ be a $solution$ with multiplicity $n$.\\\\
  Then $\gcd(x,y) = 1$.
\end{lemma}
\begin{proof}
  \leanok
  \uses{lmm:lambda_not_dvd_y, lmm:associated_of_dvd_a_add_b_of_dvd_a_add_eta_mul_b,
  lmm:lambda_not_dvd_x}
  Since $y \neq 0$ by \Cref{lmm:lambda_not_dvd_y}, by the properties of PIDs it suffices to prove that
  $\forall p \in \cc{O}_K$ if $p$ is prime and $p \divides x$, then $p \notdivides y$.
  Let $p \in \cc{O}_K$ be prime and suppose by contradiction that $p \divides x$ and $p \divides y$
  which implies that $p \divides \lambda^{3n-2} x = a+b$ and $p \divides \lambda y = a + \eta b$.
  Then by \Cref{lmm:associated_of_dvd_a_add_b_of_dvd_a_add_eta_mul_b}
  we have that $p$ is associated with $\lambda$, which implies that $\lambda \divides x$
  that contradicts \Cref{lmm:lambda_not_dvd_x} forcing us to conclude that $p \notdivides y$, which,
  as stated above, implies that $\gcd(x,y)=1$.
\end{proof}

\begin{lemma}
  \label{lmm:coprime_x_z}
  \lean{Solution.coprime_x_z}
  \leanok
  \uses{def:Solution, def:Solution_x_y_z_w}
  Let $S$ be a $solution$.\\\\
  Then $\gcd(x,z) = 1$.
\end{lemma}
\begin{proof}
  \leanok
  \uses{lmm:lambda_not_dvd_z, lmm:associated_of_dvd_a_add_b_of_dvd_a_add_eta_sq_mul_b,
  lmm:lambda_not_dvd_x}
  Since $z \neq 0$ by \Cref{lmm:lambda_not_dvd_z}, by the properties of PIDs it suffices to prove that
  $\forall p \in \cc{O}_K$ if $p$ is prime and $p \divides x$, then $p \notdivides z$.
  Let $p \in \cc{O}_K$ be prime and suppose by contradiction that $p \divides x$ and $p \divides z$
  which implies that $p \divides \lambda^{3n-2} x = a+b$ and $p \divides \lambda z = a + \eta^2 b$.
  Then by \Cref{lmm:associated_of_dvd_a_add_b_of_dvd_a_add_eta_sq_mul_b}
  we have that $p$ is associated with $\lambda$, which implies that $\lambda \divides x$
  that contradicts \Cref{lmm:lambda_not_dvd_x} forcing us to conclude that $p \notdivides z$, which,
  as stated above, implies that $\gcd(x,z)=1$.
\end{proof}

\begin{lemma}
  \label{lmm:coprime_y_z}
  \lean{Solution.coprime_y_z}
  \leanok
  \uses{def:Solution, def:Solution_x_y_z_w}
  Let $S$ be a $solution$.\\\\
  Then $\gcd(y, z) = 1$.
\end{lemma}
\begin{proof}
  \leanok
  \uses{lmm:lambda_not_dvd_z, lmm:associated_of_dvd_a_add_eta_mul_b_of_dvd_a_add_eta_sq_mul_b,
  lmm:lambda_not_dvd_y}
  Since $z \neq 0$ by \Cref{lmm:lambda_not_dvd_z}, by the properties of PIDs it suffices to prove that
  $\forall p \in \cc{O}_K$ if $p$ is prime and $p \divides y$, then $p \notdivides z$.
  Let $p \in \cc{O}_K$ be prime and suppose by contradiction that $p \divides y$ and $p \divides z$
  which implies that $p \divides \lambda y = a+\eta b$ and $p \divides \lambda z = a + \eta^2 b$.
  Then by \Cref{lmm:associated_of_dvd_a_add_eta_mul_b_of_dvd_a_add_eta_sq_mul_b}
  we have that $p$ is associated with $\lambda$, which implies that $\lambda \divides y$
  that contradicts \Cref{lmm:lambda_not_dvd_y} forcing us to conclude that $p \notdivides z$, which,
  as stated above, implies that $\gcd(y,z)=1$.
\end{proof}

\begin{lemma}
  \label{lmm:mult_minus_two_plus_one_plus_one}
  \lean{Solution.mult_minus_two_plus_one_plus_one}
  \leanok
  \uses{def:Solution}
  Let $S$ be a $solution$ with multiplicity $n$.\\\\
  Then $3n - 2 + 1 + 1 = 3n$.
\end{lemma}
\begin{proof}
  \leanok
  \uses{lmm:Solution_two_le_multiplicity}
  It directly follows from \Cref{lmm:Solution_two_le_multiplicity}
  and calculations using ring properties.
\end{proof}

\begin{lemma}
  \label{lmm:x_mul_y_mul_z_eq_u_w_pow_three}
  \lean{Solution.x_mul_y_mul_z_eq_u_w_pow_three}
  \leanok
  \uses{def:Solution, def:Solution_x_y_z_w}
  Let $S=(a,b,c,u)$ be a $solution$.\\\\
  Then $x y z = u w^3$.
\end{lemma}
\begin{proof}
  \leanok
  \uses{lmm:Solution_two_le_multiplicity, lmm:lambda_ne_zero, lmm:cube_add_cube_eq_mul,
  def:Solution_x_y_z_w}
  It directly follows from \Cref{def:Solution_x_y_z_w}, \Cref{lmm:cube_add_cube_eq_mul},
  \Cref{lmm:lambda_ne_zero}, \Cref{lmm:Solution_two_le_multiplicity} and
  calculations using ring properties.
\end{proof}

\begin{lemma}
  \label{lmm:x_eq_unit_mul_cube}
  \lean{Solution.x_eq_unit_mul_cube}
  \leanok
  \uses{def:Solution}
  Let $S$ be a $solution$.\\\\
  Then $\exists u_1 \in \cc{O}^\times_K$ and $\exists X \in \cc{O}_K$
  such that $x = u_1 X^3$.
\end{lemma}
\begin{proof}
  \leanok
  \uses{lmm:x_mul_y_mul_z_eq_u_w_pow_three, lmm:coprime_x_y, lmm:coprime_x_z}
  By the properties of PIDs, it suffices to prove that there exists a $k\in \cc{O}_K$ such that
  $xk$ is a cube and $\gcd(x,k)=1$.
  Let $k = yzu^{-1}$, then $xk = x y z u^{-1} = w^3$ by \Cref{lmm:x_mul_y_mul_z_eq_u_w_pow_three}.
  Moreover, since $\gcd(x,y)=1$ by \Cref{lmm:coprime_x_y} and $\gcd(x,z)=1$ by \Cref{lmm:coprime_x_z},
  then $\gcd(x,yz)=1$, which implies that $\gcd(x,k)=1$.
\end{proof}

\begin{lemma}
  \label{lmm:y_eq_unit_mul_cube}
  \lean{Solution.y_eq_unit_mul_cube}
  \leanok
  \uses{def:Solution}
  Let $S$ be a $solution$.\\\\
  Then $\exists u_2 \in \cc{O}^\times_K$ and $\exists Y \in \cc{O}_K$
  such that $y = u_2 Y^3$.
\end{lemma}
\begin{proof}
  \leanok
  \uses{lmm:x_mul_y_mul_z_eq_u_w_pow_three, lmm:coprime_x_y, lmm:coprime_y_z}
  By the properties of PIDs, it suffices to prove that there exists a $k\in \cc{O}_K$ such that
  $yk$ is a cube and $\gcd(y,k)=1$.
  Let $k = xzu^{-1}$, then $yk = y x z u^{-1} = w^3$ by \Cref{lmm:x_mul_y_mul_z_eq_u_w_pow_three}.
  Moreover, since $\gcd(x,y)=1$ by \Cref{lmm:coprime_x_y} and $\gcd(y,z)=1$ by \Cref{lmm:coprime_y_z},
  then $\gcd(y,xz)=1$, which implies that $\gcd(y,k)=1$.
\end{proof}

\begin{lemma}
  \label{lmm:z_eq_unit_mul_cube}
  \lean{Solution.z_eq_unit_mul_cube}
  \leanok
  \uses{def:Solution}
  Let $S$ be a $solution$.\\\\
  Then $\exists u_3 \in \cc{O}^\times_K$ and $\exists Z \in \cc{O}_K$
  such that $z = u_3  Z^3$.
\end{lemma}
\begin{proof}
  \leanok
  \uses{lmm:x_mul_y_mul_z_eq_u_w_pow_three, lmm:coprime_x_z, lmm:coprime_y_z}
  By the properties of PIDs, it suffices to prove that there exists a $k\in \cc{O}_K$ such that
  $zk$ is a cube and $\gcd(z,k)=1$.
  Let $k = xyu^{-1}$, then $zk = z x y u^{-1} = w^3$ by \Cref{lmm:x_mul_y_mul_z_eq_u_w_pow_three}.
  Moreover, since $\gcd(x,z)=1$ by \Cref{lmm:coprime_x_z} and $\gcd(y,z)=1$ by \Cref{lmm:coprime_y_z},
  then $\gcd(z,xy)=1$, which implies that $\gcd(z,k)=1$.
\end{proof}

\begin{definition}[$u_1,u_2,u_3,u_4,u_5,X,Y,Z$]
  \label{def:Solution_u1_u2_u3_u4_u5_X_Y_Z}
  %\lean{}
  \leanok
  \uses{def:Solution, lmm:x_eq_unit_mul_cube,
  lmm:y_eq_unit_mul_cube, lmm:z_eq_unit_mul_cube}
  Let $S$ be a $solution$.\\\\
  We define $u_1 \in \cc{O}^\times_K$ and $X \in \cc{O}_K$
  such that $x = u_1 X^3$.\\
  We define $u_2 \in \cc{O}^\times_K$ and $Y \in \cc{O}_K$
  such that $y = u_2 Y^3$.\\
  We define $u_3 \in \cc{O}^\times_K$ and $Z \in \cc{O}_K$
  such that $z = u_3 Z^3$.\\
  We define $u_4 = \eta u_3 u_2^{-1}$.\\
  We define $u_5 = -\eta^2 u_1 u_2^{-1}$.\\
\end{definition}

\begin{lemma}
  \label{lmm:X_ne_zero}
  \lean{Solution.X_ne_zero}
  \leanok
  \uses{def:Solution, def:Solution_u1_u2_u3_u4_u5_X_Y_Z}
  Let $S$ be a $solution$.\\\\
  Then $X \neq 0$.
\end{lemma}
\begin{proof}
  \leanok
  \uses{def:Solution_u1_u2_u3_u4_u5_X_Y_Z, lmm:lambda_not_dvd_x}
  By contradiction we assume that $X = 0$, then $x = 0$ by \Cref{def:Solution_u1_u2_u3_u4_u5_X_Y_Z}.
  Therefore $\lambda$ trivially divides $x$ (as any number divides zero) which contradicts
  \Cref{lmm:lambda_not_dvd_x} forcing us to conclude that $X \neq 0$.
\end{proof}

\begin{lemma}
  \label{lmm:lambda_not_dvd_X}
  \lean{Solution.lambda_not_dvd_X}
  \leanok
  \uses{def:Solution, def:Solution_u1_u2_u3_u4_u5_X_Y_Z}
  Let $K = \Q(\zeta_3)$ be the third cyclotomic field. \\
  Let $\cc{O}_K = \Z[\zeta_3]$ be the ring of integers of $K$. \\
  Let $\cc{O}^\times_K$ be the group of units of $\cc{O}_K$. \\
  Let $\zeta_3 \in K$ be any primitive third root of unity. \\
  Let $\eta \in \cc{O}_K$ be the element corresponding to $\zeta_3 \in K$. \\
  Let $\lambda \in \cc{O}_K$ be such that $\lambda = \eta -1$. \\
  Let $S$ be a $solution$.\\\\
  Then $\lambda \notdivides X$.
\end{lemma}
\begin{proof}
  \leanok
  \uses{def:Solution_u1_u2_u3_u4_u5_X_Y_Z,, lmm:lambda_not_dvd_x}
  By contradiction we assume that $\lambda \divides X$, then, by the properties of divisibility,
  $\lambda \divides u_1 X^3$, which implies, by \Cref{def:Solution_u1_u2_u3_u4_u5_X_Y_Z},
  that $\lambda \divides x$.
  However, this contradicts \Cref{lmm:lambda_not_dvd_x}
  forcing us to conclude that $\lambda \notdivides X$.
\end{proof}

\begin{lemma}
  \label{lmm:lambda_not_dvd_Y}
  \lean{Solution.lambda_not_dvd_Y}
  \leanok
  \uses{def:Solution, def:Solution_u1_u2_u3_u4_u5_X_Y_Z}
  Let $K = \Q(\zeta_3)$ be the third cyclotomic field. \\
  Let $\cc{O}_K = \Z[\zeta_3]$ be the ring of integers of $K$. \\
  Let $\cc{O}^\times_K$ be the group of units of $\cc{O}_K$. \\
  Let $\zeta_3 \in K$ be any primitive third root of unity. \\
  Let $\eta \in \cc{O}_K$ be the element corresponding to $\zeta_3 \in K$. \\
  Let $\lambda \in \cc{O}_K$ be such that $\lambda = \eta -1$. \\
  Let $S$ be a $solution$.\\\\
  Then $\lambda \notdivides Y$.
\end{lemma}
\begin{proof}
  \leanok
  \uses{lmm:lambda_not_dvd_y}
  By contradiction we assume that $\lambda \divides Y$, then, by the properties of divisibility,
  $\lambda \divides u_2 Y^3$, which implies, by \Cref{def:Solution_u1_u2_u3_u4_u5_X_Y_Z},
  that $\lambda \divides y$.
  However, this contradicts \Cref{lmm:lambda_not_dvd_y}
  forcing us to conclude that $\lambda \notdivides Y$.
\end{proof}

\begin{lemma}
  \label{lmm:lambda_not_dvd_Z}
  \lean{Solution.lambda_not_dvd_Z}
  \leanok
  \uses{def:Solution, def:Solution_u1_u2_u3_u4_u5_X_Y_Z}
  Let $K = \Q(\zeta_3)$ be the third cyclotomic field. \\
  Let $\cc{O}_K = \Z[\zeta_3]$ be the ring of integers of $K$. \\
  Let $\cc{O}^\times_K$ be the group of units of $\cc{O}_K$. \\
  Let $\zeta_3 \in K$ be any primitive third root of unity. \\
  Let $\eta \in \cc{O}_K$ be the element corresponding to $\zeta_3 \in K$. \\
  Let $\lambda \in \cc{O}_K$ be such that $\lambda = \eta -1$. \\
  Let $S$ be a $solution$.\\\\
  Then $\lambda \notdivides Z$.
\end{lemma}
\begin{proof}
  \leanok
  \uses{lmm:lambda_not_dvd_z}
  By contradiction we assume that $\lambda \divides Z$, then, by the properties of divisibility,
  $\lambda \divides u_3 Z^3$, which implies, by \Cref{def:Solution_u1_u2_u3_u4_u5_X_Y_Z},
  that $\lambda \divides z$.
  However, this contradicts \Cref{lmm:lambda_not_dvd_z}
  forcing us to conclude that $\lambda \notdivides Z$.
\end{proof}

\begin{lemma}
  \label{lmm:coprime_Y_Z}
  \lean{Solution.coprime_Y_Z}
  \leanok
  \uses{def:Solution, def:Solution_u1_u2_u3_u4_u5_X_Y_Z}
  Let $S$ be a $solution$.\\\\
  Then $\gcd(Y, Z) = 1$.
\end{lemma}
\begin{proof}
  \leanok
  \uses{lmm:lambda_not_dvd_Z, lmm:coprime_y_z}
  Since $Z \neq 0$ by \Cref{lmm:lambda_not_dvd_Z}, by the properties of PIDs it suffices to prove that
  $\forall p \in \cc{O}_K$ if $p$ is prime and $p \divides Y$, then $p \notdivides Z$.
  Let $p \in \cc{O}_K$ be prime and suppose by contradiction that $p \divides Y$ and $p \divides Z$
  which implies that $p \divides u_2 Y^3 = y$ and $p \divides \lambda u_3 Z^3 = z$.
  But this contradicts \Cref{lmm:coprime_y_z} forcing us to conclude that $p \notdivides Z$, which,
  as stated above, implies that $\gcd(Y,Z)=1$.
\end{proof}

\begin{lemma}
  \label{lmm:formula1}
  \lean{Solution.formula1}
  \leanok
  \uses{def:Solution, def:Solution_u1_u2_u3_u4_u5_X_Y_Z}
  Let $K = \Q(\zeta_3)$ be the third cyclotomic field. \\
  Let $\cc{O}_K = \Z[\zeta_3]$ be the ring of integers of $K$. \\
  Let $\cc{O}^\times_K$ be the group of units of $\cc{O}_K$. \\
  Let $\zeta_3 \in K$ be any primitive third root of unity. \\
  Let $\eta \in \cc{O}_K$ be the element corresponding to $\zeta_3 \in K$. \\
  Let $\lambda \in \cc{O}_K$ be such that $\lambda = \eta -1$. \\
  Let $S$ be a $solution$ with multiplicity $n$.\\\\
  Then $u_1 X^3 \lambda^{3n-2}+u_2 \eta Y^3 \lambda +
  u_3 \eta^2 Z^3 \lambda = 0$.
\end{lemma}
\begin{proof}
  \leanok
  \uses{def:Solution_u1_u2_u3_u4_u5_X_Y_Z, def:Solution_x_y_z_w,
  lmm:toInteger_cube_eq_one, lmm:toInteger_eval_cyclo}
  Applying \Cref{def:Solution_u1_u2_u3_u4_u5_X_Y_Z}, \Cref{def:Solution_x_y_z_w},
  \Cref{lmm:toInteger_cube_eq_one} and \Cref{lmm:toInteger_eval_cyclo}, we have
  \begin{align*}
      u_1 X^3 \lambda^{3n-2}+u_2 \eta Y^3 \lambda + u_3 \eta^2 Z^3 \lambda
      &= x \lambda^{3n-2} + \eta y \lambda + \eta^2 z \lambda \\
      &= (a+b) + \eta (a+\eta b) + \eta^2 (a+\eta^2 b) \\
      &= a (1 + \eta + \eta^2) + b (1 + \eta^4 + \eta^2) \\
      &= (a+b)(1+\eta+\eta^2)\\
      &= (a+b)0 = 0
  \end{align*}
\end{proof}

\begin{lemma}
  \label{lmm:u₄_isUnit}
  \lean{Solution.u₄'_isUnit}
  \leanok
  \uses{def:Solution, def:Solution_u1_u2_u3_u4_u5_X_Y_Z}
  Let $S$ be a $solution$.\\\\
  Then $u_4$ is a unit.
\end{lemma}
\begin{proof}
  \leanok
  \uses{def:Solution_u1_u2_u3_u4_u5_X_Y_Z, lmm:eta_isUnit}
  By \Cref{def:Solution_u1_u2_u3_u4_u5_X_Y_Z} $u_4 = \eta u_3 u_2^{-1}$,
  which is a product of units by \Cref{lmm:eta_isUnit}.
  Since the product of units is a unit (multiplicative closure),
  it follows that $u_4$ must also be a unit.
\end{proof}

\begin{lemma}
  \label{lmm:u₅_isUnit}
  \lean{Solution.u₅'_isUnit}
  \leanok
  \uses{def:Solution, def:Solution_u1_u2_u3_u4_u5_X_Y_Z}
  Let $S$ be a $solution$.\\\\
  Then $u_5$ is a unit.
\end{lemma}
\begin{proof}
  \leanok
  \uses{lmm:toInteger_cube_eq_one}
  By \Cref{def:Solution_u1_u2_u3_u4_u5_X_Y_Z} $u_5 = -\eta^2 u_1 u_2^{-1}$,
  which is a product of units since $\eta^3 = 1$ by \Cref{lmm:toInteger_cube_eq_one} and
  $-\eta (-\eta^2) = \eta^3$.
  Since the product of units is a unit (multiplicative closure),
  it follows that $u_5$ must also be a unit.
\end{proof}

\begin{lemma}
  \label{lmm:formula2}
  \lean{Solution.formula2}
  \leanok
  \uses{def:Solution, def:Solution_u1_u2_u3_u4_u5_X_Y_Z}
  Let $K = \Q(\zeta_3)$ be the third cyclotomic field. \\
  Let $\cc{O}_K = \Z[\zeta_3]$ be the ring of integers of $K$. \\
  Let $\cc{O}^\times_K$ be the group of units of $\cc{O}_K$. \\
  Let $\zeta_3 \in K$ be any primitive third root of unity. \\
  Let $\eta \in \cc{O}_K$ be the element corresponding to $\zeta_3 \in K$. \\
  Let $\lambda \in \cc{O}_K$ be such that $\lambda = \eta -1$. \\
  Let $S$ be a $solution$ with multiplicity $n$.\\\\
  Then $Y^3 + u_4 Z^3 = u_5 (\lambda^(n-1) X)^3$.
\end{lemma}
\begin{proof}
  \leanok
  \uses{lmm:eta_isUnit, lmm:lambda_ne_zero, lmm:toInteger_cube_eq_one,
  lmm:Solution_two_le_multiplicity, lmm:formula1}
  Using \Cref{lmm:eta_isUnit}, \Cref{lmm:lambda_ne_zero}, it suffices to show that
  $$\lambda \eta u_2 (Y^3 + u_4 Z^3) = \lambda \eta u_2 u_5 (\lambda^(n-1) X)^3$$
  which can be proved by simple calculations involving \Cref{lmm:toInteger_cube_eq_one},
  \Cref{lmm:Solution_two_le_multiplicity} and \Cref{lmm:formula1}.
\end{proof}

\begin{lemma}
  \label{lmm:lambda_sq_div_lambda_fourth}
  \lean{Solution.lambda_sq_div_lambda_fourth}
  \leanok
  \uses{def:Solution}
  Let $K = \Q(\zeta_3)$ be the third cyclotomic field. \\
  Let $\cc{O}_K = \Z[\zeta_3]$ be the ring of integers of $K$. \\
  Let $\cc{O}^\times_K$ be the group of units of $\cc{O}_K$. \\
  Let $\zeta_3 \in K$ be any primitive third root of unity. \\
  Let $\eta \in \cc{O}_K$ be the element corresponding to $\zeta_3 \in K$. \\
  Let $\lambda \in \cc{O}_K$ be such that $\lambda = \eta -1$. \\
  Let $S$ be a $solution$.\\\\
  Then $\lambda^2 \divides \lambda^4$.
\end{lemma}
\begin{proof}
  \leanok
  Straightforward application of the definition of divisibility.
\end{proof}

\begin{lemma}
  \label{lmm:lambda_sq_div_new_X_cubed}
  \lean{Solution.lambda_sq_div_new_X_cubed}
  \leanok
  \uses{def:Solution, def:Solution_u1_u2_u3_u4_u5_X_Y_Z}
  Let $K = \Q(\zeta_3)$ be the third cyclotomic field. \\
  Let $\cc{O}_K = \Z[\zeta_3]$ be the ring of integers of $K$. \\
  Let $\cc{O}^\times_K$ be the group of units of $\cc{O}_K$. \\
  Let $\zeta_3 \in K$ be any primitive third root of unity. \\
  Let $\eta \in \cc{O}_K$ be the element corresponding to $\zeta_3 \in K$. \\
  Let $\lambda \in \cc{O}_K$ be such that $\lambda = \eta -1$. \\
  Let $S$ be a $solution$ with multiplicity $n$.\\\\
  Then $\lambda^2 \divides u_5 (\lambda^{n - 1} X)^3$.
\end{lemma}
\begin{proof}
  \leanok
  \uses{lmm:Solution_two_le_multiplicity}
  Using \Cref{lmm:Solution_two_le_multiplicity}, we have that $\lambda^2 \divides
  \lambda^2 u_5 \lambda^{3n-5} X^3 = u_5 (\lambda^{n - 1} X)^3$.
\end{proof}

\begin{lemma}
  \label{lmm:by_kummer}
  \lean{Solution.by_kummer}
  \leanok
  \uses{def:Solution, def:Solution_u1_u2_u3_u4_u5_X_Y_Z}
  Let $S$ be a $solution$.\\\\
  Then $u_4 \in \set{-1,1} \subset \cc{O}_K$.
\end{lemma}
\begin{proof}
  \leanok
  \uses{lmm:lambda_sq_div_lambda_fourth, lmm:lambda_sq_div_new_X_cubed,
  lmm:eq_one_or_neg_one_of_unit_of_congruent,
  lmm:lambda_pow_four_dvd_cube_sub_one_or_add_one_of_lambda_not_dvd,
  lmm:lambda_not_dvd_Z, lmm:lambda_not_dvd_Y, lmm:formula2}
  Let $n \in \N$ be the multiplicity of the solution $S$.\\
  By \Cref{lmm:eq_one_or_neg_one_of_unit_of_congruent}, it suffices to prove that
  $$\exists m \in \Z \text{ such that } \lambda^2 \divides u_4 - m.$$
  By \Cref{lmm:lambda_pow_four_dvd_cube_sub_one_or_add_one_of_lambda_not_dvd}
  and \Cref{lmm:lambda_not_dvd_Y}, we have that
  $$(\lambda^4 \divides Y^3 - 1) \lor (\lambda^4 \divides Y^3 + 1).$$
  By \Cref{lmm:lambda_pow_four_dvd_cube_sub_one_or_add_one_of_lambda_not_dvd}
  and \Cref{lmm:lambda_not_dvd_Z}, we have that
  $$(\lambda^4 \divides Z^3 - 1) \lor (\lambda^4 \divides Z^3 + 1).$$
  We proceed by analysing each case:
  \begin{itemize}
      \item Case $(\lambda^4 \divides Y^3 - 1) \land (\lambda^4 \divides Z^3 - 1)$. \\
            Let $m=-1$ and consider the fact that
            $$u_4 - m = Y^3 + u_4 Z^3 - (Y^3 - 1) - u_4 (Z^3 - 1).$$
            By \Cref{lmm:formula2}, we have that
            $$u_4 - m = u_5 (λ^{n-1} X)^3 - (Y^3 - 1) - u_4 (Z^3 - 1).$$
            Since, by \Cref{lmm:lambda_sq_div_new_X_cubed}, we know that
            $$\lambda^2 \divides u_5 (λ^{n-1} X)^3$$
            and, by \Cref{lmm:lambda_sq_div_lambda_fourth} and by assumption, we have that
            $$\lambda^2 \divides Y^3 - 1 \land \lambda^2 \divides Z^3 - 1,$$
            Then, we can conclude that
            $$\lambda^2 \divides u_4 - m.$$
      \item Case $(\lambda^4 \divides Y^3 - 1) \land (\lambda^4 \divides Z^3 + 1)$. \\
            Let $m=1$ and proceed similarly to the first case.
      \item Case $(\lambda^4 \divides Y^3 + 1) \land (\lambda^4 \divides Z^3 - 1)$. \\
            Let $m=1$ and proceed similarly to the first case.
      \item Case $(\lambda^4 \divides Y^3 + 1) \land (\lambda^4 \divides Z^3 + 1)$. \\
            Let $m=-1$ and proceed similarly to the first case.
  \end{itemize}
\end{proof}

\begin{lemma}
  \label{lmm:final}
  \lean{Solution.final}
  \leanok
  \uses{def:Solution, def:Solution_u1_u2_u3_u4_u5_X_Y_Z}
  Let $K = \Q(\zeta_3)$ be the third cyclotomic field. \\
  Let $\cc{O}_K = \Z[\zeta_3]$ be the ring of integers of $K$. \\
  Let $\cc{O}^\times_K$ be the group of units of $\cc{O}_K$. \\
  Let $\zeta_3 \in K$ be any primitive third root of unity. \\
  Let $\eta \in \cc{O}_K$ be the element corresponding to $\zeta_3 \in K$. \\
  Let $\lambda \in \cc{O}_K$ be such that $\lambda = \eta -1$. \\
  Let $S$ be a $solution$ with multiplicity $n$.\\\\
  Then $Y^3 + (u_4 Z)^3 = u_5 (\lambda^{n-1} X)^3$.
\end{lemma}
\begin{proof}
  \leanok
  \uses{lmm:formula2, lmm:by_kummer}
  By \Cref{lmm:by_kummer}, we have that $u_4 \in \set{-1,1}$, which implies that $u_4^2 = 1$.\\
  Therefore, by \Cref{lmm:formula2}, we can conclude that
  $$Y^3 + (u_4 Z)^3 = u_5 (\lambda^{n-1} X)^3.$$
\end{proof}

\begin{definition}[Final Solution']
  \label{def:Solution1_final}
  \lean{Solution'_final}
  \leanok
  \uses{def:Solution1, def:Solution_u1_u2_u3_u4_u5_X_Y_Z, lmm:Solution_two_le_multiplicity,
  lmm:final, lmm:coprime_Y_Z, lmm:lambda_not_dvd_Y, lmm:lambda_not_dvd_Z, lmm:lambda_ne_zero,
  lmm:X_ne_zero}
  Let $K = \Q(\zeta_3)$ be the third cyclotomic field. \\
  Let $\cc{O}_K = \Z[\zeta_3]$ be the ring of integers of $K$. \\
  Let $\cc{O}^\times_K$ be the group of units of $\cc{O}_K$. \\
  Let $\zeta_3 \in K$ be any primitive third root of unity. \\
  Let $\eta \in \cc{O}_K$ be the element corresponding to $\zeta_3 \in K$. \\
  Let $\lambda \in \cc{O}_K$ be such that $\lambda = \eta -1$. \\
  Let $S = (a,b,c,u)$ be a $solution$ with multiplicity $n$.\\
  Let $S_f' = (Y,u_4 Z, \lambda^{n-1} X, u_5)$.\\\\
  Then $S_f'$ is a $solution'$.
\end{definition}

\begin{lemma}
  \label{lmm:Solution1_final_multiplicity}
  \lean{Solution'_final_multiplicity}
  \leanok
  \uses{def:Solution, def:Solution1_final}
  Let $S$ be a $solution$ with multiplicity $n$.\\\\
  Then $S_f'$ has multiplicity $n-1$.
\end{lemma}
\begin{proof}
  \leanok
  \uses{lmm:lambda_not_dvd_X,
  lmm:lambda_ne_zero}
  Let $K = \Q(\zeta_3)$ be the third cyclotomic field. \\
  Let $\cc{O}_K = \Z[\zeta_3]$ be the ring of integers of $K$. \\
  Let $\cc{O}^\times_K$ be the group of units of $\cc{O}_K$. \\
  Let $\zeta_3 \in K$ be any primitive third root of unity. \\
  Let $\eta \in \cc{O}_K$ be the element corresponding to $\zeta_3 \in K$. \\
  Let $\lambda \in \cc{O}_K$ be such that $\lambda = \eta -1$. \\
  Let $(a',b',c',u') = S_f'$ be the final $solution'$, then
  $\lambda^{n-1} \divides \lambda^{n-1} X = c'$.
  By contradiction we assume that $\lambda^n \divides c'$ which implies that $\lambda \divides X$,
  that contradicts \Cref{lmm:lambda_not_dvd_X} forcing us to conclude
  that $\lambda^{n} \notdivides c'$. Then $S_f'$ has multiplicity $n-1$.
\end{proof}

\begin{lemma}
  \label{lmm:Solution1_final_multiplicity_lt}
  \lean{Solution'_final_multiplicity_lt}
  \leanok
  \uses{def:Solution, def:Solution1_final}
  Let $S$ be a $solution$ with multiplicity $n$.\\\\
  Then $S_f'$ has multiplicity $m<n$.
\end{lemma}
\begin{proof}
  \leanok
  \uses{lmm:Solution1_final_multiplicity, lmm:Solution_two_le_multiplicity}
  It directly follows from \Cref{lmm:Solution1_final_multiplicity} since $m = n-1 < n$.
\end{proof}

\begin{theorem}
  \label{lmm:exists_Solution_multiplicity_lt}
  \lean{Solution.exists_Solution_multiplicity_lt}
  \leanok
  \uses{def:Solution}
  Let $S$ be a $solution$ with multiplicity $n$.\\\\
  Then there is a $solution$ with multiplicity $m<n$.
\end{theorem}
\begin{proof}
  \leanok
  \uses{lmm:exists_Solution_of_Solution1, lmm:Solution1_final_multiplicity_lt}
  It directly follows from \Cref{lmm:Solution1_final_multiplicity} and
  \Cref{lmm:Solution1_final_multiplicity_lt}.
\end{proof}

\begin{theorem}[Generalised Fermat's Last Theorem for Exponent $3$]
  \label{thm:fermatLastTheoremForThreeGen}
  \lean{fermatLastTheoremForThreeGen}
  \leanok
  Let $K = \Q(\zeta_3)$ be the third cyclotomic field. \\
  Let $\cc{O}_K = \Z[\zeta_3]$ be the ring of integers of $K$. \\
  Let $\cc{O}^\times_K$ be the group of units of $\cc{O}_K$. \\
  Let $\zeta_3 \in K$ be any primitive third root of unity. \\
  Let $\eta \in \cc{O}_K$ be the element corresponding to $\zeta_3 \in K$. \\
  Let $\lambda \in \cc{O}_K$ be such that $\lambda = \eta -1$. \\
  Let $a, b, c \in \cc{O}_K$ and $u \in \cc{O}^\times_K$ such that $c \neq 0$ and $\gcd(a,b)=1$.\\
  Let $\lambda \notdivides a$, $\lambda \notdivides b$ and $\lambda \divides c$. \\\\
  Then $a^3 + b^3 \neq u c^3$.
\end{theorem}
\begin{proof}
  \leanok
  \uses{lmm:exists_Solution_of_Solution1,
  lmm:exists_minimal,
  lmm:exists_Solution_multiplicity_lt}
  By contradiction we assume that there are $a, b, c \in \cc{O}_K$ and $u \in \cc{O}^\times_K$
  such that $c \neq 0$, $\gcd(a,b)=1$, $\lambda \notdivides a$, $\lambda \notdivides b$,
  $\lambda \divides c$ and $a^3 + b^3 = u c^3$.
  Then $S'=(a,b,c,u)$ is a $solution'$, which implies that there is a $solution$ $S$ by
  \Cref{lmm:exists_Solution_of_Solution1}.
  Then, by \Cref{lmm:exists_minimal}, there is a minimal solution $S_0$ with multiplicity $n$.
  Hence, there is a $solution'$ $S_1'$ with multiplicity $m<n$ by \Cref{lmm:exists_Solution_multiplicity_lt},
  which implies that there is a $solution$ $S_1$  with multiplicity $m$ by \Cref{lmm:exists_Solution_of_Solution1}.
  However, this contradicts the minimality of $S_0$
  forcing us to conclude that $a^3 + b^3 \neq u c^3$.
\end{proof}

\begin{lemma}
  \label{lmm:FermatLastTheoremForThree_of_FermatLastTheoremThreeGen}
  \lean{FermatLastTheoremForThree_of_FermatLastTheoremThreeGen}
  \leanok
  To prove \Cref{thm:fermatLastTheoremThree},
  it suffices to prove \Cref{thm:fermatLastTheoremForThreeGen}. \\
  Equivalently, \Cref{thm:fermatLastTheoremForThreeGen} implies
  \Cref{thm:fermatLastTheoremThree}.
\end{lemma}
\begin{proof}
  \leanok
  \uses{
  thm:fermatLastTheoremThree_of_three_dvd_only_c,
  lmm:norm_lambda_prime,
  lmm:norm_lambda,
  lmm:lambda_dvd_three}
  Assume that $\forall a, b, c \in \cc{O}_K,\, \forall u \in \cc{O}^\times_K$ such that $c \neq 0$,
  $\gcd(a,b)=1$, $\lambda \notdivides a$, $\lambda \notdivides b$ and $\lambda \divides c$,
  it holds that $a^3 + b^3 \neq u c^3$.
  Let $a, b, c \in \Z$ such that $a\neq 0$, $b\neq 0$ and $c\neq 0$.
  By \Cref{thm:fermatLastTheoremThree_of_three_dvd_only_c}, we can assume that
  $\gcd(a,b)=1$, $3 \notdivides a$, $3 \notdivides b$, $3 \divides c$.
  By contradiction we assume that $a^3 + b^3 = c^3$ and let $u = 1$.
  \begin{itemize}
      \item By contradiction we assume that $\lambda \divides a$, which implies that the norm of
      $\lambda$ divides $a$ by \Cref{lmm:norm_lambda_prime}, which implies that $3 \divides a$ by
      \Cref{lmm:norm_lambda}, that contradicts the assumption that $3 \notdivides a$ forcing us
      to conclude that $\lambda \notdivides a$.
      \item By contradiction we assume that $\lambda \divides b$, which implies that the norm of
      $\lambda$ divides $b$ by \Cref{lmm:norm_lambda_prime}, which implies that $3 \divides b$ by
      \Cref{lmm:norm_lambda}, that contradicts the assumption that $3 \notdivides b$ forcing us
      to conclude that $\lambda \notdivides b$.
      \item $\lambda \divides 3$ by \Cref{lmm:lambda_dvd_three} and $3 \divides c$,
      then $\lambda \divides c$.
  \end{itemize}
  By our first assumption $a^3 + b^3 \neq u c^3 = 1 c^3 = c^3 = a^3 + b^3$ which is absurd.
\end{proof}

\section{Conclusion}
% A FEW CONCLUDING WORDS / REMARKS

\begin{theorem}[Fermat's Last Theorem for Exponent $3$]
    \label{thm:fermatLastTheoremThree}
    \lean{fermatLastTheoremThree}
    \leanok
    Let $a, b, c \in \N$. \\
    Let $a \neq 0$, $b \neq 0$ and $c \neq 0$. \\\\
    Then $a^3 + b^3 \neq c^3$.
\end{theorem}
\begin{proof}
    \leanok
    \uses{
    lmm:FermatLastTheoremForThree_of_FermatLastTheoremThreeGen,
    thm:fermatLastTheoremForThreeGen}
    Apply \Cref{lmm:FermatLastTheoremForThree_of_FermatLastTheoremThreeGen}
    and \Cref{thm:fermatLastTheoremForThreeGen}.
\end{proof}
% Acknowledgements
%% \input{chapters/3-acknowledgements}

% Bibliography
\nocite{*}
\printbibliography[title=References]
\addcontentsline{toc}{chapter}{References}
\end{document}
% Import author
\input{author}

% Set hyperref parameters
\hypersetup{
    colorlinks=true, % TRUE: colors the text of the links; FALSE: puts a colored box around them
    linkcolor=NavyBlue, % Sets color of internal links (\ref or \label)
    urlcolor=NavyBlue, % Sets color of external links
    citecolor=NavyBlue, % Sets color of citation links (\cite).
    anchorcolor=NavyBlue % Sets color of anchor text
}

% Set source for bibliography
\addbibresource{references.bib}

% Set title and subtitle
\title{Formalising Fermat's Last Theorem for Exponent 3 in Lean}
% Set date
\date{\today}

\begin{document}

% Add the title
\maketitle

% Add the abstract
\begin{abstract}
    This paper serves as the blueprint for the project aimed at formalising Fermat’s Last Theorem
    for the exponent 3 (\href{https://github.com/pitmonticone/FLT3}{FLT3}) using the Lean 4 proof assistant.
    It offers comprehensive coverage of all necessary definitions, theorems, and proofs, presented in natural,
    informal language to facilitate understanding and implementation. The related code is being integrated
    into the Lean library of formalized mathematics (\href{https://github.com/leanprover-community/mathlib4}{Mathlib})
    and imported as a dependency in the broader formalisation project of Fermat's Last Theorem led by
    Kevin Buzzard and Richard Taylor (\href{https://github.com/ImperialCollegeLondon/FLT}{FLT}).
\end{abstract}

% Add the table of contents
\tableofcontents

% Add main contents
% In this file you should put the actual content of the blueprint.
% It will be used both by the web and the print version.
% It should *not* include the \begin{document}
%
% If you want to split the blueprint content into several files then
% the current file can be a simple sequence of \input. Otherwise It
% can start with a \section or \chapter for instance.

% This is the main point of entry to the blueprint.
% Add chapters of the blueprint here.
% This file is not meant to be built. Build src/web.tex or src/print.tex instead.

% Introduction
\chapter*{Introduction}
% \label{chap:introduction}
\addcontentsline{toc}{chapter}{Introduction}

% https://formal-mathematics.github.io/intro.html

% https://github.com/pitmonticone/FLT3

% https://pitmonticone.github.io/FLT3/docs/

% https://pitmonticone.github.io/FLT3/blueprint

% https://pitmonticone.github.io/FLT3/blueprint/dep_graph_document.html

% https://pitmonticone.github.io/FLT3/blueprint.pdf


% What is Formalisation / a proof assistant / interactive theorem prover? Why is it important?

% Proof assistants, also known as interactive theorem provers, are tools for producing formally correct mathematics.
% Mathematicians write definitions, theorems, and proofs in a specialized language, with the computer giving instant feedback about each line.
% Examples of popular proof assistants include Agda, Coq, HOL Light, Isabelle, Lean, and Mizar.
% These tools have been used to develop large libraries of "known" mathematics as well as to verify major recent or controversial results.
% The act of formalizing a proof is not mechanical: it involves substantial mathematical insight in its own right,
% and it constitutes the essence of formalized mathematics, which is now recognized as a mathematical discipline on its own.
% Despite its young age, this discipline is blossoming at a very fast pace; it nowadays occupies a significant place in the mathematical horizon.

% Related to proof assistants, automated reasoning tools take in the formal statement of a theorem and attempt to prove it without further human input. While the study of these tools themselves is the domain of computer science, their use on mathematical theorems requires sophisticated reductions and reformulations.


% What is Lean? Why is it important?
%% interactive theorem prover and general-purpose programming language
% What is Mathlib? Why is it important?

% Context, Motivation, Basic Definitions?

% MOTIVATION
% It's going to be used in the general FLT project coordinated by Kevin Buzzard https://github.com/ImperialCollegeLondon/FLT

% NOTATION
% number sets

% MAIN DEFINITIONS AND STRUCTURES?

% GOAL / AIM

% CONTRIBUTIONS
% - About 20 formal / formalised proofs
% - Code:
% - All informal / informalised proofs
% - all blueprint, PRs to Mathlib (https://github.com/leanprover-community/mathlib4/pull/11677, https://github.com/leanprover-community/mathlib4/pull/11695), Porting to Mathlib (https://github.com/leanprover-community/mathlib4/pull/11767), see LaTeX comments, talk, ...
% Code refactoring


% PROOF STRATEGY

% Motivation: It's going to be used in the general FLT project
% – Disclaimer / Acknowledgement with my contributions
% – Informalisation should be clear
% – Introduction: definitions + theorem statements + proof strategy
% Porting to Mathlib
% Fermat's Last Theorem (FLT) is a famous problem in number theory. It states that there are no positive integers $a$, $b$, and $c$ such that $a^n + b^n = c^n$ for any integer value of $n$ greater than $2$. The theorem was first conjectured by Pierre de Fermat in 1637 and remained unproven for over 350 years. The first successful proof was given by Andrew Wiles in 1994. The proof is long and complex, and it relies on many different areas of mathematics, including algebraic geometry and modular forms.


% \section*{Motivation}
% \section*{Contributions}
% \section*{Overview / Structure}

% https://github.com/pitmonticone/FLT3/graphs/contributors

% SEE FLT PROJECT
% Preliminaries
\chapter{Preliminaries}
%\label{chap:preliminaries}
%\addcontentsline{toc}{chapter}{Preliminaries}

\section{Notation}

% \begin{center}
\begin{tabular}{>{\centering}m{1.8cm} m{5.8cm}}
\toprule
\textbf{Symbol} & \multicolumn{1}{c}{\textbf{Description}} \\
\midrule
$\lnot$ & Logical negation \\
$\top$  & Logical truth / Tautology \\
$\bot$  & Logical falsehood / Contradiction \\
$\land$ & Logical conjunction \\
$\lor$ & Logical inclusive disjunction \\
$:=$   & Definition \\
$\forall$ & Universal quantification \\
$\exists$ & Existential quantification \\
$\exists!$ & Unique existential quantification \\
$\N$ & Set of natural numbers \\
$\Z$ & Set of integer numbers \\
$\Z_n$ & Set of integers modulo $n$ \\
$\Q$ & Set of rational numbers \\
$[n]$ & Equivalence class of $n$ \\
$\divides$ & Divisibility relation \\
$\notdivides$ & Non-divisibility relation \\
$\gcd$ & Greatest common divisor \\
$\zeta_n$ & Primitive $n$-th root of unity \\
\bottomrule
\end{tabular}
% \end{center}

\newpage

\section{Definitions}

\begin{definition-intro}[Monoid]
   % \label{def:monoid}
    Let \(X\) be a non-empty set.\\
    Let \(\circ: X\times X \to X\) be an internal composition law on \(X\). \\\\
    A \textit{monoid} is a pair \(\cc{M}:= (X, \circ)\) satisfying:
    \begin{enumerate}
        \item [\textbf{(A)}] \(\forall x,y,z\in X,\ (x\circ y)\circ z= x\circ (y\circ z) = x\circ y \circ z\)
        \item [\textbf{(N)}] \(\exists e \in X : \forall x \in X,\ x\circ e = e \circ x = x\)
    \end{enumerate}
\end{definition-intro}

\begin{definition-intro}[Commutative Monoid]
    %\label{def:commutative_monoid}
    Let \(X\) be a non-empty set.\\
    Let \(\circ: X\times X \to X\) be an internal composition law on \(X\). \\\\
    A \textit{commutative monoid} is a pair \(\cc{M}_a:= (X, \circ)\) satisfying:
    \begin{enumerate}
        \item [\textbf{(A)}] \(\forall x,y,z\in X,\ (x\circ y)\circ z= x\circ (y\circ z) = x\circ y \circ z\)
        \item [\textbf{(N)}] \(\exists e \in X : \forall x \in X,\ x\circ e = e \circ x = x\)
        \item [\textbf{(C)}] \(\forall x,y\in X,\ x\circ y = y\circ x\)
    \end{enumerate}
\end{definition-intro}

\begin{definition-intro}[Group]
    %\label{def:group}
    Let \(X\) be a non-empty set.\\
    Let \(\circ: X\times X \to X\) be an internal composition law on \(X\). \\\\
    A \textit{group} is a pair \(\cc{G}:= (X, \circ)\) satisfying:
    \begin{enumerate}
        \item [\textbf{(A)}] \(\forall x,y,z\in X,\ (x\circ y)\circ z= x\circ (y\circ z) = x\circ y \circ z\)
        \item [\textbf{(N)}] \(\exists e \in X : \forall x \in X,\ x\circ e = e \circ x = x\)
        \item [\textbf{(I)}] \(\forall x \in X,\ \exists x'\in X: x\circ x' = x'\circ x = e\)
    \end{enumerate}
\end{definition-intro}

\begin{definition-intro}[Commutative Group]
    %\label{def:commutative_group}
    Let \(X\) be a non-empty set.\\
    Let \(\circ: X\times X \to X\) be an internal composition law on \(X\). \\\\
    A \textit{commutative group} is a pair \(\cc{G}_a:= (X, \circ)\) satisfying:
    \begin{enumerate}
        \item [\textbf{(A)}] \(\forall x,y,z\in X,\ (x\circ y)\circ z= x\circ (y\circ z) = x\circ y \circ z\)
        \item [\textbf{(N)}] \(\exists e \in X : \forall x \in X,\ x\circ e = e \circ x = x\)
        \item [\textbf{(I)}] \(\forall x \in X,\ \exists x'\in X: x\circ x' = x'\circ x = e\)
        \item [\textbf{(C)}] \(\forall x,y\in X,\ x\circ y = y\circ x\)
    \end{enumerate}
\end{definition-intro}

\begin{definition-intro}[Semiring]
    %\label{def:semiring}
\end{definition-intro}

\begin{definition-intro}[Commutative Semiring]
    %\label{def:commutative_semiring}
\end{definition-intro}

\begin{definition-intro}[Ring]
    %\label{def:ring}
\end{definition-intro}

\begin{definition-intro}[Field]
    %\label{def:field}
\end{definition-intro}
\begin{definition-intro}[Algebraic Number Field / Number Field]
    %\label{def:number_field}
\end{definition-intro}

\begin{definition-intro}[Fundamental System]
    %\label{def:fundamental_system}
\end{definition-intro}

\begin{definition-intro}[Unit]
    %\label{def:primitive_element}
\end{definition-intro}
\begin{definition-intro}[Primitive Root of Unity]
    %\label{def:primitive_root_of_unity}
\end{definition-intro}

\begin{definition-intro}[Cyclotomic Polynomial]
    %\label{def:cyclotomic_polynomial}
\end{definition-intro}
\begin{definition-intro}[Cyclotomic Extension of a Ring / Cyclotomic Ring]
    %\label{def:cyclotomic_extension_field}
\end{definition-intro}
\begin{definition-intro}[Cyclotomic Extension of a Fiel / Cyclotomic Field]
    %\label{def:cyclotomic_extension_field}
\end{definition-intro}

\section{Results}

\begin{theorem}
    \label{thm:zeta_sub_one_prime1}
    \lean{IsPrimitiveRoot.zeta_sub_one_prime'}
    \leanok
    Let $p \in \N$ be prime. \\\\
    If $\zeta_p$ is a primitive $p$-th root of unity, then $\zeta_p - 1$ is prime.
\end{theorem}
\begin{proof}
    \leanok
    This has already been formalised and included in \href{https://pitmonticone.github.io/FLT3/docs/FLT3/Mathlib/NumberTheory/Cyclotomic/Rat.html#IsPrimitiveRoot.zeta_sub_one_prime'}{Mathlib}.
\end{proof}

\begin{lemma}
    \label{lmm:fermatLastTheoremWith_of_fermatLastTheoremWith_coprime}
    \lean{fermatLastTheoremWith_of_fermatLastTheoremWith_coprime}
    \leanok
    Let $R$ be a commutative semiring, domain and normalised $\gcd$ monoid.\\% ASK EXPERTS
    Let $a, b, c \in R$. \\
    Let $n \in \N$. \\\\
    Then, to prove Fermat's Last Theorem for exponent $n$ in $R$,
    one can assume, without loss of generality, that $\gcd(a,b,c)=1$.
  \end{lemma}
  \begin{proof}
    \leanok
    This has already been formalised and included in \href{https://pitmonticone.github.io/FLT3/docs/FLT3/Mathlib/NumberTheory/FLT/Basic.html#fermatLastTheoremWith_of_fermatLastTheoremWith_coprime}{Mathlib}.
  \end{proof}

  \begin{lemma}
    \label{lmm:cube_of_castHom_ne_zero}
    \lean{cube_of_castHom_ne_zero}
    \leanok
    Let $\Z_9$ be the ring of integers modulo $9$. \\
    Let $\Z_3$ be the ring of integers modulo $3$. \\
    Let $n \in \Z_9$. \\
    Let $\phi : \Z_9 \to \Z_3$ be the canonical ring homomorphism. \\
    Let $\phi(n) \neq 0$. \\ \\
    Then $n^3=1 \lor n^3=8$.
  \end{lemma}
  \begin{proof}
    \leanok
    This has already been formalised and included in \href{https://pitmonticone.github.io/FLT3/docs/FLT3/Mathlib/NumberTheory/FLT/Three.html#cube_of_castHom_ne_zero}{Mathlib}.
  \end{proof}
% The Third Cyclotomic Field
\chapter{Third Cyclotomic Extensions}
%\chapter{Cyclotomic Extensions}
%\chapter{Cyclotomic Fields and Rings}
%\chapter{Third Cyclotomic Fields and Rings}
%\chapter{Cyclotomic Extensions of Fields and Rings}
%\label{chap:cyclo}
%\addcontentsline{toc}{chapter}{The Third Cyclotomic Field}

% INSERT BASIC DEFINITIONS
% - Number field
% - Cyclotomic extension of a field
% - Cyclotomic extension of a ring
% - Group of units
% - Fundamental system

\begin{theorem}
    \label{thm:mem}
    \lean{IsCyclotomicExtension.Rat.Three.Units.mem}
    \leanok
    Let $K = \Q(\zeta_3)$ be the third cyclotomic field. \\
    Let $\cc{O}_K = \Z[\zeta_3]$ be the ring of integers of $K$. \\
    Let $\cc{O}^\times_K$ be the group of units of $\cc{O}_K$. \\
    Let $\zeta_3 \in K$ be any primitive third root of unity. \\
    Let $\eta \in \cc{O}_K$ be the element corresponding to $\zeta_3 \in K$. \\
    Let $\lambda \in \cc{O}_K$ be such that $\lambda = \eta -1$. \\
    Let $u \in \cc{O}^\times_K$ be a unit. \\\\
    Then $u \in \set{1, -1, \eta, -\eta, \eta^2, -\eta^2}$.
\end{theorem}
\begin{proof}
    \leanok
    Let $\cc{F}$ be the fundamental system of $K$. \\
    By properties of cyclotomic fields, we know that $\rank{K} = 0$
    (see \href{https://pitmonticone.github.io/FLT3/docs/Mathlib/NumberTheory/NumberField/Embeddings.html#NumberField.InfinitePlace.card_eq_nrRealPlaces_add_nrComplexPlaces}{this lemma},
    \href{https://pitmonticone.github.io/FLT3/docs/FLT3/Mathlib/NumberTheory/Cyclotomic/Embeddings.html#IsCyclotomicExtension.Rat.nrRealPlaces_eq_zero}{this lemma}
    and \href{https://pitmonticone.github.io/FLT3/docs/FLT3/Mathlib/NumberTheory/Cyclotomic/Embeddings.html#IsCyclotomicExtension.Rat.nrComplexPlaces_eq_totient_div_two}{this lemma}
    which have already been formalised and included in Mathlib).
    By the Dirichlet Unit Theorem (see \href{https://pitmonticone.github.io/FLT3/docs/Mathlib/NumberTheory/NumberField/Units.html#NumberField.Units.exist_unique_eq_mul_prod}{Mathlib}),
    we know that
    $$\exists x \in K \text{ with finite order, such that } u = x \prod_{v\in\cc{F}} v,$$
    but since $\rank{K} = 0$, then $\cc{F} = \emptyset$, which implies that $u = x$.\\
    Since $u = x$ has finite order, by properties of primitive roots
    (see \href{https://pitmonticone.github.io/FLT3/docs/Mathlib/NumberTheory/Cyclotomic/PrimitiveRoots.html#IsPrimitiveRoot.exists_pow_or_neg_mul_pow_of_isOfFinOrder}{this lemma}
    that has already been formalised and included in Mathlib), we can deduce that
    $$\exists r < 3 \text{ such that } u = \eta^r \lor u = -\eta^r.$$
    Therefore, we can conclude
    $$u \in \setb{\pm \eta^r}{r \in \set{0,1,2}} = \set{1, -1, \eta, -\eta, \eta^2, -\eta^2}.$$
\end{proof}

\begin{theorem}
    \label{thm:not_exists_int_three_dvd_sub}
    \lean{IsCyclotomicExtension.Rat.Three.Units.not_exists_int_three_dvd_sub}
    \leanok
    Let $K = \Q(\zeta_3)$ be the third cyclotomic field. \\
    Let $\cc{O}_K = \Z[\zeta_3]$ be the ring of integers of $K$. \\
    Let $\cc{O}^\times_K$ be the group of units of $\cc{O}_K$. \\
    Let $\zeta_3 \in K$ be any primitive third root of unity. \\
    Let $\eta \in \cc{O}_K$ be the element corresponding to $\zeta_3 \in K$. \\
    Let $m \in \Z$. \\\\
    Then $3 \notdivides \eta - m$.
\end{theorem}
\begin{proof}
    \leanok
    By properties of cyclotomic fields, we know that $\set{1,\eta}$ is an integral power basis of $\cc{O}_K$
    (see \href{https://pitmonticone.github.io/FLT3/docs/FLT3/Mathlib/NumberTheory/Cyclotomic/Rat.html#IsPrimitiveRoot.integralPowerBasis'}{this lemma},
    \href{https://pitmonticone.github.io/FLT3/docs/FLT3/Mathlib/NumberTheory/Cyclotomic/Rat.html#IsPrimitiveRoot.power_basis_int'_dim}{this lemma}
    and \href{https://pitmonticone.github.io/FLT3/docs/FLT3/Mathlib/NumberTheory/Cyclotomic/Rat.html#IsPrimitiveRoot.integralPowerBasis'_gen}{this lemma}
    which have already been formalised and included in Mathlib).\\
    For every $\xi \in \cc{O}_K$, we define $\pi_1(\xi)$ and $\pi_2(\xi)$ to be the first and second
    coordinates of $\xi$ with respect to the basis $\set{1,\eta} \in \cc{O}_K$, i.e.
    $$\xi = \pi_1(\xi) + \pi_2(\xi)\eta.$$
    By contradiction we assume that
    $$\exists m \in \Z \text{ such that } 3 \divides \eta - m,$$
    which implies that
    $$\exists x \in \cc{O}_K \text{ such that } \eta - m = 3 x.$$
    By linearity of $\pi_2$,
    $$\pi_2(\eta) = \pi_2(3x + m) = 3\pi_2(x) + \pi_2(m).$$
    Since $\pi_2(\eta) = 1$ and $\pi_2(m) = 0$, then we have that $3 \divides 1$, which is a contradiction.
\end{proof}

\begin{lemma}
    \label{lmm:lambda_sq}
    \lean{IsCyclotomicExtension.Rat.Three.lambda_sq}
    \leanok
    Let $K = \Q(\zeta_3)$ be the third cyclotomic field. \\
    Let $\cc{O}_K = \Z[\zeta_3]$ be the ring of integers of $K$. \\
    Let $\cc{O}^\times_K$ be the group of units of $\cc{O}_K$. \\
    Let $\zeta_3 \in K$ be any primitive third root of unity. \\
    Let $\eta \in \cc{O}_K$ be the element corresponding to $\zeta_3 \in K$. \\
    Let $\lambda \in \cc{O}_K$ be such that $\lambda = \eta -1$. \\\\
    Then $\lambda^2 = -3 \eta$.
\end{lemma}
\begin{proof}
    \leanok
    By definition we have that $\lambda = \eta -1$, which implies that
    $$\lambda^2 = (\eta - 1)^2 = \eta^2 - 2\eta + 1.$$
    Since $\eta$ corresponds to a root of the equation $x^2 + x + 1 = 0$, then $\eta^2 = -1 - \eta$.
    Substituting back, we can conclude that
    $$\lambda^2 = (-1 - \eta) - 2\eta + 1 = -3\eta.$$
\end{proof}

\begin{theorem}
    \label{lmm:eq_one_or_neg_one_of_unit_of_congruent}
    \lean{IsCyclotomicExtension.Rat.Three.eq_one_or_neg_one_of_unit_of_congruent}
    \leanok
    Let $K = \Q(\zeta_3)$ be the third cyclotomic field. \\
    Let $\cc{O}_K = \Z[\zeta_3]$ be the ring of integers of $K$. \\
    Let $\cc{O}^\times_K$ be the group of units of $\cc{O}_K$. \\
    Let $\zeta_3 \in K$ be any primitive third root of unity. \\
    Let $\eta \in \cc{O}_K$ be the element corresponding to $\zeta_3 \in K$. \\
    Let $\lambda \in \cc{O}_K$ be such that $\lambda = \eta -1$. \\
    Let $u \in \cc{O}^\times_K$ be a unit. \\\\
    If $\exists m \in \Z$ such that $\lambda^2 \divides u - m$, then
    $u = 1 \lor u = -1$. \\
    This is a special case of the Kummer's Lemma.
\end{theorem}
\begin{proof}
    \leanok
    \uses{thm:not_exists_int_three_dvd_sub, thm:mem, lmm:lambda_sq}
    By \Cref{lmm:lambda_sq}, we have that $-3\eta = \lambda^2 \divides u - m$, which implies that
    $3 \divides u - m$.\\
    By \Cref{thm:mem}, we know that $u \in \set{1, -1, \eta, -\eta, \eta^2, -\eta^2}$. \\
    We proceed by analysing each case:
    \begin{itemize}
        \item Case $u = 1 \lor u = -1$. This finishes the proof.
        \item Case $u = \eta$.\\
              Since $3 \divides u - m$, we have that $3 \divides \eta - m$, which contradicts
              \Cref{thm:not_exists_int_three_dvd_sub} forcing us to conclude that $u \neq \eta$.
        \item Case $u = -\eta$.\\
              Since $3 \divides u - m$, we have that $3 \divides - \eta - m$, then by properties of
              divisibility $3 \divides \eta + m$, which contradicts
              \Cref{thm:not_exists_int_three_dvd_sub} forcing us to conclude that $u \neq -\eta$.
        \item Case $u = \eta^2$.\\
              Since $3 \divides u - m$, we have that $3 \divides \eta^2 - m$, which contradicts
              \Cref{thm:not_exists_int_three_dvd_sub} since $\eta^2$ is a third root of unity
              (see \href{https://pitmonticone.github.io/FLT3/docs/Mathlib/RingTheory/RootsOfUnity/Basic.html#IsPrimitiveRoot.pow_of_coprime}{Mathlib}),
              forcing us to conclude that $u \neq \eta^2$.
        \item Case $u = -\eta^2$.\\
              Since $3 \divides u - m$, we have that $3 \divides - \eta^2 - m$, then by properties of
              divisibility $3 \divides \eta^2 + m$, which contradicts
              \Cref{thm:not_exists_int_three_dvd_sub} since $\eta^2$ is a third root of unity
              (see \href{https://pitmonticone.github.io/FLT3/docs/Mathlib/RingTheory/RootsOfUnity/Basic.html#IsPrimitiveRoot.pow_of_coprime}{Mathlib}),
              forcing us to conclude that $u \neq -\eta^2$.
    \end{itemize}
    Therefore, $u = 1 \lor u = -1$.
\end{proof}

\begin{lemma}
    \label{lmm:norm_lambda}
    \lean{IsCyclotomicExtension.Rat.Three.norm_lambda}
    \leanok
    Let $K = \Q(\zeta_3)$ be the third cyclotomic field. \\
    Let $\cc{O}_K = \Z[\zeta_3]$ be the ring of integers of $K$. \\
    Let $\cc{O}^\times_K$ be the group of units of $\cc{O}_K$. \\
    Let $\zeta_3 \in K$ be any primitive third root of unity. \\
    Let $\eta \in \cc{O}_K$ be the element corresponding to $\zeta_3 \in K$. \\
    Let $\lambda \in \cc{O}_K$ be such that $\lambda = \eta -1$. \\\\
    Then the norm of $\lambda$ is $3$.
\end{lemma}
\begin{proof}
    \leanok
    Since the third cyclotomic polynomial over $\Q$ is irreducible, then the norm of $\lambda$ is $3$
    by properties of primitive roots (see
    \href{https://pitmonticone.github.io/FLT3/docs/Mathlib/NumberTheory/Cyclotomic/PrimitiveRoots.html#IsPrimitiveRoot.sub_one_norm_prime}{this lemma}
    that has already been formalised and included in Mathlib).
\end{proof}

\begin{lemma}
    \label{lmm:norm_lambda_prime}
    \lean{IsCyclotomicExtension.Rat.Three.norm_lambda_prime}
    \leanok
    Let $K = \Q(\zeta_3)$ be the third cyclotomic field. \\
    Let $\cc{O}_K = \Z[\zeta_3]$ be the ring of integers of $K$. \\
    Let $\cc{O}^\times_K$ be the group of units of $\cc{O}_K$. \\
    Let $\zeta_3 \in K$ be any primitive third root of unity. \\
    Let $\eta \in \cc{O}_K$ be the element corresponding to $\zeta_3 \in K$. \\
    Let $\lambda \in \cc{O}_K$ be such that $\lambda = \eta -1$. \\\\
    Then the norm of $\lambda$ is a prime number.
\end{lemma}
\begin{proof}
    \leanok
    \uses{lmm:norm_lambda}
    It directly follows from \Cref{lmm:norm_lambda} since $3$ is a prime number.
\end{proof}

\begin{lemma}
    \label{lmm:lambda_dvd_three}
    \lean{IsCyclotomicExtension.Rat.Three.lambda_dvd_three}
    \leanok
    Let $K = \Q(\zeta_3)$ be the third cyclotomic field. \\
    Let $\cc{O}_K = \Z[\zeta_3]$ be the ring of integers of $K$. \\
    Let $\cc{O}^\times_K$ be the group of units of $\cc{O}_K$. \\
    Let $\zeta_3 \in K$ be any primitive third root of unity. \\
    Let $\eta \in \cc{O}_K$ be the element corresponding to $\zeta_3 \in K$. \\
    Let $\lambda \in \cc{O}_K$ be such that $\lambda = \eta -1$. \\\\
    Then $\lambda \divides 3$.
\end{lemma}
\begin{proof}
    \leanok
    \uses{lmm:norm_lambda}
    By properties of norms and divisibility, if the norm of an element in the ring of integers
    divides a number, then the element itself must divide that number.
    In this case, by \Cref{lmm:norm_lambda} we know that the norm of $\lambda$ is $3$, that divides $3$,
    which implies that $\lambda \divides 3$.
\end{proof}

\begin{theorem}
    \label{thm:zeta_sub_one_prime1}
    \lean{IsPrimitiveRoot.zeta_sub_one_prime'}
    \leanok
    Let $p \in \N$ be prime. \\\\
    If $\zeta_p$ is a primitive $p$-th root of unity, then $\zeta_p - 1$ is prime.
\end{theorem}
\begin{proof}
    \leanok
    This has already been formalised and included in \href{https://pitmonticone.github.io/FLT3/docs/FLT3/Mathlib/NumberTheory/Cyclotomic/Rat.html#IsPrimitiveRoot.zeta_sub_one_prime'}{Mathlib}.
\end{proof}

\begin{lemma}
    \label{lmm:lambda_prime}
    \lean{IsPrimitiveRoot.lambda_prime}
    \leanok
    Let $K = \Q(\zeta_3)$ be the third cyclotomic field. \\
    Let $\cc{O}_K = \Z[\zeta_3]$ be the ring of integers of $K$. \\
    Let $\cc{O}^\times_K$ be the group of units of $\cc{O}_K$. \\
    Let $\zeta_3 \in K$ be any primitive third root of unity. \\
    Let $\eta \in \cc{O}_K$ be the element corresponding to $\zeta_3 \in K$. \\
    Let $\lambda \in \cc{O}_K$ be such that $\lambda = \eta -1$. \\\\
    Then $\lambda$ is prime.
\end{lemma}
\begin{proof}
    \leanok
    \uses{thm:zeta_sub_one_prime1}
    Since $3$ is prime and $\zeta_3$ is a primitive third root of unity, then $\lambda$ is prime
    by \Cref{thm:zeta_sub_one_prime1}.
\end{proof}

\begin{lemma}
    \label{lmm:lambda_ne_zero}
    \lean{IsCyclotomicExtension.Rat.Three.lambda_ne_zero}
    \leanok
    Let $K = \Q(\zeta_3)$ be the third cyclotomic field. \\
    Let $\cc{O}_K = \Z[\zeta_3]$ be the ring of integers of $K$. \\
    Let $\cc{O}^\times_K$ be the group of units of $\cc{O}_K$. \\
    Let $\zeta_3 \in K$ be any primitive third root of unity. \\
    Let $\eta \in \cc{O}_K$ be the element corresponding to $\zeta_3 \in K$. \\
    Let $\lambda \in \cc{O}_K$ be such that $\lambda = \eta -1$. \\\\
    Then $\lambda \neq 0$.
\end{lemma}
\begin{proof}
    \leanok
    \uses{lmm:lambda_prime}
    It directly follows from \Cref{lmm:lambda_prime} since zero is not prime.
\end{proof}

\begin{lemma}
    \label{lmm:lambda_not_unit}
    \lean{IsCyclotomicExtension.Rat.Three.lambda_not_unit}
    \leanok
    Let $K = \Q(\zeta_3)$ be the third cyclotomic field. \\
    Let $\cc{O}_K = \Z[\zeta_3]$ be the ring of integers of $K$. \\
    Let $\cc{O}^\times_K$ be the group of units of $\cc{O}_K$. \\
    Let $\zeta_3 \in K$ be any primitive third root of unity. \\
    Let $\eta \in \cc{O}_K$ be the element corresponding to $\zeta_3 \in K$. \\
    Let $\lambda \in \cc{O}_K$ be such that $\lambda = \eta -1$. \\\\
    Then $\lambda$ is not a unit.
\end{lemma}
\begin{proof}
    \leanok
    \uses{lmm:lambda_prime}
    It directly follows from \Cref{lmm:lambda_prime} since prime numbers are not units.
\end{proof}

\begin{lemma}
    \label{lmm:card_quot}
    \lean{IsCyclotomicExtension.Rat.Three.card_quot}
    \leanok
    Let $K = \Q(\zeta_3)$ be the third cyclotomic field. \\
    Let $\cc{O}_K = \Z[\zeta_3]$ be the ring of integers of $K$. \\
    Let $\cc{O}^\times_K$ be the group of units of $\cc{O}_K$. \\
    Let $\zeta_3 \in K$ be any primitive third root of unity. \\
    Let $\eta \in \cc{O}_K$ be the element corresponding to $\zeta_3 \in K$. \\
    Let $\lambda \in \cc{O}_K$ be such that $\lambda = \eta -1$. \\
    Let $I$ be the ideal generated by $\lambda$. \\\\
    Then $\cc{O}_K / I$ has cardinality $3$.
\end{lemma}
\begin{proof}
    \leanok
    \uses{lmm:norm_lambda}
    It directly follows from \Cref{lmm:norm_lambda} by the fundamental properties of ideals.
\end{proof}

\begin{lemma}
    \label{lmm:two_ne_zero}
    \lean{IsCyclotomicExtension.Rat.Three.two_ne_zero}
    \leanok
    Let $K = \Q(\zeta_3)$ be the third cyclotomic field. \\
    Let $\cc{O}_K = \Z[\zeta_3]$ be the ring of integers of $K$. \\
    Let $\cc{O}^\times_K$ be the group of units of $\cc{O}_K$. \\
    Let $\zeta_3 \in K$ be any primitive third root of unity. \\
    Let $\eta \in \cc{O}_K$ be the element corresponding to $\zeta_3 \in K$. \\
    Let $\lambda \in \cc{O}_K$ be such that $\lambda = \eta -1$. \\
    Let $I$ be the ideal generated by $\lambda$. \\
    Let $2 \in \cc{O}_K ⧸ I$. \\\\
    Then $2 \neq 0$.
\end{lemma}
\begin{proof}
    \leanok
    \uses{lmm:norm_lambda}
    By contradiction we assume that $2 \in I$, then, by definition,
    $\lambda$ would divide $2 \in \cc{O}_K$.
    Recall from \Cref{lmm:norm_lambda} that the norm of $\lambda$ is $3$.
    If $\lambda$ divided $2$, then by properties of divisibility in number fields,
    the norm of $\lambda$ would also divide $2$.
    However $3 \notdivides 2$ showing a contradiction.
    Therefore, $\lambda \notdivides 2$, then $2 \notin I$, which implies
    that $2 \in \mathcal{O}_K / I$ is non-zero.
\end{proof}

\begin{lemma}
    \label{lmm:lambda_not_dvd_two}
    \lean{IsCyclotomicExtension.Rat.Three.lambda_not_dvd_two}
    \leanok
    Let $K = \Q(\zeta_3)$ be the third cyclotomic field. \\
    Let $\cc{O}_K = \Z[\zeta_3]$ be the ring of integers of $K$. \\
    Let $\cc{O}^\times_K$ be the group of units of $\cc{O}_K$. \\
    Let $\zeta_3 \in K$ be any primitive third root of unity. \\
    Let $\eta \in \cc{O}_K$ be the element corresponding to $\zeta_3 \in K$. \\
    Let $\lambda \in \cc{O}_K$ be such that $\lambda = \eta -1$. \\\\
    Then $\lambda \notdivides 2$.
\end{lemma}
\begin{proof}
    \leanok
    \uses{lmm:two_ne_zero}
    By contradiction we assume that $\lambda \divides 2$, that implies that $2 \in I$
    from which it follows that $2 = 0$ contradicting \Cref{lmm:two_ne_zero}
    forcing us to conclude that $\lambda \notdivides 2$.
\end{proof}

\begin{lemma}
    \label{lmm:univ_quot}
    \lean{IsCyclotomicExtension.Rat.Three.univ_quot}
    \leanok
    Let $K = \Q(\zeta_3)$ be the third cyclotomic field. \\
    Let $\cc{O}_K = \Z[\zeta_3]$ be the ring of integers of $K$. \\
    Let $\cc{O}^\times_K$ be the group of units of $\cc{O}_K$. \\
    Let $\zeta_3 \in K$ be any primitive third root of unity. \\
    Let $\eta \in \cc{O}_K$ be the element corresponding to $\zeta_3 \in K$. \\
    Let $\lambda \in \cc{O}_K$ be such that $\lambda = \eta -1$. \\
    Let $I$ be the ideal generated by $\lambda$. \\\\
    Then $\cc{O}_K / I = \set{0, 1, -1}$.
\end{lemma}
\begin{proof}
    \leanok
    \uses{lmm:card_quot, lmm:two_ne_zero}
    %It is obvious that $\cc{O}_K / I \supseteq \set{0, 1, -1}$.
    By \Cref{lmm:card_quot}, the cardinality of $\cc{O}_K / I$ is $3$,
    so it suffices to prove that $1,-1$ and $0$ are distinct.\\
    We proceed by contradiction analysing each case:
    \begin{itemize}
        \item Case $1 = -1$. By basic algebraic properties, $1 = -1$ implies that $2 = 0$,
              which contradicts \Cref{lmm:two_ne_zero} forcing us to conclude that $1 \neq -1$.
        \item Case $1 = 0$. Trivial contradiction.
        \item Case $-1 = 0$. It implies that $1 = 0$, which is a contradiction.
    \end{itemize}
\end{proof}

\begin{lemma}
    \label{lmm:dvd_or_dvd_sub_one_or_dvd_add_one}
    \lean{IsCyclotomicExtension.Rat.Three.dvd_or_dvd_sub_one_or_dvd_add_one}
    \leanok
    Let $K = \Q(\zeta_3)$ be the third cyclotomic field. \\
    Let $\cc{O}_K = \Z[\zeta_3]$ be the ring of integers of $K$. \\
    Let $\cc{O}^\times_K$ be the group of units of $\cc{O}_K$. \\
    Let $\zeta_3 \in K$ be any primitive third root of unity. \\
    Let $\eta \in \cc{O}_K$ be the element corresponding to $\zeta_3 \in K$. \\
    Let $\lambda \in \cc{O}_K$ be such that $\lambda = \eta -1$. \\
    Let $x \in \cc{O}_K$. \\\\
    Then $(\lambda \divides x) \lor (\lambda \divides x-1) \lor (\lambda \divides x+1)$.
\end{lemma}
\begin{proof}
    \leanok
    \uses{lmm:univ_quot}
    Let $I$ be the ideal generated by $\lambda$. Let $\pi : \cc{O}_K \to \cc{O}_K / I$.\\
    By \Cref{lmm:univ_quot}, we have that $\pi(x) \in \cc{O}_K / I = \set{0, 1, -1}$.\\
    We proceed by analysing each case:
    \begin{itemize}
        \item Case $\pi(x) = 0$. By properties of ideals, $\lambda \divides x$.
        \item Case $\pi(x) = 1$. Then $0=\pi(x)-1=\pi(x-1)$, which, by properties of ideals,
        implies that $\lambda \divides x-1$.
        \item Case $\pi(x) = -1$. Then $0=\pi(x)+1=\pi(x+1)$, which, by properties of ideals,
        implies that $\lambda \divides x+1$.
    \end{itemize}
\end{proof}

\begin{lemma}
    \label{lmm:toInteger_cube_eq_one}
    \lean{IsPrimitiveRoot.toInteger_cube_eq_one}
    \leanok
    Let $K = \Q(\zeta_3)$ be the third cyclotomic field. \\
    Let $\cc{O}_K = \Z[\zeta_3]$ be the ring of integers of $K$. \\
    Let $\cc{O}^\times_K$ be the group of units of $\cc{O}_K$. \\
    Let $\zeta_3 \in K$ be any primitive third root of unity. \\
    Let $\eta \in \cc{O}_K$ be the element corresponding to $\zeta_3 \in K$. \\\\
    Then $\eta^3 = 1$.
\end{lemma}
\begin{proof}
    \leanok
    Since $\zeta_3 \in K$ is a primitive third root of unity, then $\zeta_3^3 = 1$.
    Given that $\eta \in \cc{O}_K$ is the element corresponding to $\zeta_3 \in K$, then
    $\eta^3 = 1$ by the extension of the field properties into the ring of integers.
\end{proof}

\begin{lemma}
    \label{lmm:eta_isUnit}
    \lean{IsPrimitiveRoot.eta_isUnit}
    \leanok
    Let $K = \Q(\zeta_3)$ be the third cyclotomic field. \\
    Let $\cc{O}_K = \Z[\zeta_3]$ be the ring of integers of $K$. \\
    Let $\cc{O}^\times_K$ be the group of units of $\cc{O}_K$. \\
    Let $\zeta_3 \in K$ be any primitive third root of unity. \\
    Let $\eta \in \cc{O}_K$ be the element corresponding to $\zeta_3 \in K$. \\\\
    Then $\eta$ is a unit.
\end{lemma}
\begin{proof}
    \leanok
    \uses{lmm:toInteger_cube_eq_one}
    It directly follows from \Cref{lmm:toInteger_cube_eq_one}.
\end{proof}

\begin{lemma}
    \label{lmm:toInteger_eval_cyclo}
    \lean{IsPrimitiveRoot.toInteger_eval_cyclo}
    \leanok
    Let $K = \Q(\zeta_3)$ be the third cyclotomic field. \\
    Let $\cc{O}_K = \Z[\zeta_3]$ be the ring of integers of $K$. \\
    Let $\cc{O}^\times_K$ be the group of units of $\cc{O}_K$. \\
    Let $\zeta_3 \in K$ be any primitive third root of unity. \\
    Let $\eta \in \cc{O}_K$ be the element corresponding to $\zeta_3 \in K$. \\\\
    Then $\eta^2 + \eta + 1 = 0$.
\end{lemma}
\begin{proof}
    \leanok
    Since $\eta$ corresponds to a root of the equation $x^2 + x + 1 = 0$,
    then $\eta^2 + \eta + 1 = 0$.
\end{proof}

\begin{lemma}
    \label{lmm:cube_sub_one}
    \lean{IsCyclotomicExtension.Rat.Three.cube_sub_one}
    \leanok
    Let $K = \Q(\zeta_3)$ be the third cyclotomic field. \\
    Let $\cc{O}_K = \Z[\zeta_3]$ be the ring of integers of $K$. \\
    Let $\cc{O}^\times_K$ be the group of units of $\cc{O}_K$. \\
    Let $\zeta_3 \in K$ be any primitive third root of unity. \\
    Let $\eta \in \cc{O}_K$ be the element corresponding to $\zeta_3 \in K$. \\
    Let $x \in \cc{O}_K$. \\\\
    Then $x^3 - 1 = (x - 1)(x - \eta)(x - \eta^ 2)$.
\end{lemma}
\begin{proof}
    \leanok
    \uses{lmm:toInteger_cube_eq_one, lmm:toInteger_eval_cyclo}
    Applying \Cref{lmm:toInteger_cube_eq_one} and \Cref{lmm:toInteger_eval_cyclo}, we have that
    \begin{align*}
        (x - 1)(x - \eta)(x - \eta^ 2)
        &= x^3 - x^2 (\eta^2 + \eta + 1) + x (\eta^2 + \eta + \eta^3) - \eta^3 \\
        &= x^3 - x^2 (\eta^2 + \eta + 1) + x (\eta^2 + \eta + 1) - 1 \\
        &= x^3 - 1.
    \end{align*}
\end{proof}

\begin{lemma}
    \label{lmm:lambda_dvd_mul_sub_one_mul_sub_eta_add_one}
    \lean{IsCyclotomicExtension.Rat.Three.lambda_dvd_mul_sub_one_mul_sub_eta_add_one}
    \leanok
    Let $K = \Q(\zeta_3)$ be the third cyclotomic field. \\
    Let $\cc{O}_K = \Z[\zeta_3]$ be the ring of integers of $K$. \\
    Let $\cc{O}^\times_K$ be the group of units of $\cc{O}_K$. \\
    Let $\zeta_3 \in K$ be any primitive third root of unity. \\
    Let $\eta \in \cc{O}_K$ be the element corresponding to $\zeta_3 \in K$. \\
    Let $\lambda \in \cc{O}_K$ be such that $\lambda = \eta -1$. \\
    Let $x \in \cc{O}_K$. \\\\
    Then $\lambda \divides x(x - 1)(x - (\eta + 1))$.
\end{lemma}
\begin{proof}
    \leanok
    \uses{lmm:dvd_or_dvd_sub_one_or_dvd_add_one, lmm:lambda_dvd_three}
    By \Cref{lmm:dvd_or_dvd_sub_one_or_dvd_add_one}, we have that
    $$(\lambda \divides x) \lor (\lambda \divides x-1) \lor (\lambda \divides x+1).$$
    We proceed by analysing each case:
    \begin{itemize}
        \item Case $\lambda \divides x$. \\
              By properties of divisibility, we have that
              $\lambda \divides x(x - 1)(x - (\eta + 1))$.
        \item Case $\lambda \divides x-1$. \\
              By properties of divisibility, we have that
              $\lambda \divides x(x - 1)(x - (\eta + 1))$.
        \item Case $\lambda \divides x+1$.\\
              By properties of divisibility, it suffices to prove that
              $$\lambda \divides x - (\eta + 1) = x + 1 - (\eta - 1 + 3).$$
              By definition of $\lambda$, we have that
              $$x + 1 - (\eta - 1 + 3) = x + 1 - (\lambda + 3).$$
              By properties of divisibility and \Cref{lmm:lambda_dvd_three}, we can deduce that
              $\lambda \divides \lambda + 3$.\\
              Therefore, by properties of divisibility, we can conclude that
              $$\lambda \divides x(x - 1)(x - (\eta + 1)).$$
    \end{itemize}
\end{proof}

\begin{lemma}
    \label{lmm:lambda_pow_four_dvd_cube_sub_one_of_dvd_sub_one}
    \lean{IsCyclotomicExtension.Rat.Three.lambda_pow_four_dvd_cube_sub_one_of_dvd_sub_one}
    \leanok
    Let $K = \Q(\zeta_3)$ be the third cyclotomic field. \\
    Let $\cc{O}_K = \Z[\zeta_3]$ be the ring of integers of $K$. \\
    Let $\cc{O}^\times_K$ be the group of units of $\cc{O}_K$. \\
    Let $\zeta_3 \in K$ be any primitive third root of unity. \\
    Let $\eta \in \cc{O}_K$ be the element corresponding to $\zeta_3 \in K$. \\
    Let $\lambda \in \cc{O}_K$ be such that $\lambda = \eta -1$. \\
    Let $x \in \cc{O}_K$. \\\\
    If $\lambda\divides x - 1$, then $\lambda^ 4 \divides x^3 - 1$.
\end{lemma}
\begin{proof}
    \leanok
    \uses{lmm:cube_sub_one, lmm:lambda_dvd_mul_sub_one_mul_sub_eta_add_one}
    Let $\lambda \divides x - 1$, which is equivalent to say that
    $$\exists y\in \cc{O}_K \text{ such that } x - 1 = \lambda y.$$
    By ring properties and \Cref{lmm:cube_sub_one}, we have that
    $$x^3 - 1 = \lambda^3 (y (y - 1) (y - (\eta + 1))).$$
    By properties of divisibility and \Cref{lmm:lambda_dvd_mul_sub_one_mul_sub_eta_add_one},
    we can conclude that $$\lambda^ 4 \divides x^3 - 1.$$
\end{proof}

\begin{lemma}
    \label{lmm:lambda_pow_four_dvd_cube_add_one_of_dvd_add_one}
    \lean{IsCyclotomicExtension.Rat.Three.lambda_pow_four_dvd_cube_add_one_of_dvd_add_one}
    \leanok
    Let $K = \Q(\zeta_3)$ be the third cyclotomic field. \\
    Let $\cc{O}_K = \Z[\zeta_3]$ be the ring of integers of $K$. \\
    Let $\cc{O}^\times_K$ be the group of units of $\cc{O}_K$. \\
    Let $\zeta_3 \in K$ be any primitive third root of unity. \\
    Let $\eta \in \cc{O}_K$ be the element corresponding to $\zeta_3 \in K$. \\
    Let $\lambda \in \cc{O}_K$ be such that $\lambda = \eta -1$. \\
    Let $x \in \cc{O}_K$. \\\\
    If $\lambda \divides x + 1$, then $\lambda^4 \divides x ^ 3 + 1$.
\end{lemma}
\begin{proof}
    \leanok
    \uses{lmm:lambda_pow_four_dvd_cube_sub_one_of_dvd_sub_one}
    By properties of divisibility, if $\lambda \divides x + 1$
    then $$\lambda \divides - (x + 1) = (- x) - 1.$$
    By \Cref{lmm:lambda_dvd_mul_sub_one_mul_sub_eta_add_one}, we can deduce that
    $$\lambda^ 4 \divides (-x)^3 - 1.$$
    By divisibility and ring properties we can conclude that $$\lambda^ 4 \divides x^3 + 1.$$
\end{proof}

\begin{lemma}
    \label{lmm:lambda_pow_four_dvd_cube_sub_one_or_add_one_of_lambda_not_dvd}
    \lean{IsCyclotomicExtension.Rat.Three.lambda_pow_four_dvd_cube_sub_one_or_add_one_of_lambda_not_dvd}
    \leanok
    Let $K = \Q(\zeta_3)$ be the third cyclotomic field. \\
    Let $\cc{O}_K = \Z[\zeta_3]$ be the ring of integers of $K$. \\
    Let $\cc{O}^\times_K$ be the group of units of $\cc{O}_K$. \\
    Let $\zeta_3 \in K$ be any primitive third root of unity. \\
    Let $\eta \in \cc{O}_K$ be the element corresponding to $\zeta_3 \in K$. \\
    Let $\lambda \in \cc{O}_K$ be such that $\lambda = \eta -1$. \\
    Let $x \in \cc{O}_K$. \\\\
    If $\lambda \notdivides x$, then $(\lambda^4 \divides x^3 - 1)
    \lor (\lambda^4 \divides x^3 + 1)$.
\end{lemma}
\begin{proof}
    \leanok
    \uses{lmm:dvd_or_dvd_sub_one_or_dvd_add_one,
    lmm:lambda_pow_four_dvd_cube_sub_one_of_dvd_sub_one,
    lmm:lambda_pow_four_dvd_cube_add_one_of_dvd_add_one}
    By \Cref{lmm:dvd_or_dvd_sub_one_or_dvd_add_one}, we have that
    $$(\lambda \divides x) \lor (\lambda \divides x-1) \lor (\lambda \divides x+1).$$
    We proceed by analysing each case:
    \begin{itemize}
        \item Case $\lambda \divides x$. From trivially contradictory hypotheses we can conclude that
        $$(\lambda^4 \divides x^3 - 1) \lor (\lambda^4 \divides x^3 + 1).$$
        \item Case $\lambda \divides x-1$. By \Cref{lmm:lambda_pow_four_dvd_cube_sub_one_of_dvd_sub_one},
        we have that $\lambda^ 4 \divides x^3 - 1$, which implies that
        $$(\lambda^4 \divides x^3 - 1) \lor (\lambda^4 \divides x^3 + 1).$$
        \item Case $\lambda \divides x+1$. By \Cref{lmm:lambda_pow_four_dvd_cube_add_one_of_dvd_add_one},
        we have that $\lambda^ 4 \divides x^3 + 1$, which implies that
        $$(\lambda^4 \divides x^3 - 1) \lor (\lambda^4 \divides x^3 + 1).$$
    \end{itemize}
\end{proof}
% Fermat's Last Theorem for Exponent 3

\chapter{Fermat's Last Theorem for Exponent 3}
%\label{chap:flt3}
%\addcontentsline{toc}{chapter}{Fermat's Last Theorem for Exponent 3}

\section{Case 1}

\begin{lemma}
  \label{lmm:cube_of_castHom_ne_zero}
  \lean{cube_of_castHom_ne_zero}
  \leanok
  Let $\Z_9$ be the ring of integers modulo $9$. \\
  Let $\Z_3$ be the ring of integers modulo $3$. \\
  Let $n \in \Z_9$. \\
  Let $\phi : \Z_9 \to \Z_3$ be the canonical ring homomorphism. \\
  Let $\phi(n) \neq 0$. \\ \\
  Then $n^3=1 \lor n^3=8$.
\end{lemma}
\begin{proof}
  \leanok
  This has already been formalised and included in \href{https://pitmonticone.github.io/FLT3/docs/FLT3/Mathlib/NumberTheory/FLT/Three.html#cube_of_castHom_ne_zero}{Mathlib}.
\end{proof}

\begin{lemma}
  \label{lmm:cube_of_not_dvd}
  \lean{cube_of_not_dvd}
  \leanok
  Let $n \in \N$. \\
  Let $\left[n \right] \in \Z_9$. \\
  Let $3 \notdivides n$. \\ \\
  Then $\left[n \right]^3 = 1 \lor \left[n \right]^3 = 8$.
\end{lemma}
\begin{proof}
  \leanok
  \uses{lmm:cube_of_castHom_ne_zero}
  By \Cref{lmm:cube_of_castHom_ne_zero}, we can conclude that $\left[n \right]^3 = 1 \lor \left[n \right]^3 = 8$.
\end{proof}

\begin{theorem}[Fermat's Last Theorem for 3: Case 1]
    \label{thm:fermatLastTheoremThree_case_1}
    \lean{fermatLastTheoremThree_case_1}
    \leanok
    Let $a, b, c \in \N$. \\
    Let $3 \notdivides abc$. \\\\
    Then $a ^ 3 + b ^ 3 \neq c ^ 3$.
\end{theorem}
\begin{proof}
  \leanok
  \uses{lmm:cube_of_not_dvd}
  By hypothesis we know that $3 \notdivides abc$, which implies that $3 \notdivides a$, $3 \notdivides b$ and $3 \notdivides c$. \\
  It is enough to apply \Cref{lmm:cube_of_not_dvd} repeatedly and compute each case.
\end{proof}

\section{Case 2}

\begin{lemma}
  \label{lmm:three_dvd_gcd_of_dvd_a_of_dvd_b}
  \lean{three_dvd_gcd_of_dvd_a_of_dvd_b}
  \leanok
  Let $a, b, c \in \N$. \\
  Let $3 \divides a$ and $3 \divides b$. \\
  Let $a ^ 3 + b ^ 3 = c ^ 3$. \\\\
  Then $3 \divides \gcd(a,b,c)$.
\end{lemma}
\begin{proof}
  \leanok
  By hypothesis we have that $3 \divides a^3 + b^3 = c^3$, which implies that $3 \divides c$,
  from which we can conclude that $3 \divides \gcd(a,b,c)$.
\end{proof}

\begin{lemma}
  \label{lmm:three_dvd_gcd_of_dvd_a_of_dvd_c}
  \lean{three_dvd_gcd_of_dvd_a_of_dvd_c}
  \leanok
  Let $a, b, c \in \N$. \\
  Let $3 \divides a$ and $3 \divides c$. \\
  Let $a ^ 3 + b ^ 3 = c ^ 3$. \\\\
  Then $3 \divides \gcd(a,b,c)$.
\end{lemma}
\begin{proof}
  \leanok
  By hypothesis we have that $3 \divides c^3 - a^3 = b^3$, which implies that $3 \divides b$,
  from which we can conclude that $3 \divides \gcd(a,b,c)$.
\end{proof}

\begin{lemma}
  \label{lmm:three_dvd_gcd_of_dvd_b_of_dvd_c}
  \lean{three_dvd_gcd_of_dvd_b_of_dvd_c}
  \leanok
  Let $a, b, c \in \N$. \\
  Let $3 \divides b$ and $3 \divides c$. \\
  Let $a ^ 3 + b ^ 3 = c ^ 3$. \\\\
  Then $3 \divides \gcd(a,b,c)$.
\end{lemma}
\begin{proof}
  \leanok
  By hypothesis we have that $3 \divides c^3 - b^3 = a^3$, which implies that $3 \divides a$,
  from which we can conclude that $3 \divides \gcd(a,b,c)$.
\end{proof}

\begin{lemma}
  \label{lmm:fermatLastTheoremWith_of_fermatLastTheoremWith_coprime}
  \lean{fermatLastTheoremWith_of_fermatLastTheoremWith_coprime}
  \leanok
  Let $R$ be a commutative semiring, domain and normalised gcd monoid.\\% ASK EXPERTS
  Let $a, b, c \in R$. \\
  Let $n \in \N$. \\\\
  Then, to prove Fermat's Last Theorem for exponent $n$ in $R$,
  one can assume, without loss of generality, that $\gcd(a,b,c)=1$.
\end{lemma}
\begin{proof}
  \leanok
  This has already been formalised and included in \href{https://pitmonticone.github.io/FLT3/docs/FLT3/Mathlib/NumberTheory/FLT/Basic.html#fermatLastTheoremWith_of_fermatLastTheoremWith_coprime}{Mathlib}.
\end{proof}

\begin{theorem}
  \label{thm:fermatLastTheoremThree_of_three_dvd_only_c}
  \lean{fermatLastTheoremThree_of_three_dvd_only_c}
  \leanok
  To prove \Cref{thm:fermatLastTheoremThree}, it suffices to prove that
  $$\forall a, b, c \in \Z, \text{ if } c \neq 0 \text{ and } 3 \notdivides a \text{ and }
  3 \notdivides b \text{ and } 3 \divides c \text{ and } \gcd(a,b)=1,
  \text{ then } a^3 + b^3 \neq c^3.$$
  Equivalently, $$\forall a, b, c \in \Z, \text{ if } c \neq 0 \text{ and } 3 \notdivides a \text{ and }
  3 \notdivides b \text{ and } 3 \divides c \text{ and } \gcd(a,b)=1,
  \text{ then } a^3 + b^3 \neq c^3$$ implies \Cref{thm:fermatLastTheoremThree}.
\end{theorem}
\begin{proof}
  \leanok
  \uses{lmm:fermatLastTheoremWith_of_fermatLastTheoremWith_coprime,
  thm:fermatLastTheoremThree_case_1,
  lmm:three_dvd_gcd_of_dvd_a_of_dvd_b,
  lmm:three_dvd_gcd_of_dvd_a_of_dvd_c,
  lmm:three_dvd_gcd_of_dvd_b_of_dvd_c}
  By contradiction we assume that
  $$\exists a,b,c \in \N \smallsetminus \set{0} \text{ such that } a^3 + b^3 = c^3.$$
  By \Cref{lmm:fermatLastTheoremWith_of_fermatLastTheoremWith_coprime}
  we can assume that $\gcd(a,b,c)=1$. \\
  By \Cref{thm:fermatLastTheoremThree_case_1} we can assume that $3 \divides a b c$,
  from which it follows that $$(3 \divides a) \lor (3 \divides b) \lor (3 \divides c).$$
  We proceed by analysing each case:
  \begin{itemize}
      \item Case $3 \divides a$. \\
      Let $a'=-c$, $b'=b$, $c'=-a$, then $3 \divides c'$ and
      $$(a'\neq 0) \land (b'\neq 0) \land (c' \neq 0).$$
      Then $3 \notdivides a'$ since otherwise by \Cref{lmm:three_dvd_gcd_of_dvd_a_of_dvd_c}
      we would have that $3 \divides \gcd(a,b,c)=1$ which is absurd. \\
      Analogously, by \Cref{lmm:three_dvd_gcd_of_dvd_a_of_dvd_b} we have that $3 \notdivides b'$.\\
      By contradiction we assume that $\gcd(a',b') \neq 1$ which, by basic divisibility properties,
      implies that there is a prime $p$ such that $p \divides a'$ and $p \divides b'$.
      It follows that $p \divides b'^3 + a'^3 = b^3 - c^3 = -a^3$, which implies that $p \divides a$.\\
      Therefore $p \divides \gcd(a,b,c)=1$ which is absurd. \\
      Moreover, we have that $a'^3 + b'^3 = -c^3 + b^3 = -a^3 = c'^3$ that contradicts our hypothesis.
      \item Case $3 \divides b$. \\
      Let $a'=a$, $b'=-c$, $c'=-b$.\\
      The rest of the proof is analogous to the first case using \Cref{lmm:three_dvd_gcd_of_dvd_a_of_dvd_b} and
      \Cref{lmm:three_dvd_gcd_of_dvd_b_of_dvd_c}.
      \item Case $3 \divides c$.
      Let $a'=a$, $b'=b$, $c'=c$.\\
      The rest of the proof is analogous to the first case using \Cref{lmm:three_dvd_gcd_of_dvd_a_of_dvd_c} and
      \Cref{lmm:three_dvd_gcd_of_dvd_b_of_dvd_c}.
  \end{itemize}
  Therefore, we can conclude that $a^3 + b^3 \neq c^3$.
\end{proof}

\begin{definition}[Solution']
  \label{def:Solution1}
  \lean{Solution'}
  \leanok
  Let $a, b, c \in \cc{O}_K$ such that $c \neq 0$ and $\gcd(a,b)=1$.\\
  Let $\lambda \notdivides a$, $\lambda \notdivides b$ and $\lambda \divides c$. \\\\
  A $\boldsymbol{solution'}$ is a tuple $S'=(a, b, c, u)$
  satisfying the equation $a^3 + b^3 = u c^3.$
\end{definition}

\begin{definition}[Solution]
  \label{def:Solution}
  \lean{Solution}
  \leanok
  Let $a, b, c \in \cc{O}_K$ such that $c \neq 0$ and $\gcd(a,b)=1$.\\
  Let $\lambda \notdivides a$, $\lambda \notdivides b$, $\lambda \divides c$ and
  $\lambda^2 \divides a+b$. \\\\
  A $\boldsymbol{solution}$ is a tuple $S=(a, b, c, u)$
  satisfying the equation $a^3 + b^3 = u c^3$.
\end{definition}

\begin{definition}[Multiplicity of Solution']
  \label{def:Solution1_Multiplicity}
  \lean{Solution'.multiplicity}
  \leanok
  \uses{def:Solution1}
  Let $S'=(a, b, c, u)$ be a $solution'$. \\\\
  The $\boldsymbol{multiplicity}$ of $S'$ is the largest $n \in \N$ such that
  $\lambda^n \divides c$.
\end{definition}

\begin{definition}[Multiplicity of Solution]
  \label{def:Solution_Multiplicity}
  \lean{Solution.multiplicity}
  \leanok
  \uses{def:Solution}
  Let $S=(a, b, c, u)$ be a $solution$. \\\\
  The $\boldsymbol{multiplicity}$ of $S$ is the largest $n \in \N$ such that
  $\lambda^n \divides c$.
\end{definition}

\begin{definition}[Minimal Solution]
  \label{def:Solution_Minimal}
  \lean{Solution.isMinimal}
  \leanok
  \uses{def:Solution}
  Let $S=(a, b, c, u)$ be a $solution$. \\\\
  We say that $S$ is $\boldsymbol{minimal}$ if for all solutions $S_1=(a_1,b_1,c_1,u_1)$,
  the $multiplicity$ of $S$ is less than or equal to the $multiplicity$ of $S_1$.
\end{definition}

\begin{lemma}
  \label{lmm:multiplicity_lambda_c_finite}
  \lean{Solution'.multiplicity_lambda_c_finite}
  \leanok
  \uses{def:Solution1}
  Let $S'=(a, b, c, u)$ be a $solution'$. \\\\
  Then the multiplicity of $S'$ is finite.
\end{lemma}
\begin{proof}
  \leanok
  \uses{lmm:lambda_not_unit}
  It directly follows from \Cref{lmm:lambda_not_unit}.
\end{proof}

\begin{lemma}
  \label{lmm:exists_minimal}
  \lean{Solution.exists_minimal}
  \leanok
  \uses{def:Solution, def:Solution_Minimal}
  Let $S$ be a $solution$ with multiplicity $n$. \\\\
  Then there is a minimal solution $S_1$.
\end{lemma}
\begin{proof}
  \leanok
  Straightforward since $n \in \N$ and $\N$ is well-ordered.
\end{proof}

\begin{lemma}
  \label{lmm:a_cube_b_cube_same_congr}
  \lean{a_cube_b_cube_same_congr}
  \leanok
  \uses{def:Solution1}
  Let $S'=(a, b, c, u)$ be a $solution'$. \\\\
  Then $\lambda^4 \divides a^3 - 1 \land \lambda^4 \divides b^3 + 1$ or
  $\lambda^4 \divides a^3 + 1 \land \lambda^4 \divides b^3 - 1$.
\end{lemma}
\begin{proof}
  \leanok
  \uses{lmm:lambda_pow_four_dvd_cube_sub_one_or_add_one_of_lambda_not_dvd,
  lmm:lambda_not_dvd_two}
  Since $\lambda \notdivides a$, then
  $\lambda^4 \divides a^3 - 1 \lor \lambda^4 \divides a^3 + 1$ by
  \Cref{lmm:lambda_pow_four_dvd_cube_sub_one_or_add_one_of_lambda_not_dvd}.
  Since $\lambda \notdivides b$, then
  $\lambda^4 \divides b^3 - 1 \lor \lambda^4 \divides b^3 + 1$ by
  \Cref{lmm:lambda_pow_four_dvd_cube_sub_one_or_add_one_of_lambda_not_dvd}.
  We proceed by analysing each case:
  \begin{itemize}
      \item Case $\lambda^4 \divides a^3 - 1 \land \lambda^4 \divides b^3 - 1$.
      Since $\lambda \divides c$ we have that $\lambda \divides c^3-(a^3-1)-(b^3-1) = 2$,
      which is absurd by \Cref{lmm:lambda_not_dvd_two}.
      \item Case $\lambda^4 \divides a^3 + 1 \land \lambda^4 \divides b^3 + 1$.
      Since $\lambda \divides c$ we have that $\lambda \divides (a^3-1)+(b^3-1)-c^3 = 2$,
      which is absurd by \Cref{lmm:lambda_not_dvd_two}.
      \item Case $\lambda^4 \divides a^3 - 1 \land \lambda^4 \divides b^3 + 1$. Trivial.
      \item Case $\lambda^4 \divides a^3 + 1 \land \lambda^4 \divides b^3 - 1$. Trivial.
  \end{itemize}
\end{proof}

\begin{lemma}
  \label{lmm:lambda_pow_four_dvd_c_cube}
  \lean{lambda_pow_four_dvd_c_cube}
  \leanok
  \uses{def:Solution1}
  Let $S'=(a, b, c, u)$ be a $solution'$. \\\\
  Then $\lambda^4 \divides c^3$.
\end{lemma}
\begin{proof}
  \leanok
  \uses{lmm:a_cube_b_cube_same_congr}
  Apply \Cref{lmm:a_cube_b_cube_same_congr} and then compute each case.
\end{proof}

\begin{lemma}
  \label{lmm:lambda_pow_two_dvd_c}
  \lean{lambda_pow_two_dvd_c}
  \leanok
  \uses{def:Solution1}
  Let $S'=(a, b, c, u)$ be a $solution'$. \\\\
  Then $\lambda^2 \divides c$.
\end{lemma}
\begin{proof}
  \leanok
  \uses{lmm:multiplicity_lambda_c_finite,
  lmm:lambda_pow_four_dvd_c_cube, lmm:lambda_prime}
  Apply \Cref{lmm:lambda_pow_four_dvd_c_cube}.
\end{proof}

\begin{lemma}
  \label{lmm:Solution1_two_le_multiplicity}
  \lean{Solution'.two_le_multiplicity}
  \leanok
  \uses{def:Solution1}
  Let $S'=(a, b, c, u)$ be a $solution'$ with multiplicity $n$.\\\\
  Then $2 \leq n$.
\end{lemma}
\begin{proof}
  \leanok
  \uses{lmm:lambda_pow_two_dvd_c}
  It directly follows from \Cref{lmm:lambda_pow_two_dvd_c}.
\end{proof}

\begin{lemma}
  \label{lmm:Solution_two_le_multiplicity}
  \lean{Solution.two_le_multiplicity}
  \leanok
  \uses{def:Solution}
  Let $S=(a, b, c, u)$ be a $solution$ with multiplicity $n$.\\\\
  Then $2 \leq n$.
\end{lemma}
\begin{proof}
  \leanok
  \uses{lmm:Solution1_two_le_multiplicity}
  It directly follows from \Cref{lmm:Solution1_two_le_multiplicity}.
\end{proof}

\begin{lemma}
  \label{lmm:cube_add_cube_eq_mul}
  \lean{cube_add_cube_eq_mul}
  \leanok
  \uses{def:Solution1}
  Let $K = \Q(\zeta_3)$ be the third cyclotomic field. \\
  Let $\cc{O}_K = \Z[\zeta_3]$ be the ring of integers of $K$. \\
  Let $\cc{O}^\times_K$ be the group of units of $\cc{O}_K$. \\
  Let $\zeta_3 \in K$ be any primitive third root of unity. \\
  Let $\eta \in \cc{O}_K$ be the element corresponding to $\zeta_3 \in K$. \\
  Let $S'=(a, b, c, u)$ be a $solution'$.\\\\
  Then $a^3 + b^3 = (a + b) (a + \eta b) (a + \eta^2 b)$.
\end{lemma}
\begin{proof}
  \leanok
  \uses{lmm:toInteger_cube_eq_one, lmm:toInteger_eval_cyclo}
  Straightforward calculation using \Cref{lmm:toInteger_cube_eq_one}
  and \Cref{lmm:toInteger_eval_cyclo}.
\end{proof}

\begin{lemma}
  \label{lmm:lambda_sq_dvd_or_dvd_or_dvd}
  \lean{lambda_sq_dvd_or_dvd_or_dvd}
  \leanok
  \uses{def:Solution1}
  Let $K = \Q(\zeta_3)$ be the third cyclotomic field. \\
  Let $\cc{O}_K = \Z[\zeta_3]$ be the ring of integers of $K$. \\
  Let $\cc{O}^\times_K$ be the group of units of $\cc{O}_K$. \\
  Let $\zeta_3 \in K$ be any primitive third root of unity. \\
  Let $\eta \in \cc{O}_K$ be the element corresponding to $\zeta_3 \in K$. \\
  Let $\lambda \in \cc{O}_K$ be such that $\lambda = \eta -1$. \\
  Let $S'=(a, b, c, u)$ be a $solution'$.\\\\
  Then $(\lambda^2 \divides a + b) \lor (\lambda^2 \divides a +
  \eta b) \lor (\lambda^2 \divides a + \eta^2 b)$.
\end{lemma}
\begin{proof}
  \leanok
  \uses{lmm:lambda_pow_two_dvd_c, lmm:cube_add_cube_eq_mul,
  lmm:lambda_prime}
  By contradiction we assume that
  $$(\lambda^2 \notdivides a + b) \land (\lambda^2 \notdivides a +
  \eta b) \land (\lambda^2 \notdivides a + \eta^2 b).$$
  Then, by definition, the multiplicity of $\lambda$ in $a + b$, in $a +
  \eta b$ and in $a + \eta^2 b$ is less than $2$.
  By properties of divisibility, \Cref{lmm:lambda_pow_two_dvd_c} and \Cref{lmm:cube_add_cube_eq_mul},
  we have that
  $$\lambda^6 ∣ u c^3 = a^3 + b^3 = (a + b) (a + \eta b) (a + \eta^2 b).$$
  Then, the multiplicity of $\lambda$ in $(a + b) (a + \eta b) (a + \eta^2 b)$ is greater than
  or equal to $6$. \\
  By \Cref{lmm:lambda_prime} $\lambda$ is prime, so we have that the multiplicity of $\lambda$
  in $(a + b) (a + \eta b) (a + \eta^2 b)$ is the sum of the multiplicities of $\lambda$ in
  $a + b$, in $a + \eta b$ and in $a + \eta^2 b$, which is less than $6$.
  This is a contradiction that forces us to conclude that
  $$(\lambda^2 \divides a + b) \lor (\lambda^2 \divides a +
  \eta b) \lor (\lambda^2 \divides a + \eta^2 b).$$
\end{proof}

\begin{lemma}
  \label{lmm:ex_dvd_a_add_b}
  \lean{ex_dvd_a_add_b}
  \leanok
  \uses{def:Solution1}
  Let $S'=(a, b, c, u)$ be a $solution'$.\\\\
  Then $\exists a_1,b_1 \in \cc{O}_k$ such that $S_1=(a_1,b_1,c,u)$ is a $solution$.
\end{lemma}
\begin{proof}
  \leanok
  \uses{lmm:lambda_sq_dvd_or_dvd_or_dvd, lmm:toInteger_cube_eq_one, lmm:eta_isUnit}
  By \Cref{lmm:lambda_sq_dvd_or_dvd_or_dvd}, we have that
  $$(\lambda^2 \divides a + b) \lor (\lambda^2 \divides a +
  \eta b) \lor (\lambda^2 \divides a + \eta^2 b).$$
  We proceed by analysing each case:
  \begin{itemize}
      \item Case $\lambda^2 \divides a + b$. Trivial using $a_1=a$ and $b_1=b$.
      \item Case $\lambda^2 \divides a + \eta b$. Let $a_1=a$ and $b_1=\eta b$. \\
      By \Cref{lmm:toInteger_cube_eq_one}, we have that $a^3 + (\eta b)^3 = a^3 + b^3 = u c^3$.\\
      By properties of coprimes and \Cref{lmm:eta_isUnit}, we have that
      $\gcd(a,b)=1$ implies that $\gcd(a,\eta b)=1$.\\
      Since $a_1=a$, we already know that $\lambda \notdivides a = a_1$.\\
      By contradiction we assume that $\lambda \divides b_1 = \eta b$, which,
      by \Cref{lmm:toInteger_cube_eq_one}, it implies that $\lambda \divides \eta^2 \eta b = b$
      that contradicts our assumption, forcing us to conclude that $\lambda \notdivides b_1$.
      \item Case $\lambda^2 \divides a + \eta^2 b$. Let $a_1=a$ and $b_1=\eta^2 b$. \\
      By \Cref{lmm:toInteger_cube_eq_one}, we have that $a^3 + (\eta^2 b)^3 = a^3 + b^3 = u c^3$.\\
      By properties of coprimes and \Cref{lmm:eta_isUnit}, we have that
      $\gcd(a,b)=1$ implies that $\gcd(a,\eta^2 b)=1$.\\
      Since $a_1=a$, we already know that $\lambda \notdivides a = a_1$.\\
      By contradiction we assume that $\lambda \divides b_1 = \eta^2 b$, which,
      by \Cref{lmm:toInteger_cube_eq_one}, it implies that $\lambda \divides \eta \eta^2 b = b$
      that contradicts our assumption, forcing us to conclude that $\lambda \notdivides b_1$.
  \end{itemize}
  Therefore, we can conclude that
  $\exists a_1,b_1 \in \cc{O}_k$ such that $S_1=(a_1,b_1,c,u)$ is a $solution$.
\end{proof}

\begin{lemma}
  \label{lmm:exists_Solution_of_Solution1}
  \lean{exists_Solution_of_Solution'}
  \leanok
  \uses{def:Solution1, def:Solution}
  Let $S'$ be a $solution'$ with multiplicity $n$.\\\\
  Then there is a $solution\text{ }S$ with multiplicity $n$.
\end{lemma}
\begin{proof}
  \leanok
  \uses{lmm:ex_dvd_a_add_b}
  Let $S'=(a',b',c',u')$. Let $a, b$ be the units given by \Cref{lmm:ex_dvd_a_add_b}.
  Then $S=(a,b,c',u')$ is a $solution'$ with multiplicity $n$.
\end{proof}

\begin{lemma}
  \label{lmm:a_add_eta_b}
  \lean{Solution.a_add_eta_b}
  \leanok
  \uses{def:Solution}
  Let $K = \Q(\zeta_3)$ be the third cyclotomic field. \\
  Let $\cc{O}_K = \Z[\zeta_3]$ be the ring of integers of $K$. \\
  Let $\cc{O}^\times_K$ be the group of units of $\cc{O}_K$. \\
  Let $\zeta_3 \in K$ be any primitive third root of unity. \\
  Let $\eta \in \cc{O}_K$ be the element corresponding to $\zeta_3 \in K$. \\
  Let $S=(a, b, c, u)$ be a $solution$.\\\\
  Then $a + \eta  b = (a + b) + \lambda  b$.
\end{lemma}
\begin{proof}
  \leanok
  Trivial calculation.
\end{proof}

\begin{lemma}
  \label{lmm:lambda_dvd_a_add_eta_mul_b}
  \lean{Solution.lambda_dvd_a_add_eta_mul_b}
  \leanok
  \uses{def:Solution}
  Let $K = \Q(\zeta_3)$ be the third cyclotomic field. \\
  Let $\cc{O}_K = \Z[\zeta_3]$ be the ring of integers of $K$. \\
  Let $\cc{O}^\times_K$ be the group of units of $\cc{O}_K$. \\
  Let $\zeta_3 \in K$ be any primitive third root of unity. \\
  Let $\eta \in \cc{O}_K$ be the element corresponding to $\zeta_3 \in K$. \\
  Let $\lambda \in \cc{O}_K$ be such that $\lambda = \eta -1$. \\
  Let $S=(a, b, c, u)$ be a $solution$.\\\\
  Then $\lambda \divides a + \eta  b$.
\end{lemma}
\begin{proof}
  \leanok
  \uses{lmm:a_add_eta_b}
  Trivial since $\lambda \divides a+b$.
\end{proof}

\begin{lemma}
  \label{lmm:lambda_dvd_a_add_eta_sq_mul_b}
  \lean{Solution.lambda_dvd_a_add_eta_sq_mul_b}
  \leanok
  \uses{def:Solution}
  Let $K = \Q(\zeta_3)$ be the third cyclotomic field. \\
  Let $\cc{O}_K = \Z[\zeta_3]$ be the ring of integers of $K$. \\
  Let $\cc{O}^\times_K$ be the group of units of $\cc{O}_K$. \\
  Let $\zeta_3 \in K$ be any primitive third root of unity. \\
  Let $\eta \in \cc{O}_K$ be the element corresponding to $\zeta_3 \in K$. \\
  Let $\lambda \in \cc{O}_K$ be such that $\lambda = \eta -1$. \\
  Let $S=(a, b, c, u)$ be a $solution$.\\\\
  Then $\lambda \divides a + \eta^2  b$.
\end{lemma}
\begin{proof}
  \leanok
  Since $\lambda \divides a+b$, then
  $\lambda \divides (a + b) + \lambda^2  b + 2  \lambda  b
  = a + \eta^2  b$.
\end{proof}

\begin{lemma}
  \label{lmm:lambda_sq_not_dvd_a_add_eta_mul_b}
  \lean{Solution.lambda_sq_not_dvd_a_add_eta_mul_b}
  \leanok
  \uses{def:Solution}
  Let $K = \Q(\zeta_3)$ be the third cyclotomic field. \\
  Let $\cc{O}_K = \Z[\zeta_3]$ be the ring of integers of $K$. \\
  Let $\cc{O}^\times_K$ be the group of units of $\cc{O}_K$. \\
  Let $\zeta_3 \in K$ be any primitive third root of unity. \\
  Let $\eta \in \cc{O}_K$ be the element corresponding to $\zeta_3 \in K$. \\
  Let $\lambda \in \cc{O}_K$ be such that $\lambda = \eta -1$. \\
  Let $S=(a, b, c, u)$ be a $solution$.\\\\
  Then $\lambda^2 \notdivides a + \eta b$.
\end{lemma}
\begin{proof}
  \leanok
  \uses{lmm:a_add_eta_b, lmm:lambda_ne_zero}
  By contradiction we assume that $\lambda^2 \divides a + \eta b$, which implies that
  $\lambda^2 \divides a + b + \lambda  b$ by \Cref{lmm:a_add_eta_b}.
  Since $\lambda^2 \divides a+b$, then $\lambda^2 \divides \lambda  b$, which implies that
  $\lambda \divides b$, that contradicts \Cref{def:Solution} forcing us to conclude that
  $\lambda^2 \notdivides a + \eta b$.
\end{proof}

\begin{lemma}
  \label{lmm:lambda_sq_not_dvd_a_add_eta_sq_mul_b}
  \lean{Solution.lambda_sq_not_dvd_a_add_eta_sq_mul_b}
  \leanok
  \uses{def:Solution}
  Let $K = \Q(\zeta_3)$ be the third cyclotomic field. \\
  Let $\cc{O}_K = \Z[\zeta_3]$ be the ring of integers of $K$. \\
  Let $\cc{O}^\times_K$ be the group of units of $\cc{O}_K$. \\
  Let $\zeta_3 \in K$ be any primitive third root of unity. \\
  Let $\eta \in \cc{O}_K$ be the element corresponding to $\zeta_3 \in K$. \\
  Let $\lambda \in \cc{O}_K$ be such that $\lambda = \eta -1$. \\
  Let $S=(a, b, c, u)$ be a $solution$.\\\\
  Then $\lambda^2 \notdivides a + \eta^2  b$.
\end{lemma}
\begin{proof}
  \leanok
  \uses{lmm:lambda_ne_zero, lmm:toInteger_eval_cyclo}
  By contradiction using \Cref{lmm:toInteger_eval_cyclo}, we assume
  $\lambda^2 \divides a +\eta^2 b = a + b -b + \eta^2  b$.
  Since $\lambda^2 \divides a+b$, then $\lambda^2 \divides b (\eta^2 -1)
  = \lambda b (\eta + 1)$. Since $\lambda \notdivides b$, then
  $\lambda \divides \eta+1 = \lambda +2$, then $\lambda \divides 2$ which is absurd.
\end{proof}

\begin{lemma}
  \label{lmm:eta_add_one_inv}
  \lean{Solution.eta_add_one_inv}
  \leanok
  \uses{def:Solution}
  Let $K = \Q(\zeta_3)$ be the third cyclotomic field. \\
  Let $\cc{O}_K = \Z[\zeta_3]$ be the ring of integers of $K$. \\
  Let $\cc{O}^\times_K$ be the group of units of $\cc{O}_K$. \\
  Let $\zeta_3 \in K$ be any primitive third root of unity. \\
  Let $\eta \in \cc{O}_K$ be the element corresponding to $\zeta_3 \in K$. \\
  Let $S=(a, b, c, u)$ be a $solution$.\\\\
  Then $(\eta + 1)  (-\eta) = 1$.
\end{lemma}
\begin{proof}
  \leanok
  \uses{lmm:toInteger_eval_cyclo}
  Trivial calculation using \Cref{lmm:toInteger_eval_cyclo}.
\end{proof}

\begin{lemma}
  \label{lmm:associated_of_dvd_a_add_b_of_dvd_a_add_eta_mul_b}
  \lean{Solution.associated_of_dvd_a_add_b_of_dvd_a_add_eta_mul_b}
  \leanok
  \uses{def:Solution}
  Let $K = \Q(\zeta_3)$ be the third cyclotomic field. \\
  Let $\cc{O}_K = \Z[\zeta_3]$ be the ring of integers of $K$. \\
  Let $\cc{O}^\times_K$ be the group of units of $\cc{O}_K$. \\
  Let $\zeta_3 \in K$ be any primitive third root of unity. \\
  Let $\eta \in \cc{O}_K$ be the element corresponding to $\zeta_3 \in K$. \\
  Let $\lambda \in \cc{O}_K$ be such that $\lambda = \eta -1$. \\
  Let $S=(a, b, c, u)$ be a $solution$.\\
  Let $p \in \cc{O}_K$ be a prime such that $p \divides a+b$
  and $p \divides a+\eta  b$.\\\\
  Then $p$ is associated with $\lambda$.
\end{lemma}
\begin{proof}
  \leanok
  \uses{lmm:lambda_prime}
  We proceed by analysis each case:
  \begin{itemize}
      \item Case $p \divides \lambda$. It directly follows from \Cref{lmm:lambda_prime}.
      \item Case $p \notdivides \lambda$. \\
            By hypothesis, we have that $p \divides a+b$ and $p \divides a+\eta b$.
            Then $p \divides (a+\eta b) - (a+b) = b (\eta-1) = b \lambda$, which implies that
            $p \divides b$ and we proceed analogously to show that $p \divides a$.\\
            Therefore $p \divides \gcd(a,b)=1$ which is absurd.
  \end{itemize}
  Therefore, we can conclude that $p$ is associated with $\lambda$.
\end{proof}

\begin{lemma}
  \label{lmm:associated_of_dvd_a_add_b_of_dvd_a_add_eta_sq_mul_b}
  \lean{Solution.associated_of_dvd_a_add_b_of_dvd_a_add_eta_sq_mul_b}
  \leanok
  \uses{def:Solution}
  Let $K = \Q(\zeta_3)$ be the third cyclotomic field. \\
  Let $\cc{O}_K = \Z[\zeta_3]$ be the ring of integers of $K$. \\
  Let $\cc{O}^\times_K$ be the group of units of $\cc{O}_K$. \\
  Let $\zeta_3 \in K$ be any primitive third root of unity. \\
  Let $\eta \in \cc{O}_K$ be the element corresponding to $\zeta_3 \in K$. \\
  Let $\lambda \in \cc{O}_K$ be such that $\lambda = \eta -1$. \\
  Let $S=(a, b, c, u)$ be a $solution$.\\
  Let $p \in \cc{O}_K$ be a prime such that $p \divides a+b$
  and $p \divides a+\eta^2  b$.\\\\
  Then $p$ is associated with $\lambda$.
\end{lemma}
\begin{proof}
  \leanok
  \uses{lmm:lambda_prime, lmm:toInteger_cube_eq_one, lmm:eta_isUnit}
  We proceed by analysis each case:
  \begin{itemize}
      \item Case $p \divides \lambda$. It directly follows from \Cref{lmm:lambda_prime}.
      \item Case $p \notdivides \lambda$. \\
            By hypothesis, we have that $p \divides a+ b$ and $p \divides a+\eta^2 b$.
            By \Cref{lmm:toInteger_cube_eq_one} and \Cref{lmm:eta_isUnit}, we have that
            $$p \divides \eta ((a+\eta^2 b) - (a+ b)) = - (\eta^3 - \eta) b = \lambda b,$$
            which implies that $p \divides b$
            and we proceed analogously to show that $p \divides a$.\\
            Therefore $p \divides \gcd(a,b)=1$ which is absurd.
  \end{itemize}
  Therefore, we can conclude that $p$ is associated with $\lambda$.
\end{proof}

\begin{lemma}
  \label{lmm:associated_of_dvd_a_add_eta_mul_b_of_dvd_a_add_eta_sq_mul_b}
  \lean{Solution.associated_of_dvd_a_add_eta_mul_b_of_dvd_a_add_eta_sq_mul_b}
  \leanok
  \uses{def:Solution}
  Let $K = \Q(\zeta_3)$ be the third cyclotomic field. \\
  Let $\cc{O}_K = \Z[\zeta_3]$ be the ring of integers of $K$. \\
  Let $\cc{O}^\times_K$ be the group of units of $\cc{O}_K$. \\
  Let $\zeta_3 \in K$ be any primitive third root of unity. \\
  Let $\eta \in \cc{O}_K$ be the element corresponding to $\zeta_3 \in K$. \\
  Let $\lambda \in \cc{O}_K$ be such that $\lambda = \eta -1$. \\
  Let $S=(a, b, c, u)$ be a $solution$.\\
  Let $p \in \cc{O}_K$ be a prime such that $p \divides a+\eta b$
  and $p \divides a+\eta^2  b$.\\\\
  Then $p$ is associated with $\lambda$.
\end{lemma}
\begin{proof}
  \leanok
  \uses{lmm:lambda_prime, lmm:eta_isUnit}
  We proceed by analysis each case:
  \begin{itemize}
      \item Case $p \divides \lambda$. It directly follows from \Cref{lmm:lambda_prime}.
      \item Case $p \notdivides \lambda$. \\
            By hypothesis, we have that $p \divides a+\eta b$ and $p \divides a+\eta^2 b$.
            Then $p \divides (a+\eta^2 b) - (a+\eta b) = \eta b (\eta-1) = \eta b \lambda$,
            which, by \Cref{lmm:eta_isUnit}, implies that $p \divides b$
            and we proceed analogously to show that $p \divides a$.\\
            Therefore $p \divides \gcd(a,b)=1$ which is absurd.
  \end{itemize}
  Therefore, we can conclude that $p$ is associated with $\lambda$.
\end{proof}

\begin{definition}[$x,y,z,w$]
  \label{def:Solution_x_y_z_w}
  %\lean{}
  \leanok
  \uses{def:Solution}
  Let $K = \Q(\zeta_3)$ be the third cyclotomic field. \\
  Let $\cc{O}_K = \Z[\zeta_3]$ be the ring of integers of $K$. \\
  Let $\cc{O}^\times_K$ be the group of units of $\cc{O}_K$. \\
  Let $\zeta_3 \in K$ be any primitive third root of unity. \\
  Let $\eta \in \cc{O}_K$ be the element corresponding to $\zeta_3 \in K$. \\
  Let $\lambda \in \cc{O}_K$ be such that $\lambda = \eta -1$. \\
  Let $S=(a, b, c, u)$ be a $solution$.\\\\
  We define $x \in \cc{O}_K$ such that $a + b = \lambda^{3n-2}  x$.\\
  We define $y \in \cc{O}_K$ such that $a + \eta  b = \lambda  y$.\\
  We define $z \in \cc{O}_K$ such that $a + \eta^2  b = \lambda  z$.\\
  We define $w \in \cc{O}_K$ such that $c = \lambda^n  w$.
\end{definition}

\begin{lemma}
  \label{lmm:lambda_not_dvd_y}
  \lean{Solution.lambda_not_dvd_y}
  \leanok
  \uses{def:Solution}
  Let $K = \Q(\zeta_3)$ be the third cyclotomic field. \\
  Let $\cc{O}_K = \Z[\zeta_3]$ be the ring of integers of $K$. \\
  Let $\cc{O}^\times_K$ be the group of units of $\cc{O}_K$. \\
  Let $\zeta_3 \in K$ be any primitive third root of unity. \\
  Let $\eta \in \cc{O}_K$ be the element corresponding to $\zeta_3 \in K$. \\
  Let $\lambda \in \cc{O}_K$ be such that $\lambda = \eta -1$. \\
  Let $S$ be a $solution$.\\\\
  Then $\lambda \notdivides y$.
\end{lemma}
\begin{proof}
  \leanok
  \uses{lmm:lambda_sq_not_dvd_a_add_eta_mul_b}
  By contradiction we assume that $\lambda \divides y$, which implies that
  $\lambda^2 \divides \lambda y = a + \eta b$, that contradicts
  \Cref{lmm:lambda_sq_not_dvd_a_add_eta_mul_b} forcing us to conclude that
  $\lambda \notdivides y$.
\end{proof}

\begin{lemma}
  \label{lmm:lambda_not_dvd_z}
  \lean{Solution.lambda_not_dvd_z}
  \leanok
  \uses{def:Solution}
  Let $K = \Q(\zeta_3)$ be the third cyclotomic field. \\
  Let $\cc{O}_K = \Z[\zeta_3]$ be the ring of integers of $K$. \\
  Let $\cc{O}^\times_K$ be the group of units of $\cc{O}_K$. \\
  Let $\zeta_3 \in K$ be any primitive third root of unity. \\
  Let $\eta \in \cc{O}_K$ be the element corresponding to $\zeta_3 \in K$. \\
  Let $\lambda \in \cc{O}_K$ be such that $\lambda = \eta -1$. \\
  Let $S$ be a $solution$.\\\\
  Then $\lambda \notdivides z$.
\end{lemma}
\begin{proof}
  \leanok
  \uses{lmm:lambda_sq_not_dvd_a_add_eta_sq_mul_b}
  By contradiction we assume that $\lambda \divides z$, which implies that
  $\lambda^2 \divides \lambda z = a + \eta^2 b$, that contradicts
  \Cref{lmm:lambda_sq_not_dvd_a_add_eta_sq_mul_b} forcing us to conclude $\lambda \notdivides z$.
\end{proof}

\begin{lemma}
  \label{lmm:lambda_pow_dvd_a_add_b}
  \lean{Solution.lambda_pow_dvd_a_add_b}
  \leanok
  \uses{def:Solution}
  Let $K = \Q(\zeta_3)$ be the third cyclotomic field. \\
  Let $\cc{O}_K = \Z[\zeta_3]$ be the ring of integers of $K$. \\
  Let $\cc{O}^\times_K$ be the group of units of $\cc{O}_K$. \\
  Let $\zeta_3 \in K$ be any primitive third root of unity. \\
  Let $\eta \in \cc{O}_K$ be the element corresponding to $\zeta_3 \in K$. \\
  Let $\lambda \in \cc{O}_K$ be such that $\lambda = \eta -1$. \\
  Let $S=(a, b, c, u)$ be a $solution$ with multiplicity $n$.\\\\
  Then $\lambda^{3n -2} \divides a + b$.
\end{lemma}
\begin{proof}
  \leanok
  \uses{lmm:cube_add_cube_eq_mul, lmm:lambda_prime,
  lmm:lambda_not_dvd_z, lmm:lambda_not_dvd_y, lmm:Solution_two_le_multiplicity,
  lmm:lambda_ne_zero}
  By \Cref{def:Solution_Multiplicity} we have that $\lambda^n \divides c$.
  Since $u$ is a unit, then by \Cref{lmm:cube_add_cube_eq_mul} we have that
  $$\lambda^{3n} \divides u  c^3 = a^3 + b^3 = (a+b)(a + \eta b)(a + \eta^2 b)
  = (a+b)(\lambda y)(\lambda z).$$
  Then applying \Cref{lmm:lambda_not_dvd_y} and \Cref{lmm:lambda_not_dvd_z}, we can conclude
  that $\lambda^{3n-2} \divides a+b$.
\end{proof}

\begin{lemma}
  \label{lmm:lambda_not_dvd_w}
  \lean{Solution.lambda_not_dvd_w}
  \leanok
  \uses{def:Solution}
  Let $K = \Q(\zeta_3)$ be the third cyclotomic field. \\
  Let $\cc{O}_K = \Z[\zeta_3]$ be the ring of integers of $K$. \\
  Let $\cc{O}^\times_K$ be the group of units of $\cc{O}_K$. \\
  Let $\zeta_3 \in K$ be any primitive third root of unity. \\
  Let $\eta \in \cc{O}_K$ be the element corresponding to $\zeta_3 \in K$. \\
  Let $\lambda \in \cc{O}_K$ be such that $\lambda = \eta -1$. \\
  Let $S$ be a $solution$.\\\\
  Then $\lambda \notdivides w$.
\end{lemma}
\begin{proof}
  \leanok
  \uses{lmm:multiplicity_lambda_c_finite}
  By contradiction we assume that $\lambda \divides w$, which implies
  $\lambda^{n+1} \divides \lambda^n  w = c$ that contradicts \Cref{def:Solution_Multiplicity}
  forcing us to conclude that $\lambda \notdivides w$.
\end{proof}

\begin{lemma}
  \label{lmm:lambda_not_dvd_x}
  \lean{Solution.lambda_not_dvd_x}
  \leanok
  \uses{def:Solution}
  Let $K = \Q(\zeta_3)$ be the third cyclotomic field. \\
  Let $\cc{O}_K = \Z[\zeta_3]$ be the ring of integers of $K$. \\
  Let $\cc{O}^\times_K$ be the group of units of $\cc{O}_K$. \\
  Let $\zeta_3 \in K$ be any primitive third root of unity. \\
  Let $\eta \in \cc{O}_K$ be the element corresponding to $\zeta_3 \in K$. \\
  Let $\lambda \in \cc{O}_K$ be such that $\lambda = \eta -1$. \\
  Let $S$ be a $solution$.\\\\
  Then $\lambda \notdivides x$.
\end{lemma}
\begin{proof}
  \leanok
  \uses{lmm:lambda_dvd_a_add_eta_mul_b, lmm:lambda_dvd_a_add_eta_sq_mul_b,
  lmm:cube_add_cube_eq_mul, lmm:Solution_two_le_multiplicity, lmm:lambda_prime,
  lmm:lambda_not_dvd_w, lmm:lambda_ne_zero}
  By contradiction, if $\lambda \divides x$, then
  $\lambda^{3n-1} \divides \lambda^{3n-2}  x = a+b$. Using \Cref{lmm:lambda_dvd_a_add_eta_mul_b}
  and \Cref{lmm:lambda_dvd_a_add_eta_sq_mul_b}, we have that $\lambda^{3n+1} \divides
  (a+b)  (a + \eta  b)  (a + \eta^2 cdot b) = a^3+b^3
  = u c^3 = u \lambda^{3n} w^3$.
  Then $\lambda \divides w^3$ which implies that $\lambda \divides w$, that
  contradicts \Cref{lmm:lambda_not_dvd_w} forcing us to conclude $\lambda \notdivides x$.
\end{proof}

\begin{lemma}
  \label{lmm:coprime_x_y}
  \lean{Solution.coprime_x_y}
  \leanok
  \uses{def:Solution, def:Solution_x_y_z_w}
  Let $S$ be a $solution$ with multiplicity $n$.\\\\
  Then $\gcd(x,y) = 1$.
\end{lemma}
\begin{proof}
  \leanok
  \uses{lmm:lambda_not_dvd_y, lmm:associated_of_dvd_a_add_b_of_dvd_a_add_eta_mul_b,
  lmm:lambda_not_dvd_x}
  Since $y \neq 0$ by \Cref{lmm:lambda_not_dvd_y}, by the properties of PIDs it suffices to prove that
  $\forall p \in \cc{O}_K$ if $p$ is prime and $p \divides x$, then $p \notdivides y$.
  Let $p \in \cc{O}_K$ be prime and suppose by contradiction that $p \divides x$ and $p \divides y$
  which implies that $p \divides \lambda^{3n-2} x = a+b$ and $p \divides \lambda y = a + \eta b$.
  Then by \Cref{lmm:associated_of_dvd_a_add_b_of_dvd_a_add_eta_mul_b}
  we have that $p$ is associated with $\lambda$, which implies that $\lambda \divides x$
  that contradicts \Cref{lmm:lambda_not_dvd_x} forcing us to conclude that $p \notdivides y$, which,
  as stated above, implies that $\gcd(x,y)=1$.
\end{proof}

\begin{lemma}
  \label{lmm:coprime_x_z}
  \lean{Solution.coprime_x_z}
  \leanok
  \uses{def:Solution, def:Solution_x_y_z_w}
  Let $S$ be a $solution$.\\\\
  Then $\gcd(x,z) = 1$.
\end{lemma}
\begin{proof}
  \leanok
  \uses{lmm:lambda_not_dvd_z, lmm:associated_of_dvd_a_add_b_of_dvd_a_add_eta_sq_mul_b,
  lmm:lambda_not_dvd_x}
  Since $z \neq 0$ by \Cref{lmm:lambda_not_dvd_z}, by the properties of PIDs it suffices to prove that
  $\forall p \in \cc{O}_K$ if $p$ is prime and $p \divides x$, then $p \notdivides z$.
  Let $p \in \cc{O}_K$ be prime and suppose by contradiction that $p \divides x$ and $p \divides z$
  which implies that $p \divides \lambda^{3n-2} x = a+b$ and $p \divides \lambda z = a + \eta^2 b$.
  Then by \Cref{lmm:associated_of_dvd_a_add_b_of_dvd_a_add_eta_sq_mul_b}
  we have that $p$ is associated with $\lambda$, which implies that $\lambda \divides x$
  that contradicts \Cref{lmm:lambda_not_dvd_x} forcing us to conclude that $p \notdivides z$, which,
  as stated above, implies that $\gcd(x,z)=1$.
\end{proof}

\begin{lemma}
  \label{lmm:coprime_y_z}
  \lean{Solution.coprime_y_z}
  \leanok
  \uses{def:Solution, def:Solution_x_y_z_w}
  Let $S$ be a $solution$.\\\\
  Then $\gcd(y, z) = 1$.
\end{lemma}
\begin{proof}
  \leanok
  \uses{lmm:lambda_not_dvd_z, lmm:associated_of_dvd_a_add_eta_mul_b_of_dvd_a_add_eta_sq_mul_b,
  lmm:lambda_not_dvd_y}
  Since $z \neq 0$ by \Cref{lmm:lambda_not_dvd_z}, by the properties of PIDs it suffices to prove that
  $\forall p \in \cc{O}_K$ if $p$ is prime and $p \divides y$, then $p \notdivides z$.
  Let $p \in \cc{O}_K$ be prime and suppose by contradiction that $p \divides y$ and $p \divides z$
  which implies that $p \divides \lambda y = a+\eta b$ and $p \divides \lambda z = a + \eta^2 b$.
  Then by \Cref{lmm:associated_of_dvd_a_add_eta_mul_b_of_dvd_a_add_eta_sq_mul_b}
  we have that $p$ is associated with $\lambda$, which implies that $\lambda \divides y$
  that contradicts \Cref{lmm:lambda_not_dvd_y} forcing us to conclude that $p \notdivides z$, which,
  as stated above, implies that $\gcd(y,z)=1$.
\end{proof}

\begin{lemma}
  \label{lmm:mult_minus_two_plus_one_plus_one}
  \lean{Solution.mult_minus_two_plus_one_plus_one}
  \leanok
  \uses{def:Solution}
  Let $S$ be a $solution$ with multiplicity $n$.\\\\
  Then $3n - 2 + 1 + 1 = 3n$.
\end{lemma}
\begin{proof}
  \leanok
  \uses{lmm:Solution_two_le_multiplicity}
  It directly follows from \Cref{lmm:Solution_two_le_multiplicity}
  and calculations using ring properties.
\end{proof}

\begin{lemma}
  \label{lmm:x_mul_y_mul_z_eq_u_w_pow_three}
  \lean{Solution.x_mul_y_mul_z_eq_u_w_pow_three}
  \leanok
  \uses{def:Solution, def:Solution_x_y_z_w}
  Let $S=(a,b,c,u)$ be a $solution$.\\\\
  Then $x y z = u w^3$.
\end{lemma}
\begin{proof}
  \leanok
  \uses{lmm:Solution_two_le_multiplicity, lmm:lambda_ne_zero, lmm:cube_add_cube_eq_mul,
  def:Solution_x_y_z_w}
  It directly follows from \Cref{def:Solution_x_y_z_w}, \Cref{lmm:cube_add_cube_eq_mul},
  \Cref{lmm:lambda_ne_zero}, \Cref{lmm:Solution_two_le_multiplicity} and
  calculations using ring properties.
\end{proof}

\begin{lemma}
  \label{lmm:x_eq_unit_mul_cube}
  \lean{Solution.x_eq_unit_mul_cube}
  \leanok
  \uses{def:Solution}
  Let $S$ be a $solution$.\\\\
  Then $\exists u_1 \in \cc{O}^\times_K$ and $\exists X \in \cc{O}_K$
  such that $x = u_1 X^3$.
\end{lemma}
\begin{proof}
  \leanok
  \uses{lmm:x_mul_y_mul_z_eq_u_w_pow_three, lmm:coprime_x_y, lmm:coprime_x_z}
  By the properties of PIDs, it suffices to prove that there exists a $k\in \cc{O}_K$ such that
  $xk$ is a cube and $\gcd(x,k)=1$.
  Let $k = yzu^{-1}$, then $xk = x y z u^{-1} = w^3$ by \Cref{lmm:x_mul_y_mul_z_eq_u_w_pow_three}.
  Moreover, since $\gcd(x,y)=1$ by \Cref{lmm:coprime_x_y} and $\gcd(x,z)=1$ by \Cref{lmm:coprime_x_z},
  then $\gcd(x,yz)=1$, which implies that $\gcd(x,k)=1$.
\end{proof}

\begin{lemma}
  \label{lmm:y_eq_unit_mul_cube}
  \lean{Solution.y_eq_unit_mul_cube}
  \leanok
  \uses{def:Solution}
  Let $S$ be a $solution$.\\\\
  Then $\exists u_2 \in \cc{O}^\times_K$ and $\exists Y \in \cc{O}_K$
  such that $y = u_2 Y^3$.
\end{lemma}
\begin{proof}
  \leanok
  \uses{lmm:x_mul_y_mul_z_eq_u_w_pow_three, lmm:coprime_x_y, lmm:coprime_y_z}
  By the properties of PIDs, it suffices to prove that there exists a $k\in \cc{O}_K$ such that
  $yk$ is a cube and $\gcd(y,k)=1$.
  Let $k = xzu^{-1}$, then $yk = y x z u^{-1} = w^3$ by \Cref{lmm:x_mul_y_mul_z_eq_u_w_pow_three}.
  Moreover, since $\gcd(x,y)=1$ by \Cref{lmm:coprime_x_y} and $\gcd(y,z)=1$ by \Cref{lmm:coprime_y_z},
  then $\gcd(y,xz)=1$, which implies that $\gcd(y,k)=1$.
\end{proof}

\begin{lemma}
  \label{lmm:z_eq_unit_mul_cube}
  \lean{Solution.z_eq_unit_mul_cube}
  \leanok
  \uses{def:Solution}
  Let $S$ be a $solution$.\\\\
  Then $\exists u_3 \in \cc{O}^\times_K$ and $\exists Z \in \cc{O}_K$
  such that $z = u_3  Z^3$.
\end{lemma}
\begin{proof}
  \leanok
  \uses{lmm:x_mul_y_mul_z_eq_u_w_pow_three, lmm:coprime_x_z, lmm:coprime_y_z}
  By the properties of PIDs, it suffices to prove that there exists a $k\in \cc{O}_K$ such that
  $zk$ is a cube and $\gcd(z,k)=1$.
  Let $k = xyu^{-1}$, then $zk = z x y u^{-1} = w^3$ by \Cref{lmm:x_mul_y_mul_z_eq_u_w_pow_three}.
  Moreover, since $\gcd(x,z)=1$ by \Cref{lmm:coprime_x_z} and $\gcd(y,z)=1$ by \Cref{lmm:coprime_y_z},
  then $\gcd(z,xy)=1$, which implies that $\gcd(z,k)=1$.
\end{proof}

\begin{definition}[$u_1,u_2,u_3,u_4,u_5,X,Y,Z$]
  \label{def:Solution_u1_u2_u3_u4_u5_X_Y_Z}
  %\lean{}
  \leanok
  \uses{def:Solution, lmm:x_eq_unit_mul_cube,
  lmm:y_eq_unit_mul_cube, lmm:z_eq_unit_mul_cube}
  Let $S$ be a $solution$.\\\\
  We define $u_1 \in \cc{O}^\times_K$ and $X \in \cc{O}_K$
  such that $x = u_1 X^3$.\\
  We define $u_2 \in \cc{O}^\times_K$ and $Y \in \cc{O}_K$
  such that $y = u_2 Y^3$.\\
  We define $u_3 \in \cc{O}^\times_K$ and $Z \in \cc{O}_K$
  such that $z = u_3 Z^3$.\\
  We define $u_4 = \eta u_3 u_2^{-1}$.\\
  We define $u_5 = -\eta^2 u_1 u_2^{-1}$.\\
\end{definition}

\begin{lemma}
  \label{lmm:X_ne_zero}
  \lean{Solution.X_ne_zero}
  \leanok
  \uses{def:Solution, def:Solution_u1_u2_u3_u4_u5_X_Y_Z}
  Let $S$ be a $solution$.\\\\
  Then $X \neq 0$.
\end{lemma}
\begin{proof}
  \leanok
  \uses{def:Solution_u1_u2_u3_u4_u5_X_Y_Z, lmm:lambda_not_dvd_x}
  By contradiction we assume that $X = 0$, then $x = 0$ by \Cref{def:Solution_u1_u2_u3_u4_u5_X_Y_Z}.
  Therefore $\lambda$ trivially divides $x$ (as any number divides zero) which contradicts
  \Cref{lmm:lambda_not_dvd_x} forcing us to conclude that $X \neq 0$.
\end{proof}

\begin{lemma}
  \label{lmm:lambda_not_dvd_X}
  \lean{Solution.lambda_not_dvd_X}
  \leanok
  \uses{def:Solution, def:Solution_u1_u2_u3_u4_u5_X_Y_Z}
  Let $K = \Q(\zeta_3)$ be the third cyclotomic field. \\
  Let $\cc{O}_K = \Z[\zeta_3]$ be the ring of integers of $K$. \\
  Let $\cc{O}^\times_K$ be the group of units of $\cc{O}_K$. \\
  Let $\zeta_3 \in K$ be any primitive third root of unity. \\
  Let $\eta \in \cc{O}_K$ be the element corresponding to $\zeta_3 \in K$. \\
  Let $\lambda \in \cc{O}_K$ be such that $\lambda = \eta -1$. \\
  Let $S$ be a $solution$.\\\\
  Then $\lambda \notdivides X$.
\end{lemma}
\begin{proof}
  \leanok
  \uses{def:Solution_u1_u2_u3_u4_u5_X_Y_Z,, lmm:lambda_not_dvd_x}
  By contradiction we assume that $\lambda \divides X$, then, by the properties of divisibility,
  $\lambda \divides u_1 X^3$, which implies, by \Cref{def:Solution_u1_u2_u3_u4_u5_X_Y_Z},
  that $\lambda \divides x$.
  However, this contradicts \Cref{lmm:lambda_not_dvd_x}
  forcing us to conclude that $\lambda \notdivides X$.
\end{proof}

\begin{lemma}
  \label{lmm:lambda_not_dvd_Y}
  \lean{Solution.lambda_not_dvd_Y}
  \leanok
  \uses{def:Solution, def:Solution_u1_u2_u3_u4_u5_X_Y_Z}
  Let $K = \Q(\zeta_3)$ be the third cyclotomic field. \\
  Let $\cc{O}_K = \Z[\zeta_3]$ be the ring of integers of $K$. \\
  Let $\cc{O}^\times_K$ be the group of units of $\cc{O}_K$. \\
  Let $\zeta_3 \in K$ be any primitive third root of unity. \\
  Let $\eta \in \cc{O}_K$ be the element corresponding to $\zeta_3 \in K$. \\
  Let $\lambda \in \cc{O}_K$ be such that $\lambda = \eta -1$. \\
  Let $S$ be a $solution$.\\\\
  Then $\lambda \notdivides Y$.
\end{lemma}
\begin{proof}
  \leanok
  \uses{lmm:lambda_not_dvd_y}
  By contradiction we assume that $\lambda \divides Y$, then, by the properties of divisibility,
  $\lambda \divides u_2 Y^3$, which implies, by \Cref{def:Solution_u1_u2_u3_u4_u5_X_Y_Z},
  that $\lambda \divides y$.
  However, this contradicts \Cref{lmm:lambda_not_dvd_y}
  forcing us to conclude that $\lambda \notdivides Y$.
\end{proof}

\begin{lemma}
  \label{lmm:lambda_not_dvd_Z}
  \lean{Solution.lambda_not_dvd_Z}
  \leanok
  \uses{def:Solution, def:Solution_u1_u2_u3_u4_u5_X_Y_Z}
  Let $K = \Q(\zeta_3)$ be the third cyclotomic field. \\
  Let $\cc{O}_K = \Z[\zeta_3]$ be the ring of integers of $K$. \\
  Let $\cc{O}^\times_K$ be the group of units of $\cc{O}_K$. \\
  Let $\zeta_3 \in K$ be any primitive third root of unity. \\
  Let $\eta \in \cc{O}_K$ be the element corresponding to $\zeta_3 \in K$. \\
  Let $\lambda \in \cc{O}_K$ be such that $\lambda = \eta -1$. \\
  Let $S$ be a $solution$.\\\\
  Then $\lambda \notdivides Z$.
\end{lemma}
\begin{proof}
  \leanok
  \uses{lmm:lambda_not_dvd_z}
  By contradiction we assume that $\lambda \divides Z$, then, by the properties of divisibility,
  $\lambda \divides u_3 Z^3$, which implies, by \Cref{def:Solution_u1_u2_u3_u4_u5_X_Y_Z},
  that $\lambda \divides z$.
  However, this contradicts \Cref{lmm:lambda_not_dvd_z}
  forcing us to conclude that $\lambda \notdivides Z$.
\end{proof}

\begin{lemma}
  \label{lmm:coprime_Y_Z}
  \lean{Solution.coprime_Y_Z}
  \leanok
  \uses{def:Solution, def:Solution_u1_u2_u3_u4_u5_X_Y_Z}
  Let $S$ be a $solution$.\\\\
  Then $\gcd(Y, Z) = 1$.
\end{lemma}
\begin{proof}
  \leanok
  \uses{lmm:lambda_not_dvd_Z, lmm:coprime_y_z}
  Since $Z \neq 0$ by \Cref{lmm:lambda_not_dvd_Z}, by the properties of PIDs it suffices to prove that
  $\forall p \in \cc{O}_K$ if $p$ is prime and $p \divides Y$, then $p \notdivides Z$.
  Let $p \in \cc{O}_K$ be prime and suppose by contradiction that $p \divides Y$ and $p \divides Z$
  which implies that $p \divides u_2 Y^3 = y$ and $p \divides \lambda u_3 Z^3 = z$.
  But this contradicts \Cref{lmm:coprime_y_z} forcing us to conclude that $p \notdivides Z$, which,
  as stated above, implies that $\gcd(Y,Z)=1$.
\end{proof}

\begin{lemma}
  \label{lmm:formula1}
  \lean{Solution.formula1}
  \leanok
  \uses{def:Solution, def:Solution_u1_u2_u3_u4_u5_X_Y_Z}
  Let $K = \Q(\zeta_3)$ be the third cyclotomic field. \\
  Let $\cc{O}_K = \Z[\zeta_3]$ be the ring of integers of $K$. \\
  Let $\cc{O}^\times_K$ be the group of units of $\cc{O}_K$. \\
  Let $\zeta_3 \in K$ be any primitive third root of unity. \\
  Let $\eta \in \cc{O}_K$ be the element corresponding to $\zeta_3 \in K$. \\
  Let $\lambda \in \cc{O}_K$ be such that $\lambda = \eta -1$. \\
  Let $S$ be a $solution$ with multiplicity $n$.\\\\
  Then $u_1 X^3 \lambda^{3n-2}+u_2 \eta Y^3 \lambda +
  u_3 \eta^2 Z^3 \lambda = 0$.
\end{lemma}
\begin{proof}
  \leanok
  \uses{def:Solution_u1_u2_u3_u4_u5_X_Y_Z, def:Solution_x_y_z_w,
  lmm:toInteger_cube_eq_one, lmm:toInteger_eval_cyclo}
  Applying \Cref{def:Solution_u1_u2_u3_u4_u5_X_Y_Z}, \Cref{def:Solution_x_y_z_w},
  \Cref{lmm:toInteger_cube_eq_one} and \Cref{lmm:toInteger_eval_cyclo}, we have
  \begin{align*}
      u_1 X^3 \lambda^{3n-2}+u_2 \eta Y^3 \lambda + u_3 \eta^2 Z^3 \lambda
      &= x \lambda^{3n-2} + \eta y \lambda + \eta^2 z \lambda \\
      &= (a+b) + \eta (a+\eta b) + \eta^2 (a+\eta^2 b) \\
      &= a (1 + \eta + \eta^2) + b (1 + \eta^4 + \eta^2) \\
      &= (a+b)(1+\eta+\eta^2)\\
      &= (a+b)0 = 0
  \end{align*}
\end{proof}

\begin{lemma}
  \label{lmm:u₄_isUnit}
  \lean{Solution.u₄'_isUnit}
  \leanok
  \uses{def:Solution, def:Solution_u1_u2_u3_u4_u5_X_Y_Z}
  Let $S$ be a $solution$.\\\\
  Then $u_4$ is a unit.
\end{lemma}
\begin{proof}
  \leanok
  \uses{def:Solution_u1_u2_u3_u4_u5_X_Y_Z, lmm:eta_isUnit}
  By \Cref{def:Solution_u1_u2_u3_u4_u5_X_Y_Z} $u_4 = \eta u_3 u_2^{-1}$,
  which is a product of units by \Cref{lmm:eta_isUnit}.
  Since the product of units is a unit (multiplicative closure),
  it follows that $u_4$ must also be a unit.
\end{proof}

\begin{lemma}
  \label{lmm:u₅_isUnit}
  \lean{Solution.u₅'_isUnit}
  \leanok
  \uses{def:Solution, def:Solution_u1_u2_u3_u4_u5_X_Y_Z}
  Let $S$ be a $solution$.\\\\
  Then $u_5$ is a unit.
\end{lemma}
\begin{proof}
  \leanok
  \uses{lmm:toInteger_cube_eq_one}
  By \Cref{def:Solution_u1_u2_u3_u4_u5_X_Y_Z} $u_5 = -\eta^2 u_1 u_2^{-1}$,
  which is a product of units since $\eta^3 = 1$ by \Cref{lmm:toInteger_cube_eq_one} and
  $-\eta (-\eta^2) = \eta^3$.
  Since the product of units is a unit (multiplicative closure),
  it follows that $u_5$ must also be a unit.
\end{proof}

\begin{lemma}
  \label{lmm:formula2}
  \lean{Solution.formula2}
  \leanok
  \uses{def:Solution, def:Solution_u1_u2_u3_u4_u5_X_Y_Z}
  Let $K = \Q(\zeta_3)$ be the third cyclotomic field. \\
  Let $\cc{O}_K = \Z[\zeta_3]$ be the ring of integers of $K$. \\
  Let $\cc{O}^\times_K$ be the group of units of $\cc{O}_K$. \\
  Let $\zeta_3 \in K$ be any primitive third root of unity. \\
  Let $\eta \in \cc{O}_K$ be the element corresponding to $\zeta_3 \in K$. \\
  Let $\lambda \in \cc{O}_K$ be such that $\lambda = \eta -1$. \\
  Let $S$ be a $solution$ with multiplicity $n$.\\\\
  Then $Y^3 + u_4 Z^3 = u_5 (\lambda^(n-1) X)^3$.
\end{lemma}
\begin{proof}
  \leanok
  \uses{lmm:eta_isUnit, lmm:lambda_ne_zero, lmm:toInteger_cube_eq_one,
  lmm:Solution_two_le_multiplicity, lmm:formula1}
  Using \Cref{lmm:eta_isUnit}, \Cref{lmm:lambda_ne_zero}, it suffices to show that
  $$\lambda \eta u_2 (Y^3 + u_4 Z^3) = \lambda \eta u_2 u_5 (\lambda^(n-1) X)^3$$
  which can be proved by simple calculations involving \Cref{lmm:toInteger_cube_eq_one},
  \Cref{lmm:Solution_two_le_multiplicity} and \Cref{lmm:formula1}.
\end{proof}

\begin{lemma}
  \label{lmm:lambda_sq_div_lambda_fourth}
  \lean{Solution.lambda_sq_div_lambda_fourth}
  \leanok
  \uses{def:Solution}
  Let $K = \Q(\zeta_3)$ be the third cyclotomic field. \\
  Let $\cc{O}_K = \Z[\zeta_3]$ be the ring of integers of $K$. \\
  Let $\cc{O}^\times_K$ be the group of units of $\cc{O}_K$. \\
  Let $\zeta_3 \in K$ be any primitive third root of unity. \\
  Let $\eta \in \cc{O}_K$ be the element corresponding to $\zeta_3 \in K$. \\
  Let $\lambda \in \cc{O}_K$ be such that $\lambda = \eta -1$. \\
  Let $S$ be a $solution$.\\\\
  Then $\lambda^2 \divides \lambda^4$.
\end{lemma}
\begin{proof}
  \leanok
  Straightforward application of the definition of divisibility.
\end{proof}

\begin{lemma}
  \label{lmm:lambda_sq_div_new_X_cubed}
  \lean{Solution.lambda_sq_div_new_X_cubed}
  \leanok
  \uses{def:Solution, def:Solution_u1_u2_u3_u4_u5_X_Y_Z}
  Let $K = \Q(\zeta_3)$ be the third cyclotomic field. \\
  Let $\cc{O}_K = \Z[\zeta_3]$ be the ring of integers of $K$. \\
  Let $\cc{O}^\times_K$ be the group of units of $\cc{O}_K$. \\
  Let $\zeta_3 \in K$ be any primitive third root of unity. \\
  Let $\eta \in \cc{O}_K$ be the element corresponding to $\zeta_3 \in K$. \\
  Let $\lambda \in \cc{O}_K$ be such that $\lambda = \eta -1$. \\
  Let $S$ be a $solution$ with multiplicity $n$.\\\\
  Then $\lambda^2 \divides u_5 (\lambda^{n - 1} X)^3$.
\end{lemma}
\begin{proof}
  \leanok
  \uses{lmm:Solution_two_le_multiplicity}
  Using \Cref{lmm:Solution_two_le_multiplicity}, we have that $\lambda^2 \divides
  \lambda^2 u_5 \lambda^{3n-5} X^3 = u_5 (\lambda^{n - 1} X)^3$.
\end{proof}

\begin{lemma}
  \label{lmm:by_kummer}
  \lean{Solution.by_kummer}
  \leanok
  \uses{def:Solution, def:Solution_u1_u2_u3_u4_u5_X_Y_Z}
  Let $S$ be a $solution$.\\\\
  Then $u_4 \in \set{-1,1} \subset \cc{O}_K$.
\end{lemma}
\begin{proof}
  \leanok
  \uses{lmm:lambda_sq_div_lambda_fourth, lmm:lambda_sq_div_new_X_cubed,
  lmm:eq_one_or_neg_one_of_unit_of_congruent,
  lmm:lambda_pow_four_dvd_cube_sub_one_or_add_one_of_lambda_not_dvd,
  lmm:lambda_not_dvd_Z, lmm:lambda_not_dvd_Y, lmm:formula2}
  Let $n \in \N$ be the multiplicity of the solution $S$.\\
  By \Cref{lmm:eq_one_or_neg_one_of_unit_of_congruent}, it suffices to prove that
  $$\exists m \in \Z \text{ such that } \lambda^2 \divides u_4 - m.$$
  By \Cref{lmm:lambda_pow_four_dvd_cube_sub_one_or_add_one_of_lambda_not_dvd}
  and \Cref{lmm:lambda_not_dvd_Y}, we have that
  $$(\lambda^4 \divides Y^3 - 1) \lor (\lambda^4 \divides Y^3 + 1).$$
  By \Cref{lmm:lambda_pow_four_dvd_cube_sub_one_or_add_one_of_lambda_not_dvd}
  and \Cref{lmm:lambda_not_dvd_Z}, we have that
  $$(\lambda^4 \divides Z^3 - 1) \lor (\lambda^4 \divides Z^3 + 1).$$
  We proceed by analysing each case:
  \begin{itemize}
      \item Case $(\lambda^4 \divides Y^3 - 1) \land (\lambda^4 \divides Z^3 - 1)$. \\
            Let $m=-1$ and consider the fact that
            $$u_4 - m = Y^3 + u_4 Z^3 - (Y^3 - 1) - u_4 (Z^3 - 1).$$
            By \Cref{lmm:formula2}, we have that
            $$u_4 - m = u_5 (λ^{n-1} X)^3 - (Y^3 - 1) - u_4 (Z^3 - 1).$$
            Since, by \Cref{lmm:lambda_sq_div_new_X_cubed}, we know that
            $$\lambda^2 \divides u_5 (λ^{n-1} X)^3$$
            and, by \Cref{lmm:lambda_sq_div_lambda_fourth} and by assumption, we have that
            $$\lambda^2 \divides Y^3 - 1 \land \lambda^2 \divides Z^3 - 1,$$
            Then, we can conclude that
            $$\lambda^2 \divides u_4 - m.$$
      \item Case $(\lambda^4 \divides Y^3 - 1) \land (\lambda^4 \divides Z^3 + 1)$. \\
            Let $m=1$ and proceed similarly to the first case.
      \item Case $(\lambda^4 \divides Y^3 + 1) \land (\lambda^4 \divides Z^3 - 1)$. \\
            Let $m=1$ and proceed similarly to the first case.
      \item Case $(\lambda^4 \divides Y^3 + 1) \land (\lambda^4 \divides Z^3 + 1)$. \\
            Let $m=-1$ and proceed similarly to the first case.
  \end{itemize}
\end{proof}

\begin{lemma}
  \label{lmm:final}
  \lean{Solution.final}
  \leanok
  \uses{def:Solution, def:Solution_u1_u2_u3_u4_u5_X_Y_Z}
  Let $K = \Q(\zeta_3)$ be the third cyclotomic field. \\
  Let $\cc{O}_K = \Z[\zeta_3]$ be the ring of integers of $K$. \\
  Let $\cc{O}^\times_K$ be the group of units of $\cc{O}_K$. \\
  Let $\zeta_3 \in K$ be any primitive third root of unity. \\
  Let $\eta \in \cc{O}_K$ be the element corresponding to $\zeta_3 \in K$. \\
  Let $\lambda \in \cc{O}_K$ be such that $\lambda = \eta -1$. \\
  Let $S$ be a $solution$ with multiplicity $n$.\\\\
  Then $Y^3 + (u_4 Z)^3 = u_5 (\lambda^{n-1} X)^3$.
\end{lemma}
\begin{proof}
  \leanok
  \uses{lmm:formula2, lmm:by_kummer}
  By \Cref{lmm:by_kummer}, we have that $u_4 \in \set{-1,1}$, which implies that $u_4^2 = 1$.\\
  Therefore, by \Cref{lmm:formula2}, we can conclude that
  $$Y^3 + (u_4 Z)^3 = u_5 (\lambda^{n-1} X)^3.$$
\end{proof}

\begin{definition}[Final Solution']
  \label{def:Solution1_final}
  \lean{Solution'_final}
  \leanok
  \uses{def:Solution1, def:Solution_u1_u2_u3_u4_u5_X_Y_Z, lmm:Solution_two_le_multiplicity,
  lmm:final, lmm:coprime_Y_Z, lmm:lambda_not_dvd_Y, lmm:lambda_not_dvd_Z, lmm:lambda_ne_zero,
  lmm:X_ne_zero}
  Let $K = \Q(\zeta_3)$ be the third cyclotomic field. \\
  Let $\cc{O}_K = \Z[\zeta_3]$ be the ring of integers of $K$. \\
  Let $\cc{O}^\times_K$ be the group of units of $\cc{O}_K$. \\
  Let $\zeta_3 \in K$ be any primitive third root of unity. \\
  Let $\eta \in \cc{O}_K$ be the element corresponding to $\zeta_3 \in K$. \\
  Let $\lambda \in \cc{O}_K$ be such that $\lambda = \eta -1$. \\
  Let $S = (a,b,c,u)$ be a $solution$ with multiplicity $n$.\\
  Let $S_f' = (Y,u_4 Z, \lambda^{n-1} X, u_5)$.\\\\
  Then $S_f'$ is a $solution'$.
\end{definition}

\begin{lemma}
  \label{lmm:Solution1_final_multiplicity}
  \lean{Solution'_final_multiplicity}
  \leanok
  \uses{def:Solution, def:Solution1_final}
  Let $S$ be a $solution$ with multiplicity $n$.\\\\
  Then $S_f'$ has multiplicity $n-1$.
\end{lemma}
\begin{proof}
  \leanok
  \uses{lmm:lambda_not_dvd_X,
  lmm:lambda_ne_zero}
  Let $K = \Q(\zeta_3)$ be the third cyclotomic field. \\
  Let $\cc{O}_K = \Z[\zeta_3]$ be the ring of integers of $K$. \\
  Let $\cc{O}^\times_K$ be the group of units of $\cc{O}_K$. \\
  Let $\zeta_3 \in K$ be any primitive third root of unity. \\
  Let $\eta \in \cc{O}_K$ be the element corresponding to $\zeta_3 \in K$. \\
  Let $\lambda \in \cc{O}_K$ be such that $\lambda = \eta -1$. \\
  Let $(a',b',c',u') = S_f'$ be the final $solution'$, then
  $\lambda^{n-1} \divides \lambda^{n-1} X = c'$.
  By contradiction we assume that $\lambda^n \divides c'$ which implies that $\lambda \divides X$,
  that contradicts \Cref{lmm:lambda_not_dvd_X} forcing us to conclude
  that $\lambda^{n} \notdivides c'$. Then $S_f'$ has multiplicity $n-1$.
\end{proof}

\begin{lemma}
  \label{lmm:Solution1_final_multiplicity_lt}
  \lean{Solution'_final_multiplicity_lt}
  \leanok
  \uses{def:Solution, def:Solution1_final}
  Let $S$ be a $solution$ with multiplicity $n$.\\\\
  Then $S_f'$ has multiplicity $m<n$.
\end{lemma}
\begin{proof}
  \leanok
  \uses{lmm:Solution1_final_multiplicity, lmm:Solution_two_le_multiplicity}
  It directly follows from \Cref{lmm:Solution1_final_multiplicity} since $m = n-1 < n$.
\end{proof}

\begin{theorem}
  \label{lmm:exists_Solution_multiplicity_lt}
  \lean{Solution.exists_Solution_multiplicity_lt}
  \leanok
  \uses{def:Solution}
  Let $S$ be a $solution$ with multiplicity $n$.\\\\
  Then there is a $solution$ with multiplicity $m<n$.
\end{theorem}
\begin{proof}
  \leanok
  \uses{lmm:exists_Solution_of_Solution1, lmm:Solution1_final_multiplicity_lt}
  It directly follows from \Cref{lmm:Solution1_final_multiplicity} and
  \Cref{lmm:Solution1_final_multiplicity_lt}.
\end{proof}

\begin{theorem}[Generalised Fermat's Last Theorem for Exponent $3$]
  \label{thm:fermatLastTheoremForThreeGen}
  \lean{fermatLastTheoremForThreeGen}
  \leanok
  Let $K = \Q(\zeta_3)$ be the third cyclotomic field. \\
  Let $\cc{O}_K = \Z[\zeta_3]$ be the ring of integers of $K$. \\
  Let $\cc{O}^\times_K$ be the group of units of $\cc{O}_K$. \\
  Let $\zeta_3 \in K$ be any primitive third root of unity. \\
  Let $\eta \in \cc{O}_K$ be the element corresponding to $\zeta_3 \in K$. \\
  Let $\lambda \in \cc{O}_K$ be such that $\lambda = \eta -1$. \\
  Let $a, b, c \in \cc{O}_K$ and $u \in \cc{O}^\times_K$ such that $c \neq 0$ and $\gcd(a,b)=1$.\\
  Let $\lambda \notdivides a$, $\lambda \notdivides b$ and $\lambda \divides c$. \\\\
  Then $a^3 + b^3 \neq u c^3$.
\end{theorem}
\begin{proof}
  \leanok
  \uses{lmm:exists_Solution_of_Solution1,
  lmm:exists_minimal,
  lmm:exists_Solution_multiplicity_lt}
  By contradiction we assume that there are $a, b, c \in \cc{O}_K$ and $u \in \cc{O}^\times_K$
  such that $c \neq 0$, $\gcd(a,b)=1$, $\lambda \notdivides a$, $\lambda \notdivides b$,
  $\lambda \divides c$ and $a^3 + b^3 = u c^3$.
  Then $S'=(a,b,c,u)$ is a $solution'$, which implies that there is a $solution$ $S$ by
  \Cref{lmm:exists_Solution_of_Solution1}.
  Then, by \Cref{lmm:exists_minimal}, there is a minimal solution $S_0$ with multiplicity $n$.
  Hence, there is a $solution'$ $S_1'$ with multiplicity $m<n$ by \Cref{lmm:exists_Solution_multiplicity_lt},
  which implies that there is a $solution$ $S_1$  with multiplicity $m$ by \Cref{lmm:exists_Solution_of_Solution1}.
  However, this contradicts the minimality of $S_0$
  forcing us to conclude that $a^3 + b^3 \neq u c^3$.
\end{proof}

\begin{lemma}
  \label{lmm:FermatLastTheoremForThree_of_FermatLastTheoremThreeGen}
  \lean{FermatLastTheoremForThree_of_FermatLastTheoremThreeGen}
  \leanok
  To prove \Cref{thm:fermatLastTheoremThree},
  it suffices to prove \Cref{thm:fermatLastTheoremForThreeGen}. \\
  Equivalently, \Cref{thm:fermatLastTheoremForThreeGen} implies
  \Cref{thm:fermatLastTheoremThree}.
\end{lemma}
\begin{proof}
  \leanok
  \uses{
  thm:fermatLastTheoremThree_of_three_dvd_only_c,
  lmm:norm_lambda_prime,
  lmm:norm_lambda,
  lmm:lambda_dvd_three}
  Assume that $\forall a, b, c \in \cc{O}_K,\, \forall u \in \cc{O}^\times_K$ such that $c \neq 0$,
  $\gcd(a,b)=1$, $\lambda \notdivides a$, $\lambda \notdivides b$ and $\lambda \divides c$,
  it holds that $a^3 + b^3 \neq u c^3$.
  Let $a, b, c \in \Z$ such that $a\neq 0$, $b\neq 0$ and $c\neq 0$.
  By \Cref{thm:fermatLastTheoremThree_of_three_dvd_only_c}, we can assume that
  $\gcd(a,b)=1$, $3 \notdivides a$, $3 \notdivides b$, $3 \divides c$.
  By contradiction we assume that $a^3 + b^3 = c^3$ and let $u = 1$.
  \begin{itemize}
      \item By contradiction we assume that $\lambda \divides a$, which implies that the norm of
      $\lambda$ divides $a$ by \Cref{lmm:norm_lambda_prime}, which implies that $3 \divides a$ by
      \Cref{lmm:norm_lambda}, that contradicts the assumption that $3 \notdivides a$ forcing us
      to conclude that $\lambda \notdivides a$.
      \item By contradiction we assume that $\lambda \divides b$, which implies that the norm of
      $\lambda$ divides $b$ by \Cref{lmm:norm_lambda_prime}, which implies that $3 \divides b$ by
      \Cref{lmm:norm_lambda}, that contradicts the assumption that $3 \notdivides b$ forcing us
      to conclude that $\lambda \notdivides b$.
      \item $\lambda \divides 3$ by \Cref{lmm:lambda_dvd_three} and $3 \divides c$,
      then $\lambda \divides c$.
  \end{itemize}
  By our first assumption $a^3 + b^3 \neq u c^3 = 1 c^3 = c^3 = a^3 + b^3$ which is absurd.
\end{proof}

\section{Conclusion}
% A FEW CONCLUDING WORDS / REMARKS

\begin{theorem}[Fermat's Last Theorem for Exponent $3$]
    \label{thm:fermatLastTheoremThree}
    \lean{fermatLastTheoremThree}
    \leanok
    Let $a, b, c \in \N$. \\
    Let $a \neq 0$, $b \neq 0$ and $c \neq 0$. \\\\
    Then $a^3 + b^3 \neq c^3$.
\end{theorem}
\begin{proof}
    \leanok
    \uses{
    lmm:FermatLastTheoremForThree_of_FermatLastTheoremThreeGen,
    thm:fermatLastTheoremForThreeGen}
    Apply \Cref{lmm:FermatLastTheoremForThree_of_FermatLastTheoremThreeGen}
    and \Cref{thm:fermatLastTheoremForThreeGen}.
\end{proof}
% Acknowledgements
%% \input{chapters/3-acknowledgements}

% Bibliography
\nocite{*}
\printbibliography[title=References]
\addcontentsline{toc}{chapter}{References}
\end{document}